% Tugas 3 Kelompok 4
% Akbar Pambudi Utomo (1154094)
% Julham Ramadhana (1154069)
% Pebridayanti Hasibuan (1154118)
\section{Basic Syntax}
\subsection{Pengenal Python}
Pengenal Python Pengenal Python adalah nama yang digunakan untuk mengidentifikasi variabel, fungsi, kelas, modulatauobjeklainnya. PengenaldimulaidenganhurufA sampai Z atau huruf a sampai z atau garis bawah \(_) diikuti oleh nol atau lebih huruf, garis bawah dan angka \(0 sampai 9). Python tidak mengizinkan karakter tanda baca seperti \@, \$, dan \% dalam pengenal. Python adalah bahasa pemrograman yang sensitif. Dengan demikian, Tenaga Kerja dan Tenaga Kerja adalah dua pengidentifikasi yang berbeda dengan Python. Berikut adalah konvensi penamaan untuk pengenal Python - Nama kelas dimulai dengan huruf besar. Semua pengenal lainnya mulai dengan huruf kecil. Memulai pengenal dengan satu garis bawah terkemuka menunjukkan bahwa pengenal bersifat pribadi. Memulai pengenal dengan dua garis bawah terkemuka menunjukkan pengenal yang sangatpribadi. Jika pengenal juga diakhiri dengan dua tanda garis bawah, identifier adalah nama khusus yang ditentukan bahasa. 
Bahasa Python memiliki banyak kesamaan dengan Perl, C, dan Java. Namun, ada beberapa perbedaan yang pasti antara bahasa. Program Python Pertama Mari kita jalankan program dalam mode pemrograman yang berbeda. Pemrograman Mode Interaktif Memohon interpreter tanpa melewatkan file script sebagai parameter menampilkan prompt berikut :
	\begin{verbatim}
		Python
		Python 2.4.3 ( #1, Nov 11 2010, 13:34:43) 
		[GCC 4.1.2 20080704 (Red Hat 4.1.2-48)] on linux2 Type ”help”, ”copyright”, ”credits” or ”license” for more information.

		Ketik teks berikut pada prompt Python dan tekan Enter:
		
		print ”Hello, Python!”
	\end{verbatim}

Jika Anda menjalankan versi baru Python, Anda perlu menggunakan pernyataan cetak dengan tanda kurung seperti pada cetak \verb|(”Halo,Python!”);|. Namun dengan versi Python 2.4.3, ini menghasilkan hasil sebagai berikut:

Hello, Python!

\subsection{Pemrograman Mode Script}
Memohon interpreter dengan parameter script memulai eksekusi script dan berlanjut sampai script selesai. Saat skrip selesai, juru bahasa tidak lagi aktif. Mari kita tuliskan program Python sederhana dalam sebuah naskah. File Python memiliki ekstensi .py. Ketik kode sumber berikut di file.
Objek Dengan Python, seperti semua bahasa berorientasi objek, ada kumpulan kode dan data yang disebut objek, yang biasanya mewakili potongan dalam model konseptual suatu sistem. Objek dengan Python dibuat (yaitu, instantiated) dari template yang disebut kelas (yang akan dibahas kemudian, sebanyak bahasa dapat digunakan tanpa memahami kelas). Mereka memiliki atribut, yang mewakili berbagai potongan kode dan data yang membentuk objek. Untuk mengakses atribut, seseorang menuliskan nama objek yang diikuti oleh suatu periode (selanjutnya disebut titik), diikuti dengan nama atribut.
Contohnya adalah atribut ’atas’ dari string, yang mengacu pada kode yang mengembalikan salinan string di mana semua huruf adalah huruf besar. Untuk mendapatkan ini, perlu untuk memiliki cara untuk merujuk ke objek (dalam contoh berikut, jalan adalah string literal yang membangun objek).

\subsection{Pengertian Python}
Python adalah salah satu pemrograman yang terus berkembang dan bertahan dikarenakan dukungan komunitas yang aktif diseluruh dunia. Banyak forum-forum ataupun blogger-blogger yang sering membagi pengalaman dalam menggunakan python. Hal ini memudahkan bagi pengguna pemula maupun pengembang untuk bertanya dan sharing tentang ilmu pemrograman ini. 

\subsubsection{Kelebihan dan Kekurangan Python}
Kelebihan yang dimiliki oleh Python :
	\begin{itemize}
		\item Tidak ada tahapan kompilasi dan penyambungan (link) sehinggakecepatanperubahanpadamasapembuatansistem aplikasi 	meningkat. 
		\item Tidak ada deklarasi tipe data yang merumitkan sehingga program menjadi lebih sederhana, singkat, dan fleksible. 
		\item Manajemen memori otomatis yaitu kumpulan sampah memori sehingga dapat menghindari pencacatan kode. 
		\item Tipe data dan operasi tingkat tinggi yaitu kecepatan pembuatan sistem aplikasi menggunakan tipe objek yang telah ada. 
		\item Pemrograman berorientasi objek.
		\item Pelekatan dan perluasan dalam C. 
		\item Terdapat kelas, modul, eksepsi sehingga terdapat dukungan pemrograman skala besar secara modular.
		\item Pemuatan dinamis modul C sehingga ekstensi menjadi sederhana dan berkas biner yang kecil 
		\item Pemuatan kembali secara dinamis modul phyton seperti memodifikasi aplikasi tanpa menghentikannya. 
		\item Model objek universal kelas Satu. 
		\item Konstruksi pada saat aplikasi berjalan.
		\item Interaktif, dinamis dan alamiah. 
		\item Akses hingga informasi interpreter. 
		\item Portabilitas secara luas seperti pemrograman antar platform tanpa ports. 
		\item Kompilasi untuk portable kode byte sehingga kecepatan eksekusi bertambah dan melindungi kode sumber.
		\item Antarmuka terpasang untuk pelayanan keluar seperti perangkat Bantu system, GUI, persistence, database, dll. 
	\end{itemize}
Kekurangan yang dimiliki Python :
	\begin{itemize}
		\item Beberapa penugasan terdapat diluar dari jangkauan python, seperti bahasa pemrograman dinamis lainnya, python tidak secepat atau efisien sebagai statis, tidak seperti bahasa pemrograman kompilasi seperti bahasa C. 
		\item Disebabkan python merupakan interpreter, python bukan merupakan perangkat bantu terbaik untuk pengantar komponen performa kritis. 
		\item Python tidak dapat digunakan sebagai dasar bahasa pemrograman implementasi untuk beberapa komponen, tetapi dapat bekerja dengan baik sebagai bagian depan skrip antarmuka untuk mereka. 
		\item Python memberikan efisiensi dan fleksibilitas tradeoff by dengan tidak memberikannya secara menyeluruh. Python menyediakan bahasa pemrograman optimasi untuk kegunaan, bersama dengan perangkat bantu yang dibutuhkan untuk diintegrasikan dengan bahasa pemrograman lainnya.
		\item Banyak terdapat referensi lama terutama dari pencarian google, python adalah pemrograman yang sangat lambat. Namun belum lama ini ditemukan bahwa Google, Youtube, DropBox dan beberapa software sistem banyak menggunakan Python.
	\end{itemize}
Dibalik kelebihan dan kekurangan yang dimiliki, Kini Python menjadi salah satu bahasa pemrograman yang populer digunakan oleh pengembangan web, aplikasi web, aplikasi perkantoran, simulasi, dan masih banyak lagi. Hal ini disebabkan karena Python bahasa pemrograman yang dinamis dan mudah dipahami. Python memiliki hak cipta. Seperti Perl, kode sumber Python sekarang tersedia di bawah GNU General Public License (GPL). Python sekarang dikelola oleh tim pengembangan inti di institut tersebut, walaupun Guido van Rossum masih memegangperanpentingdalammengarahkankemajuannya.

\subsubsection{Fitur Python}
Python memiliki fitur yang meliputi:
	\begin{enumerate}
		\item Mudah dipelajari: Python memiliki beberapa kata kunci, struktur sederhana, dan sintaks yang jelas. Hal ini memungkinkan siswa untuk mengambil bahasa dengan cepat.
		\item Mudah dibaca: kode Python lebih jelas dan terlihat oleh mata.
		\item Mudahdipelihara: kodesumberPythoncukupmudahuntuk dipelihara.
		\item Perpustakaan standar yang luas: sebagian besar perpustakaan Python sangat portabel dan kompatibel dengan platformcross-platformdiUNIX,Windows,danMacintosh. 
		\item Mode Interaktif: Python memiliki dukungan untuk mode interaktif yang memungkinkan pengujian interaktif dan debugging dari cuplikan kode.
		\item Portable: Python dapat berjalan di berbagai platform perangkat keras dan memiliki antarmuka yang sama pada semua platform.
		\item Dapat diperpanjang: Anda dapat menambahkan modul tingkat rendah ke penerjemah Python. Modul ini memungkinkan programmer untuk menambahkan atau menyesuaikan alat mereka agar lebih efisien.
		\item Database: Python menyediakan antarmuka untuk semua database komersial utama. 
	\end{enumerate}

\subsection {Pernyataan Multi-Line}
Pernyataan di Python biasanya diakhiri dengan baris baru. Python, bagaimanapun, memungkinkan penggunaan karakter kelanjutan baris (\) untuk menunjukkan bahwa garis tersebut harus dilanjutkan. Misalnya –
	\begin {equation}
	total = item_one + \
		item_two + \
		item_three
	\end {equation}

Pernyataan yang ada di dalam kurung [], {}, atau () tidak perlu menggunakan karakter kelanjutan baris. Misalnya –
	\begin {equation}
	days = ['Monday', 'Tuesday', 'Wednesday',
		'Thursday', 'Friday']
	\end {equation}

\subection{Kutipan pada Python}
Python menerima kutipan tunggal ('), ganda (") dan triple (' '' atau '" ") untuk menunjukkan literal string, selama jenis kutipan yang sama dimulai dan mengakhiri string.
Tanda kutip triple digunakan untuk membentang string di beberapa baris. Misalnya:
	\begin {equation}
	word = 'word'
	sentence = "This is a sentence."
	paragraph = """This is a paragraph. It is
	made up of multiple lines and sentences."""
	\end {equation}

\subsection{Komentar pada Python}
Tanda hash (#) yang tidak berada di dalam string literal memulai sebuah komentar. Semua karakter setelah # dan sampai akhir garis fisik adalah bagian dari komentar dan penafsir Python mengabaikannya.
	\begin {equation}
	#!/usr/bin/python

	# First comment
	print \"Hello, Python!\" # second comment
	\end {equation}

	 Dan hasilnya sebagai berikut –
	\begin {equation}
	    Hello, Python!
	\end {equation}

Anda bisa mengetikkan komentar di baris yang sama setelah sebuah pernyataan atau ungkapan –
	\begin {equation}
	    name = \"Madisetti\" # This is again comment
	\end {equation}

Anda dapat mengomentari beberapa baris sebagai berikut –
	\begin {equation}
	# This is a comment.
	# This is a comment, too.
	# This is a comment, too.
	# I said that already.
	\end {equation}

\subsubsection{Variabel}	
 Jadi variable itu untuk menyimpan data sebagai penampung nya dan nilai adalah isi dari variable itu sendiri.
Contoh variable katakana saja my int = 7 itu juga bisa di ubah menjadi my int =3  
Code merubah ni;ai variable nya  
	My int = 7
	My int = 
	Print my int 

\subsubsection{Whitespace} 
Whitespace adalah penyusun kode untuk penulisan oleh karena itu harus berhati hati karena sangat sensitive
Contoh code nya: 
 	Def spam() 
	Eggs + 12 
 	Return eggs 
 	Print spam() 
Itu merupakan kode penulisan yang salah maka akan muncul pesan error.
Pesan error yang muncul tentang penulisan dan bisa di edit agar tidak terjadi kesalahan penulisan lagi.
Contoh yang benar :  
 	Def spam(): 
 	Eggs = 12 
 	Return eggs 
 	Print spam() 
Kita coba contohkan penulisan oprasi matematika nya.
Code nya adalah: 
	Jumlah = 10+10 
	Print jumlah 
	Pejumlahan = 72 + 23 
	Pengurangan 108 – 204 
	Perkalian = 108 * 0,5 
	Pembagian = 108 / 9 
Maka akan muncul print yang sesuai inputan yang kalian buat. 
Selain itu bisa juga perpangkatan 

Kalkulator pangkatan meggunakan eight 
Contohnya eight = \2** 3\ 
Eight itu adalah variable baru yang kita buat untuk mengeset nilai menjadi 8.

Contoh: stopwatch sederhana yang biasa nya di gunakan untuk ujian peraktek.  
Penjelasan tentang stopwatch yaitu alat untuk mengukur kecepatan suatu benda secara akurat. 
Sebelum membuat nya kita harus mengetahui cara kerja stopwatch itu sendiri. Memulai angka nya dari 0 karena angka 0 adalah angka pertama dalam waktu.
Code nya antara lain:
	Depanjam = 0 
	Jam = 0 
	Depanmenit = 0 
	Menit = 0 
	Depandetik = 0 
	Detik = 0 
Lalu gunakan import time 
Untuk system loop jadi setelah detik mencapai 9 maka menit akan bertambah 1 seperti itu. 
	While true : 
	Time.sleep(1) 
	Detik += 1 
	If detik == 9: 
	Detik = 0 
	Depandetik += 1 
	If depandetik == 6:
	Menit +=1 
	Depandetik = 0 
	Detik = 0 
	If menit == 9: 
	Menit = 0 
	Depanmenit += 1 
	If depanmenit == 6: 
	Jam +=1 
	Depanmenit = 0 
	Menit = 0 
	If ja, == 9 : 
	Depanjam += 1 
	Jam = 0 
Print ( $ " $ $  \{  $0 $  \}  $ $  \{  $1 $  \}  $ $  \{  $2 $  \}  $ $  \{  $3 $  \}  $ $  \{  $4 $  \}  $ $  \{  $5 $  \vert  $ $  \}  $ $ " $. Format(depanjam,jam,depanmenit,menit,depandetik,detik), end= $ " $ $  \setminus  $r $ " $) 
Run maka akan jadi program simpe tentang stopwatch sederhana ini.
