%%%%%%%%%%%%  Generated using www.docx2latex.com  %%%%%%%%%%%%%%


\documentclass[a4paper,12pt]{report}

% Other options in place of 'article' are 1)report 2)book 3)letter
% Other options in place of 'a4paper' are 1)a5paper 2)b5paper 3)letterpaper 4)legalpaper 5)executivepaper


 %%%%%%%%%%%%  Include Packages  %%%%%%%%%%%%%%


\usepackage{amsmath}
\usepackage{latexsym}
\usepackage{amsfonts}
\usepackage{amssymb}
\usepackage{graphicx}
\usepackage{txfonts}
\usepackage{wasysym}
\usepackage{enumitem}
\usepackage{adjustbox}
\usepackage{ragged2e}
\usepackage{tabularx}
\usepackage{changepage}
\usepackage{setspace}
\usepackage{hhline}
\usepackage{multicol}
\usepackage{float}
\usepackage{multirow}
\usepackage{makecell}
\usepackage{fancyhdr}
\usepackage[toc,page]{appendix}
\usepackage[utf8]{inputenc}
\usepackage[T1]{fontenc}
\usepackage{hyperref}


 %%%%%%%%%%%%  Define Colors For Hyperlinks  %%%%%%%%%%%%%%


\hypersetup{
colorlinks=true,
linkcolor=blue,
filecolor=magenta,
urlcolor=cyan,
}
\urlstyle{same}


 %%%%%%%%%%%%  Set Depths for Sections  %%%%%%%%%%%%%%

% 1) Section
% 1.1) SubSection
% 1.1.1) SubSubSection
% 1.1.1.1) Paragraph
% 1.1.1.1.1) Subparagraph


\setcounter{tocdepth}{5}
\setcounter{secnumdepth}{5}


 %%%%%%%%%%%%  Set Page Margins  %%%%%%%%%%%%%%


\usepackage[a4paper,bindingoffset=0.2in,headsep=0.5cm,left=1.0in,right=1.0in,bottom=2cm,top=2cm,headheight=2cm]{geometry}
\everymath{\displaystyle}


 %%%%%%%%%%%%  Set Depths for Nested Lists created by \begin{enumerate}  %%%%%%%%%%%%%%


\setlistdepth{9}
\newlist{myEnumerate}{enumerate}{9}
	\setlist[myEnumerate,1]{label=\arabic*)}
	\setlist[myEnumerate,2]{label=\alph*)}
	\setlist[myEnumerate,3]{label=(\roman*)}
	\setlist[myEnumerate,4]{label=(\arabic*)}
	\setlist[myEnumerate,5]{label=(\Alph*)}
	\setlist[myEnumerate,6]{label=(\Roman*)}
	\setlist[myEnumerate,7]{label=\arabic*}
	\setlist[myEnumerate,8]{label=\alph*}
	\setlist[myEnumerate,9]{label=\roman*}

\renewlist{itemize}{itemize}{9}
	\setlist[itemize]{label=$\cdot$}
	\setlist[itemize,1]{label=\textbullet}
	\setlist[itemize,2]{label=$\circ$}
	\setlist[itemize,3]{label=$\ast$}
	\setlist[itemize,4]{label=$\dagger$}
	\setlist[itemize,5]{label=$\triangleright$}
	\setlist[itemize,6]{label=$\bigstar$}
	\setlist[itemize,7]{label=$\blacklozenge$}
	\setlist[itemize,8]{label=$\prime$}



 %%%%%%%%%%%%  Header here  %%%%%%%%%%%%%%


\pagestyle{fancy}
\fancyhf{}


 %%%%%%%%%%%%  Footer here  %%%%%%%%%%%%%%




 %%%%%%%%%%%%  Print Page Numbers  %%%%%%%%%%%%%%


\rfoot{\thepage}


 %%%%%%%%%%%%  This sets linespacing (verticle gap between Lines) Default=1 %%%%%%%%%%%%%%


\setstretch{1.08}


 %%%%%%%%%%%%  Document Code starts here %%%%%%%%%%%%%%


\begin{document}
\sloppy
{\fontsize{14pt}{14pt}\selectfont \vspace{\baselineskip}
Basic Syntax \\} \par
\vspace{14pt}
\noindent 
{\fontsize{14pt}{14pt}\selectfont Contoh program selamat dating dengan peraturan syntax dalam python tersebut  \\} \par
\vspace{14pt}
\noindent 
{\fontsize{14pt}{14pt}\selectfont >>> print  $ " $selamat dating di pemrograman python $ " $ \\} \par
\vspace{14pt}
\noindent 
{\fontsize{14pt}{14pt}\selectfont Hasil nya adalah  \\} \par
\vspace{14pt}
\noindent 
{\fontsize{14pt}{14pt}\selectfont Selamat dating di pemrograman python  \\} \par
\vspace{14pt}
\noindent 
{\fontsize{14pt}{14pt}\selectfont Fungsi print tersebut bisa kita lihat dan kita pahami dengan mengerti fungsi input dan ouy put yang sangat penting dalam bahasa pemrograman. Untuk keluaran kita dapat menggunakan dengan print sedangkan untuk masukan nya menggunakan raw $  \_  $print(). Itu masukan berupa teks contohnya seperti. \\} \par
\vspace{14pt}
\noindent 
{\fontsize{14pt}{14pt}\selectfont >>> nama + raw $  \_  $input( $ " $ketkkan nama lengkap anda: $ " $) \\} \par
\vspace{14pt}
\noindent 
{\fontsize{14pt}{14pt}\selectfont Perintah ini akan menampilkan sebuah pesan yang meminta masukan nama anda. \\} \par
\vspace{14pt}
\noindent 
{\fontsize{14pt}{14pt}\selectfont Ketikakan nama lengkap anda : asep kumis \\} \par
\vspace{14pt}
\noindent 
{\fontsize{14pt}{14pt}\selectfont Sekarang saat nya kita akan menampilkan nama yang sudah kita masukan tadi dengan menggunakan perintah print \\} \par
\vspace{14pt}
\noindent 
{\fontsize{14pt}{14pt}\selectfont >>>print  $ " $nama anda adalah $ " $, nama \\} \par
\vspace{14pt}
\noindent 
{\fontsize{14pt}{14pt}\selectfont Hasilnya akan memanggil keluaran yang telah di masukan tadi \\} \par
\vspace{14pt}
\noindent 
{\fontsize{14pt}{14pt}\selectfont Identifier di python  \\} \par
\vspace{14pt}
\noindent 
{\fontsize{14pt}{14pt}\selectfont Pengenalan dam variable, module, vunction, class ada beberapa aturan untuk digunakan yaitu menggunakan huruf capital dari A sampai Z atau bisa juga di awali dengan huruf a kecil sampai z bisa juga di awali dengan symbol seperti underscore  $  \_  $ kemudian bisa di ikuti dengan 0 sampai 9 \\} \par
\vspace{14pt}
\noindent 
{\fontsize{14pt}{14pt}\selectfont Tidak boleh menggunkan symbol yang sepertinya memingungkan seoerti @, $  \%  $, $  \$  $ danlain lain \\} \par
\vspace{14pt}
\noindent 
{\fontsize{14pt}{14pt}\selectfont Kenapa banyak peraturan karena pemrograman python sangat lah sensitive apabila ada kesalah kecil pun untuk itu diperlukan peraturan untuk meminimalisir kesalahan dalam penulisan juga. \\} \par
\vspace{14pt}
\noindent 
{\fontsize{14pt}{14pt}\selectfont Contoh dalam sebuah program \\} \par
\vspace{14pt}
\noindent 
{\fontsize{14pt}{14pt}\selectfont >>> a = 1 + 2 \\} \par
\vspace{14pt}
\noindent 
{\fontsize{14pt}{14pt}\selectfont >>> print (a)  \\} \par
\vspace{14pt}
\noindent 
{\fontsize{14pt}{14pt}\selectfont 3 \\} \par
\vspace{14pt}
\noindent 
{\fontsize{14pt}{14pt}\selectfont >>> \\} \par
\noindent 
{\fontsize{14pt}{14pt}\selectfont Nah yang di kurung adalah huruf a itu gunanya untuk menampung hasil dari jumlah 1 + 2.  \\} \par
\vspace{14pt}
\noindent 
{\fontsize{14pt}{14pt}\selectfont Reserved word  \\} \par
\vspace{14pt}
\noindent 
{\fontsize{14pt}{14pt}\selectfont Kata cadangan salah satu istilah bahasa pemrograman yang menunjukan beberapa kata yang tidak boleh di gunakan sebagai identifier \\} \par
\vspace{14pt}
\noindent 
{\fontsize{14pt}{14pt}\selectfont Ini adalah macam macam reserved word yang ada di python \\} \par
\vspace{14pt}
\noindent 
{\fontsize{14pt}{14pt}\selectfont And  \hspace*{0.5in}  \hspace*{0.5in} assert \hspace*{0.5in}  \hspace*{0.5in} break \hspace*{0.5in}  \hspace*{0.5in} class \hspace*{0.5in}  \hspace*{0.5in} continue \\} \par
\vspace{14pt}
\noindent 
{\fontsize{14pt}{14pt}\selectfont Def \hspace*{0.5in}  \hspace*{0.5in} del \hspace*{0.5in}  \hspace*{0.5in} elif \hspace*{0.5in}  \hspace*{0.5in} else \hspace*{0.5in}  \hspace*{0.5in} except \\} \par
\noindent 
{\fontsize{14pt}{14pt}\selectfont Finally \hspace*{0.5in} for \hspace*{0.5in}  \hspace*{0.5in} from  \hspace*{0.5in}  \hspace*{0.5in} global \hspace*{0.5in}  \hspace*{0.5in} if \\} \par
\vspace{14pt}
\noindent 
{\fontsize{14pt}{14pt}\selectfont Cara menampilkan komentar kolom atau komentar dalam program \\} \par
\vspace{14pt}
\noindent 
{\fontsize{14pt}{14pt}\selectfont Contohnya seperti  \\} \par
\vspace{14pt}
\noindent 
{\fontsize{14pt}{14pt}\selectfont >>>  $  \#  $ini adalah sebuah komentar \\} \par
\vspace{14pt}
\noindent 
{\fontsize{14pt}{14pt}\selectfont >>> $  \#  $komentar tidak akan dieksekusi oleh program \\} \par
\vspace{14pt}
\noindent 
{\fontsize{14pt}{14pt}\selectfont >>> print ( $ " $belajar python) \\} \par
\vspace{14pt}
\noindent 
{\fontsize{14pt}{14pt}\selectfont Belajar python \\} \par
\vspace{14pt}
\noindent 
{\fontsize{14pt}{14pt}\selectfont >>> $  \#  $mencetak tulisan belajar python di layar computer \\} \par
\noindent 
{\fontsize{14pt}{14pt}\selectfont >>> Print (456) \\} \par
\noindent 
{\fontsize{14pt}{14pt}\selectfont 456 \\} \par
\noindent 
{\fontsize{14pt}{14pt}\selectfont >>> $  \#  $mencetak angka 456 di layar computer \\} \par
\vspace{14pt}
\vspace{14pt}
\vspace{14pt}
\noindent 
{\fontsize{14pt}{14pt}\selectfont Pengenal Python Pengenal Python adalah nama yang digunakan untuk mengidentifikasi variabel, fungsi, kelas, modul atau objek lainnya. Pengenal dimulai dengan huruf A sampai Z atau huruf a sampai z atau garis bawah ( $  \_  $) diikuti oleh nol atau lebih huruf, garis bawah dan angka (0 sampai 9). Python tidak mengizinkan karakter tanda baca seperti @,  $  \$  $, dan $  \%  $ dalam pengenal. Python adalah bahasa pemrograman yang sensitif. Dengan demikian, Tenaga Kerja dan Tenaga Kerja adalah dua pengidentifikasi yang berbeda dengan Python. Berikut adalah konvensi penamaan untuk pengenal Python - Nama kelas dimulai dengan huruf besar. Semua pengenal lainnya mulai dengan huruf kecil. Memulai pengenal dengan satu garis bawah terkemuka menunjukkan bahwa pengenal bersifat pribadi. Memulai pengenal dengan dua garis bawah terkemuka menunjukkan pengenal yang sangat pribadi. Jika pengenal juga diakhiri dengan dua tanda garis bawah, identifier adalah nama khusus yang ditentukan bahasa. \\} \par
\vspace{14pt}
\noindent 
{\fontsize{14pt}{14pt}\selectfont \vspace{\baselineskip}
Bahasa Python memiliki banyak kesamaan dengan Perl, C, dan Java. Namun, ada beberapa perbedaan yang pasti antara bahasa. Program Python Pertama Mari kita jalankan program dalam mode pemrograman yang berbeda. Pemrograman Mode Interaktif Memohon interpreter tanpa melewatkan file script sebagai parameter menampilkan prompt berikut  \\} \par
\vspace{14pt}
\noindent 
{\fontsize{14pt}{14pt}\selectfont Ada beberapa contoh yang akan di paparkan yaitu cara merubah nilai \\} \par
\noindent 
{\fontsize{14pt}{14pt}\selectfont Penulis akan menjelaskan secara ringkas tentang variable dan nilai oleh karena itu pahami tentang variable tertentu agar nilai nya valid \\} \par
\vspace{14pt}
\noindent 
{\fontsize{14pt}{14pt}\selectfont Jadi variable itu untuk menyimpan data sebagai penampung nya dan nilai adalah isi dari variable itu sendiri \\} \par
\vspace{14pt}
\noindent 
{\fontsize{14pt}{14pt}\selectfont Contoh variable katakana saja my $  \_  $int = 7 itu juga bisa di ubah menjadi my $  \_  $int =3  \\} \par
\vspace{14pt}
\noindent 
{\fontsize{14pt}{14pt}\selectfont Code merubah ni;ai variable nya  \\} \par
\noindent 
{\fontsize{14pt}{14pt}\selectfont My $  \_  $int = 7 \\} \par
\noindent 
{\fontsize{14pt}{14pt}\selectfont My $  \_  $int = \\} \par
\noindent 
{\fontsize{14pt}{14pt}\selectfont Print my $  \_  $int \\} \par
\vspace{14pt}
\noindent 
{\fontsize{14pt}{14pt}\selectfont Ada yang dinamakan whitespace  \\} \par
\vspace{14pt}
\noindent 
{\fontsize{14pt}{14pt}\selectfont Whitespace adalah penyusun kode untuk penulisan oleh karena itu harus berhati hati karena sangat sensitive \\} \par
\vspace{14pt}
\noindent 
{\fontsize{14pt}{14pt}\selectfont Contoh code nya  \\} \par
\noindent 
{\fontsize{14pt}{14pt}\selectfont Def spam() \\} \par
\noindent 
{\fontsize{14pt}{14pt}\selectfont Eggs + 12 \\} \par
\noindent 
{\fontsize{14pt}{14pt}\selectfont Return eggs \\} \par
\noindent 
{\fontsize{14pt}{14pt}\selectfont Print spam() \\} \par
\vspace{14pt}
\noindent 
{\fontsize{14pt}{14pt}\selectfont Itu merupakan kode penulisan yang salah maka akan muncul pesan error \\} \par
\vspace{14pt}
\noindent 
{\fontsize{14pt}{14pt}\selectfont Pesan error yang muncul tentang penulisan dan bisa di edit agar tidak terjadi kesalahan penulisan lagi \\} \par
\vspace{14pt}
\noindent 
{\fontsize{14pt}{14pt}\selectfont Contoh yang benar  \\} \par
\vspace{14pt}
\noindent 
{\fontsize{14pt}{14pt}\selectfont Def spam(): \\} \par
\noindent 
{\fontsize{14pt}{14pt}\selectfont Eggs = 12 \\} \par
\noindent 
{\fontsize{14pt}{14pt}\selectfont Return eggs \\} \par
\noindent 
{\fontsize{14pt}{14pt}\selectfont Print spam() \\} \par
\vspace{14pt}
\noindent 
{\fontsize{14pt}{14pt}\selectfont Kita coba contohkan penulisan oprasi matematika nya \\} \par
\vspace{14pt}
\noindent 
{\fontsize{14pt}{14pt}\selectfont Code nya adalah  \\} \par
\vspace{14pt}
\vspace{14pt}
\noindent 
{\fontsize{14pt}{14pt}\selectfont Jumlah = 10+10 \\} \par
\noindent 
{\fontsize{14pt}{14pt}\selectfont Print jumlah \\} \par
\vspace{14pt}
\noindent 
{\fontsize{14pt}{14pt}\selectfont Pejumlahan = 72 + 23 \\} \par
\noindent 
{\fontsize{14pt}{14pt}\selectfont Pengurangan 108 – 204 \\} \par
\noindent 
{\fontsize{14pt}{14pt}\selectfont Perkalian = 108 * 0,5 \\} \par
\noindent 
{\fontsize{14pt}{14pt}\selectfont Pembagian = 108 / 9 \\} \par
\vspace{14pt}
\noindent 
{\fontsize{14pt}{14pt}\selectfont Maka akan muncul print yang sesuai inputan yang kalian buat \\} \par
\vspace{14pt}
\noindent 
{\fontsize{14pt}{14pt}\selectfont Selain itu bisa juga perpangkatan  \\} \par
\noindent 
{\fontsize{14pt}{14pt}\selectfont Kalkulator pangkatan \\} \par
\noindent 
{\fontsize{14pt}{14pt}\selectfont Meggunakan eight \\} \par
\vspace{14pt}
\noindent 
{\fontsize{14pt}{14pt}\selectfont Contohnya eight = 2** 3 \\} \par
\vspace{14pt}
\noindent 
{\fontsize{14pt}{14pt}\selectfont Eight itu adalah variable baru yang kita buat untuk mengeset nilai menjadi 8 \\} \par
\vspace{14pt}
\noindent 
{\fontsize{14pt}{14pt}\selectfont Contoh stopwatch sederhana yang biasa nya di gunakan untuk ujian peraktek  \\} \par
\vspace{14pt}
\noindent 
{\fontsize{14pt}{14pt}\selectfont Penjelasan tentang stopwatch yaitu alat untuk mengukur kecepatan suatu benda secara akurat \\} \par
\vspace{14pt}
\noindent 
{\fontsize{14pt}{14pt}\selectfont Sebelum membuat nya kita harus mengetahui cara kerja stopwatch itu sendiri  \\} \par
\vspace{14pt}
\noindent 
{\fontsize{14pt}{14pt}\selectfont Memulai angka nya dari 0 karena angka 0 adalah angka pertama dalam waktu \\} \par
\noindent 
{\fontsize{14pt}{14pt}\selectfont Code nya antara lain \\} \par
\vspace{14pt}
\noindent 
{\fontsize{14pt}{14pt}\selectfont Depanjam = 0 \\} \par
\noindent 
{\fontsize{14pt}{14pt}\selectfont Jam = 0 \\} \par
\noindent 
{\fontsize{14pt}{14pt}\selectfont Depanmenit = 0 \\} \par
\noindent 
{\fontsize{14pt}{14pt}\selectfont Menit = 0 \\} \par
\noindent 
{\fontsize{14pt}{14pt}\selectfont Depandetik = 0 \\} \par
\noindent 
{\fontsize{14pt}{14pt}\selectfont Detik = 0 \\} \par
\noindent 
{\fontsize{14pt}{14pt}\selectfont  Lalu gunakan import time \\} \par
\noindent 
{\fontsize{14pt}{14pt}\selectfont Untuk system loop jadi setelah detik mencapai 9 maka menit akan bertambah 1 seperti itu. \\} \par
\vspace{14pt}
\noindent 
{\fontsize{14pt}{14pt}\selectfont While true : \\} \par
\noindent 
{\fontsize{14pt}{14pt}\selectfont Time.sleep(1) \\} \par
\noindent 
{\fontsize{14pt}{14pt}\selectfont Detik += 1 \\} \par
\noindent 
{\fontsize{14pt}{14pt}\selectfont If detik == 9: \\} \par
\noindent 
{\fontsize{14pt}{14pt}\selectfont Detik = 0 \\} \par
\noindent 
{\fontsize{14pt}{14pt}\selectfont Depandetik += 1 \\} \par
\noindent 
{\fontsize{14pt}{14pt}\selectfont If depandetik == 6: \\} \par
\noindent 
{\fontsize{14pt}{14pt}\selectfont Menit +=1 \\} \par
\noindent 
{\fontsize{14pt}{14pt}\selectfont Depandetik = 0 \\} \par
\noindent 
{\fontsize{14pt}{14pt}\selectfont Detik = 0 \\} \par
\noindent 
{\fontsize{14pt}{14pt}\selectfont If menit == 9: \\} \par
\noindent 
{\fontsize{14pt}{14pt}\selectfont Menit = 0 \\} \par
\noindent 
{\fontsize{14pt}{14pt}\selectfont Depanmenit += 1 \\} \par
\noindent 
{\fontsize{14pt}{14pt}\selectfont If depanmenit == 6: \\} \par
\noindent 
{\fontsize{14pt}{14pt}\selectfont Jam +=1 \\} \par
\noindent 
{\fontsize{14pt}{14pt}\selectfont Depanmenit = 0 \\} \par
\noindent 
{\fontsize{14pt}{14pt}\selectfont Menit = 0 \\} \par
\noindent 
{\fontsize{14pt}{14pt}\selectfont If ja, == 9 : \\} \par
\noindent 
{\fontsize{14pt}{14pt}\selectfont Depanjam += 1 \\} \par
\noindent 
{\fontsize{14pt}{14pt}\selectfont Jam = 0 \\} \par
\noindent 
{\fontsize{14pt}{14pt}\selectfont Print ( $ " $ $  \{  $0 $  \}  $ $  \{  $1 $  \}  $ $  \{  $2 $  \}  $ $  \{  $3 $  \}  $ $  \{  $4 $  \}  $ $  \{  $5 $  \vert  $ $  \}  $ $ " $. Format(depanjam,jam,depanmenit,menit,depandetik,detik), end= $ " $ $  \textbackslash  $r $ " $) \\} \par
\vspace{14pt}
\noindent 
{\fontsize{14pt}{14pt}\selectfont Run maka akan jadi program simpe tentang stopwatch sederhana ini. \\} \par
\end{document}
