
\sloppy
{\fontsize{14pt}{14pt}\selectfont OVERVIEW \\} \par
\noindent 
{\fontsize{14pt}{14pt}\selectfont Python adalah bahasa script tingkat tinggi, ditafsirkan, interaktif dan berorientasi objek. Python dirancang agar mudah dibaca. Ini menggunakan kata kunci bahasa Inggris sering di mana bahasa lainnya menggunakan tanda baca, dan memiliki konstruksi sintaksis lebih sedikit daripada bahasa lainnya. \\} \par
\vspace{14pt}
\noindent 
{\fontsize{14pt}{14pt}\selectfont Python is interpreted : diproses pada saat runtime oleh interpreter \\} \par
\vspace{14pt}
\noindent 
{\fontsize{14pt}{14pt}\selectfont Tidak perlu untuk mengkompilasi program anda sebelum mengeksekusi itu. Hal ini merupakan mirip dengan php \\} \par
\vspace{14pt}
\noindent 
{\fontsize{14pt}{14pt}\selectfont Python is Interactive: Anda dapat benar-benar duduk di prompt Python dan berinteraksi dengan penafsir langsung untuk menulis program Anda. \\} \par
\noindent 
{\fontsize{14pt}{14pt}\selectfont Python is Object-Oriented: Python mendukung gaya Berorientasi Objek atau teknik pemrograman yang merangkum kode di dalam objek. \\} \par
\noindent 
{\fontsize{14pt}{14pt}\selectfont Python is a Beginner's Language: Python adalah bahasa yang besar untuk programmer tingkat pemula dan mendukung pengembangan berbagai aplikasi dari pengolahan teks sederhana untuk browser WWW untuk game. \\} \par
\vspace{14pt}
\noindent 
{\fontsize{14pt}{14pt}\selectfont Fitur overview dalam python itu adalah \\} \par
\vspace{14pt}
\noindent 
\begin{itemize}
\item {\fontsize{11pt}{11pt}\selectfont \textbf{Easy-to-learn: Python memiliki beberapa kata kunci, struktur sederhana, dan sintaks yang jelas.} $  $Hal ini memungkinkan siswa untuk mengambil bahasa dengan cepat.} \par
\noindent 
\item {\fontsize{11pt}{11pt}\selectfont \textbf{Easy-to-read: kode Python lebih jelas dan terlihat mata.}} \par
\noindent 
\item {\fontsize{11pt}{11pt}\selectfont \textbf{Easy-to-maintain: kode sumber Python cukup mudah-untuk-menjaga.}}\end{itemize}
 \par
\noindent 
{\fontsize{14pt}{14pt}\selectfont A broad standard library: bulk Python perpustakaan sangat portabel dan cross-platform yang kompatibel pada UNIX, Windows, dan Macintosh. \\} \par
\noindent 
{\fontsize{14pt}{14pt}\selectfont Interactive Mode: Python memiliki dukungan untuk mode interaktif yang memungkinkan pengujian interaktif dan debugging dari potongan kode. \\} \par
\noindent 
{\fontsize{14pt}{14pt}\selectfont Portable: Python dapat dijalankan pada berbagai macam platform perangkat keras dan memiliki antarmuka yang sama pada semua platform. \\} \par
\noindent 
{\fontsize{14pt}{14pt}\selectfont Extendable: Anda dapat menambahkan modul tingkat rendah ke interpreter Python.Modul ini memungkinkan programmer untuk menambah atau menyesuaikan alat-alat mereka untuk menjadi lebih efisien. \\} \par
\noindent 
{\fontsize{14pt}{14pt}\selectfont Databases: Python menyediakan antarmuka untuk semua database komersial utama. \\} \par
\noindent 
{\fontsize{14pt}{14pt}\selectfont GUI Programming: Python mendukung aplikasi GUI yang dapat dibuat dan porting ke banyak panggilan sistem, perpustakaan dan sistem jendela, seperti Windows MFC, Macintosh, dan sistem X Window dari Unix. \\} \par
\noindent 
{\fontsize{14pt}{14pt}\selectfont Scalable: Python menyediakan struktur dan dukungan yang lebih baik untuk program besar dari shell scripting. \\} \par
\vspace{14pt}
\vspace{14pt}
\noindent 
{\fontsize{14pt}{14pt}\selectfont Fitur overview terbaik adalah  \\} \par
\noindent 
{\fontsize{14pt}{14pt}\selectfont IT mendukung metode pemrograman fungsional dan terstruktur serta OOP. \\} \par
\noindent 
{\fontsize{14pt}{14pt}\selectfont Hal ini dapat digunakan sebagai bahasa scripting atau dapat dikompilasi untuk byte-kode untuk membangun aplikasi besar. \\} \par
\noindent 
{\fontsize{14pt}{14pt}\selectfont Ini memberikan tingkat tinggi sangat tipe data dinamis dan mendukung memeriksa jenis dinamis. \\} \par
\noindent 
{\fontsize{14pt}{14pt}\selectfont IT mendukung pengumpulan sampah otomatis. \\} \par
\noindent 
{\fontsize{14pt}{14pt}\selectfont Hal ini dapat dengan mudah diintegrasikan dengan C, C ++, COM, ActiveX, CORBA, dan Java. \\} \par
\noindent 
{\fontsize{14pt}{14pt}\selectfont Hal tersebut menjadi terpopuler karena kemudahan bagi programmer yang menjadikan python pemograman terbaik pada tahun 2016  \\} \par
\vspace{14pt}
\noindent 
{\fontsize{14pt}{14pt}\selectfont Beberapa setandarisasi dalam python beriorientasi objek yaitu \\} \par
\vspace{14pt}
\noindent 
{\fontsize{14pt}{14pt}\selectfont Oprasi interface \\} \par
\noindent 
{\fontsize{14pt}{14pt}\selectfont Sejumlah fungsi yang terkait dengan system oprasi \\} \par
\vspace{14pt}
\noindent 
{\fontsize{14pt}{14pt}\selectfont >>> import os \\} \par
\vspace{14pt}
\noindent 
{\fontsize{14pt}{14pt}\selectfont >>> os.getcwd()~~~~~  \\} \par
\noindent 
{\fontsize{14pt}{14pt}\selectfont 'C: $  \textbackslash  $ $  \textbackslash  $Python34' \\} \par
\vspace{14pt}
\noindent 
{\fontsize{14pt}{14pt}\selectfont >>> os.chdir('/server/accesslogs') \\} \par
\vspace{14pt}
\noindent 
{\fontsize{14pt}{14pt}\selectfont >>> os.system('mkdir~today')~   \\} \par
\noindent 
{\fontsize{14pt}{14pt}\selectfont 0 \\} \par
\vspace{14pt}
\noindent 
{\fontsize{14pt}{14pt}\selectfont File wildcard \\} \par
\noindent 
{\fontsize{14pt}{14pt}\selectfont Menyediakan fungsi untuk membuat daftar file dari pencarian direktori wildcard \\} \par
\vspace{14pt}
\noindent 
{\fontsize{14pt}{14pt}\selectfont >>> import glob \\} \par
\vspace{14pt}
\noindent 
{\fontsize{14pt}{14pt}\selectfont >>> glob.glob('*.py') \\} \par
\noindent 
{\fontsize{14pt}{14pt}\selectfont ['primes.py', 'random.py', 'quote.py'] \\} \par
\vspace{14pt}
\noindent 
{\fontsize{14pt}{14pt}\selectfont Parameter baris perintah  \\} \par
\noindent 
{\fontsize{14pt}{14pt}\selectfont Script umum yang sering memanggil parameter baris perintah.  \\} \par
\vspace{14pt}
\noindent 
{\fontsize{14pt}{14pt}\selectfont >>> import sys \\} \par
\vspace{14pt}
\noindent 
{\fontsize{14pt}{14pt}\selectfont >>> print(sys.argv) \\} \par
\noindent 
{\fontsize{14pt}{14pt}\selectfont ['demo.py', 'one', 'two', 'three'] \\} \par
\vspace{14pt}
\noindent 
{\fontsize{14pt}{14pt}\selectfont Redirection dan program pemutusan \\} \par
\noindent 
{\fontsize{14pt}{14pt}\selectfont Untuk menampilkan peringatan dan pesan kesalahan \\} \par
\vspace{14pt}
\noindent 
{\fontsize{14pt}{14pt}\selectfont >>> sys.stderr.write('Warning, log file not found starting a new one $  \textbackslash  $n') \\} \par
\noindent 
{\fontsize{14pt}{14pt}\selectfont Warning, log file not found starting a new one \\} \par
\vspace{14pt}
\noindent 
{\fontsize{14pt}{14pt}\selectfont String \\} \par
\noindent 
{\fontsize{14pt}{14pt}\selectfont Penccokan kompleks dan manipulasi solusi optimal \\} \par
\vspace{14pt}
\noindent 
{\fontsize{14pt}{14pt}\selectfont >>> import re \\} \par
\vspace{14pt}
\noindent 
{\fontsize{14pt}{14pt}\selectfont >>> re.findall(r' $  \textbackslash  $bf[a-z]*', 'which foot or hand fell fastest') \\} \par
\noindent 
{\fontsize{14pt}{14pt}\selectfont ['foot', 'fell', 'fastest'] \\} \par
\vspace{14pt}
\noindent 
{\fontsize{14pt}{14pt}\selectfont >>> re.sub(r'( $  \textbackslash  $b[a-z]+)  $  \textbackslash  $1', r' $  \textbackslash  $1', 'cat in the the hat') \\} \par
\noindent 
{\fontsize{14pt}{14pt}\selectfont 'cat in the hat' \\} \par
\vspace{14pt}
\noindent 
{\fontsize{14pt}{14pt}\selectfont String biasa \\} \par
\noindent 
{\fontsize{14pt}{14pt}\selectfont >>> 'tea for too'.replace('too', 'two') \\} \par
\noindent 
{\fontsize{14pt}{14pt}\selectfont 'tea for two' \\} \par
\vspace{14pt}
\noindent 
{\fontsize{14pt}{14pt}\selectfont Matematika  \\} \par
\noindent 
{\fontsize{14pt}{14pt}\selectfont Point penghitungan matematika \\} \par
\vspace{14pt}
\noindent 
{\fontsize{14pt}{14pt}\selectfont >>> import math \\} \par
\vspace{14pt}
\noindent 
{\fontsize{14pt}{14pt}\selectfont >>> math.cos(math.pi / 4) \\} \par
\noindent 
{\fontsize{14pt}{14pt}\selectfont 0.70710678118654757 \\} \par
\vspace{14pt}
\noindent 
{\fontsize{14pt}{14pt}\selectfont >>> math.log(1024, 2) \\} \par
\noindent 
{\fontsize{14pt}{14pt}\selectfont 10.0 \\} \par
\vspace{14pt}
\noindent 
{\fontsize{14pt}{14pt}\selectfont Code acak matematika \\} \par
\vspace{14pt}
\noindent 
{\fontsize{14pt}{14pt}\selectfont >>> import random \\} \par
\vspace{14pt}
\noindent 
{\fontsize{14pt}{14pt}\selectfont >>> random.choice(['apple', 'pear', 'banana']) \\} \par
\noindent 
{\fontsize{14pt}{14pt}\selectfont 'apple' \\} \par
\vspace{14pt}
\noindent 
{\fontsize{14pt}{14pt}\selectfont >>> random.sample(range(100),~10)~   $  \#  $ sampling without replacement \\} \par
\noindent 
{\fontsize{14pt}{14pt}\selectfont [30, 83, 16, 4, 8, 81, 41, 50, 18, 33] \\} \par
\vspace{14pt}
\noindent 
{\fontsize{14pt}{14pt}\selectfont >>> random.random()~~~  $  \#  $ random float \\} \par
\noindent 
{\fontsize{14pt}{14pt}\selectfont 0.17970987693706186 \\} \par
\vspace{14pt}
\noindent 
{\fontsize{14pt}{14pt}\selectfont >>> random.randrange(6)~~~  $  \#  $ random integer chosen from range(6) \\} \par
\noindent 
{\fontsize{14pt}{14pt}\selectfont 4 \\} \par
\vspace{14pt}
\noindent 
{\fontsize{14pt}{14pt}\selectfont Akses internet \\} \par
\noindent 
{\fontsize{14pt}{14pt}\selectfont Memproses data yang di terima dari url untuk mengirim email \\} \par
\vspace{14pt}
\noindent 
{\fontsize{14pt}{14pt}\selectfont >>> from urllib.request import urlopen \\} \par
\vspace{14pt}
\noindent 
{\fontsize{14pt}{14pt}\selectfont >>> for line in urlopen('http://tycho.usno.navy.mil/cgi-bin/timer.pl'): \\} \par
\noindent 
{\fontsize{14pt}{14pt}\selectfont ...~~~~ line~= line.decode('utf-8')   $  \#  $ Decoding the binary data to text. \\} \par
\noindent 
{\fontsize{14pt}{14pt}\selectfont ...~~~~ if~'EST' in line or 'EDT' in line:   $  \#  $ look for Eastern Time \\} \par
\noindent 
{\fontsize{14pt}{14pt}\selectfont ...~~~~~~~~ print(line) \\} \par
\vspace{14pt}
\noindent 
{\fontsize{14pt}{14pt}\selectfont <BR>Nov. 25, 09:43:32 PM EST \\} \par
\vspace{14pt}
\noindent 
{\fontsize{14pt}{14pt}\selectfont >>> import smtplib \\} \par
\vspace{14pt}
\noindent 
{\fontsize{14pt}{14pt}\selectfont >>> server = smtplib.SMTP('localhost') \\} \par
\vspace{14pt}
\noindent 
{\fontsize{14pt}{14pt}\selectfont >>> server.sendmail('soothsayer@example.org', 'jcaesar@example.org', \\} \par
\noindent 
{\fontsize{14pt}{14pt}\selectfont ... """To: jcaesar@example.org \\} \par
\noindent 
{\fontsize{14pt}{14pt}\selectfont ... From: soothsayer@example.org \\} \par
\noindent 
{\fontsize{14pt}{14pt}\selectfont ... \\} \par
\noindent 
{\fontsize{14pt}{14pt}\selectfont ... Beware the Ides of March. \\} \par
\noindent 
{\fontsize{14pt}{14pt}\selectfont ... """) \\} \par
\noindent 
{\fontsize{14pt}{14pt}\selectfont >>> server.quit() \\} \par
\vspace{14pt}
\noindent 
{\fontsize{14pt}{14pt}\selectfont Tanggal dan waktu \\} \par
\noindent 
{\fontsize{14pt}{14pt}\selectfont Datetime yang kompleks \\} \par
\vspace{14pt}
\noindent 
{\fontsize{14pt}{14pt}\selectfont >>>  $  \#  $ dates are easily constructed and formatted \\} \par
\vspace{14pt}
\noindent 
{\fontsize{14pt}{14pt}\selectfont >>> from datetime import date \\} \par
\noindent 
{\fontsize{14pt}{14pt}\selectfont >>> now = date.today() \\} \par
\vspace{14pt}
\noindent 
{\fontsize{14pt}{14pt}\selectfont >>> now \\} \par
\noindent 
{\fontsize{14pt}{14pt}\selectfont datetime.date(2003, 12, 2) \\} \par
\vspace{14pt}
\noindent 
{\fontsize{14pt}{14pt}\selectfont >>> now.strftime(" $  \%  $m- $  \%  $d- $  \%  $y.  $  \%  $d  $  \%  $b  $  \%  $Y is a  $  \%  $A on the  $  \%  $d day of  $  \%  $B.") \\} \par
\noindent 
{\fontsize{14pt}{14pt}\selectfont '12-02-03. 02 Dec 2003 is a Tuesday on the 02 day of December.' \\} \par
\vspace{14pt}
\noindent 
{\fontsize{14pt}{14pt}\selectfont >>>  $  \#  $ dates support calendar arithmetic \\} \par
\vspace{14pt}
\noindent 
{\fontsize{14pt}{14pt}\selectfont >>> birthday = date(1964, 7, 31) \\} \par
\vspace{14pt}
\noindent 
{\fontsize{14pt}{14pt}\selectfont >>> age = now – birthday \\} \par
\vspace{14pt}
\noindent 
{\fontsize{14pt}{14pt}\selectfont >>> age.days \\} \par
\noindent 
{\fontsize{14pt}{14pt}\selectfont 14368 \\} \par
\vspace{14pt}
\noindent 
{\fontsize{14pt}{14pt}\selectfont Data kompresi  \\} \par
\noindent 
{\fontsize{14pt}{14pt}\selectfont data umum untuk pengarsipan dan kompresi format \\} \par
\vspace{14pt}
\noindent 
{\fontsize{14pt}{14pt}\selectfont >>> import zlib \\} \par
\vspace{14pt}
\noindent 
{\fontsize{14pt}{14pt}\selectfont >>> s = b'witch which has which witches wrist watch' \\} \par
\vspace{14pt}
\noindent 
{\fontsize{14pt}{14pt}\selectfont >>> len(s) \\} \par
\noindent 
{\fontsize{14pt}{14pt}\selectfont 41 \\} \par
\vspace{14pt}
\noindent 
{\fontsize{14pt}{14pt}\selectfont >>> t = zlib.compress(s) \\} \par
\vspace{14pt}
\noindent 
{\fontsize{14pt}{14pt}\selectfont >>> len(t) \\} \par
\noindent 
{\fontsize{14pt}{14pt}\selectfont 37 \\} \par
\vspace{14pt}
\noindent 
{\fontsize{14pt}{14pt}\selectfont >>> zlib.decompress(t) \\} \par
\noindent 
{\fontsize{14pt}{14pt}\selectfont b'witch which has which witches wrist watch' \\} \par
\vspace{14pt}
\noindent 
{\fontsize{14pt}{14pt}\selectfont >>> zlib.crc32(s) \\} \par
\noindent 
{\fontsize{14pt}{14pt}\selectfont 226805979 \\} \par
\vspace{14pt}
\noindent 
{\fontsize{14pt}{14pt}\selectfont metric kinerja \\} \par
\noindent 
{\fontsize{14pt}{14pt}\selectfont alat pengukuran yang di sediakan langsung oleh python \\} \par
\vspace{14pt}
\noindent 
{\fontsize{14pt}{14pt}\selectfont >>> from timeit import Timer \\} \par
\vspace{14pt}
\noindent 
{\fontsize{14pt}{14pt}\selectfont >>> Timer('t=a; a=b; b=t', 'a=1; b=2').timeit() \\} \par
\noindent 
{\fontsize{14pt}{14pt}\selectfont 0.57535828626024577 \\} \par
\vspace{14pt}
\noindent 
{\fontsize{14pt}{14pt}\selectfont >>> Timer('a,b = b,a', 'a=1; b=2').timeit() \\} \par
\noindent 
{\fontsize{14pt}{14pt}\selectfont 0.54962537085770791 \\} \par
\vspace{14pt}
\noindent 
{\fontsize{14pt}{14pt}\selectfont uji module  \\} \par
\noindent 
{\fontsize{14pt}{14pt}\selectfont pengembangan perangkat lunak berkualitas tinggi \\} \par
\vspace{12pt}
\noindent 
{\fontsize{14pt}{14pt}\selectfont def average(values): \\} \par
\noindent 
{\fontsize{14pt}{14pt}\selectfont ~~~ """Computes the arithmetic mean of a list of numbers. \\} \par
\vspace{14pt}
\noindent 
{\fontsize{14pt}{14pt}\selectfont ~~~ >>> print(average([20, 30, 70])) \\} \par
\noindent 
{\fontsize{14pt}{14pt}\selectfont ~~~ 40.0 \\} \par
\noindent 
{\fontsize{14pt}{14pt}\selectfont ~~~ """ \\} \par
\noindent 
{\fontsize{14pt}{14pt}\selectfont ~~~ return sum(values) / len(values) \\} \par
\vspace{14pt}
\noindent 
{\fontsize{14pt}{14pt}\selectfont import doctest \\} \par
\noindent 
{\fontsize{14pt}{14pt}\selectfont doctest.testmod()~~  \\} \par
\vspace{14pt}
\noindent 
{\fontsize{14pt}{14pt}\selectfont cara code dengang file terpisah \\} \par
\vspace{14pt}
\noindent 
{\fontsize{14pt}{14pt}\selectfont import unittest \\} \par
\vspace{14pt}
\noindent 
{\fontsize{14pt}{14pt}\selectfont class TestStatisticalFunctions(unittest.TestCase): \\} \par
\vspace{14pt}
\noindent 
{\fontsize{14pt}{14pt}\selectfont ~~~ def test $  \_  $average(self): \\} \par
\vspace{14pt}
\noindent 
{\fontsize{14pt}{14pt}\selectfont ~~~~~~~ self.assertEqual(average([20, 30, 70]), 40.0) \\} \par
\vspace{14pt}
\noindent 
{\fontsize{14pt}{14pt}\selectfont ~~~~~~~ self.assertEqual(round(average([1, 5, 7]), 1), 4.3) \\} \par
\vspace{14pt}
\noindent 
{\fontsize{14pt}{14pt}\selectfont ~~~~~~~ self.assertRaises(ZeroDivisionError, average, []) \\} \par
\vspace{14pt}
\noindent 
{\fontsize{14pt}{14pt}\selectfont ~~~~~~~ self.assertRaises(TypeError, average, 20, 30, 70) \\} \par
\vspace{14pt}
\noindent 
{\fontsize{14pt}{14pt}\selectfont unittest.main()  \\} \par
\vspace{14pt}
\noindent 
{\fontsize{14pt}{14pt}\selectfont ada juga standar ikhtisar yang sering digunakan oleh progremer  \\} \par
\noindent 
{\fontsize{14pt}{14pt}\selectfont yaitu \\} \par
\vspace{14pt}
\noindent 
{\fontsize{14pt}{14pt}\selectfont pengenalan python \\} \par
\vspace{14pt}
\noindent 
{\fontsize{14pt}{14pt}\selectfont menginstal python \\} \par
\vspace{14pt}
\noindent 
{\fontsize{14pt}{14pt}\selectfont numbers + matematika \\} \par
\vspace{14pt}
\noindent 
{\fontsize{14pt}{14pt}\selectfont string \\} \par
\vspace{14pt}
\noindent 
{\fontsize{14pt}{14pt}\selectfont list  \\} \par
\vspace{14pt}
\noindent 
{\fontsize{14pt}{14pt}\selectfont if elif else \\} \par
\vspace{14pt}
\noindent 
{\fontsize{14pt}{14pt}\selectfont for loop \\} \par
\vspace{14pt}
\noindent 
{\fontsize{14pt}{14pt}\selectfont for else, range, breake \\} \par
\vspace{14pt}
\noindent 
{\fontsize{14pt}{14pt}\selectfont continue dan pass \\} \par
\vspace{14pt}
\noindent 
{\fontsize{14pt}{14pt}\selectfont while loop \\} \par
\vspace{14pt}
\noindent 
{\fontsize{14pt}{14pt}\selectfont function \\} \par
\vspace{14pt}
\noindent 
{\fontsize{14pt}{14pt}\selectfont function dan arguments \\} \par
\vspace{14pt}
\noindent 
{\fontsize{14pt}{14pt}\selectfont return value \\} \par
\vspace{14pt}
\noindent 
{\fontsize{14pt}{14pt}\selectfont lambda function \\} \par
\vspace{14pt}
\noindent 
{\fontsize{14pt}{14pt}\selectfont scope global dan local \\} \par
\vspace{14pt}
\noindent 
{\fontsize{14pt}{14pt}\selectfont more on list \\} \par
\vspace{14pt}
\noindent 
{\fontsize{14pt}{14pt}\selectfont stacks and queues \\} \par
\vspace{14pt}
\vspace{14pt}
\vspace{14pt}
