\section {OVERVIEW}
Python adalah bahasa script tingkat tinggi, ditafsirkan, interaktif dan berorientasi objek. Python dirancang agar mudah dibaca. Ini menggunakan kata kunci bahasa Inggris sering di mana bahasa lainnya menggunakan tanda baca, dan memiliki konstruksi sintaksis lebih sedikit daripada bahasa lainnya.
Python is interpreted : diproses pada saat runtime oleh interpreter. Anda tidak perlu mengkompilasi program anda sebelum menjalankannya. Ini mirip dengan PERL dan PHP.
Tidak perlu untuk mengkompilasi program anda sebelum mengeksekusi itu. Hal ini merupakan mirip dengan php.
Python is Interactive: Anda dapat benar-benar duduk di prompt Python dan berinteraksi dengan penafsir langsung untuk menulis program Anda.
Python is Object-Oriented: Python mendukung gaya Berorientasi Objek atau teknik pemrograman yang merangkum kode di dalam objek.
Python is a Beginner's Language: Python adalah bahasa yang besar untuk programmer tingkat pemula dan mendukung pengembangan berbagai aplikasi dari pengolahan teks sederhana untuk browser WWW untuk game.
Fitur overview dalam python itu adalah :
\begin {enumerate}
\item Easy-to-learn: Python memiliki beberapa kata kunci, struktur sederhana, dan sintaks yang jelas. Hal ini memungkinkan siswa untuk mengambil bahasa dengan cepat.
\item Easy-to-read: kode Python lebih jelas dan terlihat mata.
\item Easy-to-maintain: kode sumber Python cukup mudah-untuk-menjaga.
\end {enumerate}

\subsection{Sejarah Python}
Python dikembangkan oleh Guido van Rossum pada akhir tahun delapan puluhan dan awal tahun sembilan puluhan di National Research Institute for Mathematics and Computer Science di Belanda. Python berasal dari banyak bahasa lain, termasuk ABC, Modula-3, C, C ++, Algol-68, SmallTalk, dan shell Unix dan bahasa script lainnya.
Fitur overview terbaik adalah IT mendukung metode pemrograman fungsional dan terstruktur serta OOP. Hal ini dapat digunakan sebagai bahasa scripting atau dapat dikompilasi untuk byte-kode untuk membangun aplikasi besar. Ini memberikan tingkat tinggi sangat tipe data dinamis dan mendukung memeriksa jenis dinamis. IT mendukung pengumpulan sampah otomatis. Hal ini dapat dengan mudah diintegrasikan dengan C, C ++, COM, ActiveX, CORBA, dan Java. Hal tersebut menjadi terpopuler karena kemudahan bagi programmer yang menjadikan python pemograman terbaik pada tahun 2016.

\subsection{Paradigma Pemrogramman Python}
Dalam perancangan bahasa, para perancang bahasa mengikuti paradigma-paradigma tertentu yang merupakan bentuk pemecahan masalah mengikuti aliran atau “genre” tertentu dari program dan bahasa. Berikut ini merupakan paradigma-paradigma pemograman yang utama:
\begin{enemurate}
\item Imperative programming-> program terdiri dari instruksi yang membentuk perhitungan, menerima input dan menghasilkan output. Contoh bahasa: Fortran, C, dan C++.
\item Object-oriented (OO) programming-> program adalah kumpulan objek yang saling berinteraksi melalui pesan yang mengubah state mereka. Contoh bahasa: Java, C++.
\item Functional programming-> program merupakan kumpulan fungsi matematika dengan input (domain) dan hasil (range). Fungsi-fungsi saling berinteraksi dan berkombinasi mengggunakan komposisi fungsional, kondisional, dan rekursif. Contoh bahasa: Lisp, Scheme,ML
\item Logic (declarative) programming -> memodelkan masalah menggunakan bahasa deklaratif, yang terdiri dari fakta dan aturan. Contoh bahasa : Prolog
\item Event-driven programming-> program merupakan sebuah loop yang secara kontinu  merespon event yang timbul oleh perintah yang tidak terduga.  Event ini berasal dari aksi user pada layar atau sumber lainnya. Contoh bahasa: Visual Basic dan Java.
\item Concurrent programming-> program merupakan sekumpulan proses yang bekerjasama, saling berbagi informasi dari waktu ke waktu tapi biasanya beroperasi secara tidak serempak. Contoh bahasa : SR, Linda, dan HPF.
\end{enemurate}
