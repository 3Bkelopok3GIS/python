%%%%%%%%%%%%  Generated using docx2latex.pythonanywhere.com  %%%%%%%%%%%%%%


\documentclass[a4paper,12pt]{report}

% Other options in place of 'report' are 1)article 2)book 3)letter
% Other options in place of 'a4paper' are 1)a5paper 2)b5paper 3)letterpaper 4)legalpaper 5)executivepaper


 %%%%%%%%%%%%  Include Packages  %%%%%%%%%%%%%%


\usepackage{amsmath}
\usepackage{latexsym}
\usepackage{amsfonts}
\usepackage{amssymb}
\usepackage{graphicx}
\usepackage{txfonts}
\usepackage{wasysym}
\usepackage{enumitem}
\usepackage{adjustbox}
\usepackage{ragged2e}
\usepackage{tabularx}
\usepackage{changepage}
\usepackage{setspace}
\usepackage{hhline}
\usepackage{multicol}
\usepackage{float}
\usepackage{multirow}
\usepackage{makecell}
\usepackage{fancyhdr}
\usepackage[toc,page]{appendix}
\usepackage[utf8]{inputenc}
\usepackage[T1]{fontenc}
\usepackage{hyperref}


 %%%%%%%%%%%%  Define Colors For Hyperlinks  %%%%%%%%%%%%%%


\hypersetup{
colorlinks=true,
linkcolor=blue,
filecolor=magenta,
urlcolor=cyan,
}
\urlstyle{same}


 %%%%%%%%%%%%  Set Depths for Sections  %%%%%%%%%%%%%%

% 1) Section
% 1.1) SubSection
% 1.1.1) SubSubSection
% 1.1.1.1) Paragraph
% 1.1.1.1.1) Subparagraph


\setcounter{tocdepth}{5}
\setcounter{secnumdepth}{5}


 %%%%%%%%%%%%  Set Page Margins  %%%%%%%%%%%%%%


\usepackage[a4paper,bindingoffset=0.2in,headsep=0.5cm,left=1.0in,right=1.0in,bottom=2cm,top=2cm,headheight=2cm]{geometry}
\everymath{\displaystyle}


 %%%%%%%%%%%%  Set Depths for Nested Lists created by \begin{enumerate}  %%%%%%%%%%%%%%


\setlistdepth{9}
\newlist{myEnumerate}{enumerate}{9}
	\setlist[myEnumerate,1]{label=\arabic*)}
	\setlist[myEnumerate,2]{label=\alph*)}
	\setlist[myEnumerate,3]{label=(\roman*)}
	\setlist[myEnumerate,4]{label=(\arabic*)}
	\setlist[myEnumerate,5]{label=(\Alph*)}
	\setlist[myEnumerate,6]{label=(\Roman*)}
	\setlist[myEnumerate,7]{label=\arabic*}
	\setlist[myEnumerate,8]{label=\alph*}
	\setlist[myEnumerate,9]{label=\roman*}

\renewlist{itemize}{itemize}{9}
	\setlist[itemize]{label=$\cdot$}
	\setlist[itemize,1]{label=\textbullet}
	\setlist[itemize,2]{label=$\circ$}
	\setlist[itemize,3]{label=$\ast$}
	\setlist[itemize,4]{label=$\dagger$}
	\setlist[itemize,5]{label=$\triangleright$}
	\setlist[itemize,6]{label=$\bigstar$}
	\setlist[itemize,7]{label=$\blacklozenge$}
	\setlist[itemize,8]{label=$\prime$}



 %%%%%%%%%%%%  Header here  %%%%%%%%%%%%%%


\pagestyle{fancy}
\fancyhf{}


 %%%%%%%%%%%%  Footer here  %%%%%%%%%%%%%%




 %%%%%%%%%%%%  Print Page Numbers  %%%%%%%%%%%%%%


\rfoot{\thepage}


 %%%%%%%%%%%%  This sets linespacing (verticle gap between Lines) Default=1 %%%%%%%%%%%%%%


\setstretch{1.08}


 %%%%%%%%%%%%  Document Code starts here %%%%%%%%%%%%%%


\begin{document}
\sloppy
\begin{center}{\fontsize{24pt}{24pt}\selectfont Python Date  $  \&  $ Time \\}\end{center} \par
\vspace{12pt}
\begin{adjustwidth}{0.03in}{0.03in}
 \hspace*{0.5in} Program Python dapat menangani tanggal dan waktu dengan beberapa cara. Mengkonversi antara format tanggal adalah tugas umum untuk komputer. Waktu dan modul kalender Python membantu melacak tanggal dan waktu.\end{adjustwidth}
 \par
\subsection*{What is Tick?}
 \par
\noindent 
Selang waktu adalah bilangan floating-point dalam satuan detik. Instansi tertentu dalam waktu dinyatakan dalam hitungan detik sejak pukul 12:00 pagi, 1 Januari 1970 (epoch). \par
\noindent 
Ada modul waktu populer yang tersedia dengan Python yang menyediakan fungsi untuk bekerja dengan waktu, dan untuk mengkonversi antara representasi. Fungsi time.time () mengembalikan waktu sistem saat ini di kutu sejak pukul 12:00, 1 Januari 1970 (zaman). \par
\vspace{12pt}
\subsection*{Example}
 \par
\vspace{12pt}
\noindent 
 \hspace*{0.5in}  $  \#  $!/usr/bin/python \par
\noindent 
 \hspace*{0.5in} import time;~  $  \#  $ This is required to include time module. \par
\vspace{12pt}
\noindent 
 \hspace*{0.5in} ticks = time.time() \par
\noindent 
 \hspace*{0.5in} print "Number of ticks since 12:00am, January 1, 1970:", ticks \par
\vspace{12pt}
\begin{adjustwidth}{0.03in}{0.03in}
Hal ini akan menghasilkan sesuatu sebagai berikut -\end{adjustwidth}
 \par
\noindent 
{\fontsize{9pt}{9pt}\selectfont  \hspace*{0.5in} Number of ticks since 12:00am, January 1, 1970: 7186862.73399} \par
\vspace{9pt}
\vspace{12pt}
\subsection*{Tanggal aritmatika mudah dilakukan dengan kutu. Namun, tanggal sebelum zaman tidak dapat diwakili dalam formulir ini. Tanggal di masa depan juga tidak dapat diwakili dengan cara ini - titik potong adalah sekitar 2038 untuk UNIX dan Windows.}
 \par
\vspace{12pt}
\subsection*{What is TimeTuple?}
 \par
\begin{adjustwidth}{0.03in}{0.03in}
Banyak fungsi waktu Python menangani waktu sebagai tuple dari 9 nomor, seperti yang ditunjukkan di bawah ini -\end{adjustwidth}
 \par


 %%%%%%%%%%%%  Table No:1 Here %%%%%%%%%%%%%%


\begin{table}[H]
\centering
\begin{adjustbox}{width=\textwidth}
\begin{tabular}{ p{0.67in}p{2.51in}p{3.11in} }
\hhline{---}
\multicolumn{1}{|p{0.67in}}{\textbf{Index}} & \multicolumn{1}{|p{2.51in}}{\textbf{Field}} & \multicolumn{1}{|p{3.11in}|}{\textbf{Values}} & \hhline{---}
\multicolumn{1}{|p{0.67in}}{0} & \multicolumn{1}{|p{2.51in}}{4-digit year} & \multicolumn{1}{|p{3.11in}|}{2008} & \hhline{---}
\multicolumn{1}{|p{0.67in}}{1} & \multicolumn{1}{|p{2.51in}}{Month} & \multicolumn{1}{|p{3.11in}|}{1 to 12} & \hhline{---}
\multicolumn{1}{|p{0.67in}}{2} & \multicolumn{1}{|p{2.51in}}{Day} & \multicolumn{1}{|p{3.11in}|}{1 to 31} & \hhline{---}
\multicolumn{1}{|p{0.67in}}{3} & \multicolumn{1}{|p{2.51in}}{Hour} & \multicolumn{1}{|p{3.11in}|}{0 to 23} & \hhline{---}
\multicolumn{1}{|p{0.67in}}{4} & \multicolumn{1}{|p{2.51in}}{Minute} & \multicolumn{1}{|p{3.11in}|}{0 to 59} & \hhline{---}
\multicolumn{1}{|p{0.67in}}{5} & \multicolumn{1}{|p{2.51in}}{Second} & \multicolumn{1}{|p{3.11in}|}{0 to 61 (60 or 61 are leap-seconds)} & \hhline{---}
\multicolumn{1}{|p{0.67in}}{6} & \multicolumn{1}{|p{2.51in}}{Day of Week} & \multicolumn{1}{|p{3.11in}|}{0 to 6 (0 is Monday)} & \hhline{---}
\multicolumn{1}{|p{0.67in}}{7} & \multicolumn{1}{|p{2.51in}}{Day of year} & \multicolumn{1}{|p{3.11in}|}{1 to 366 (Julian day)} & \hhline{---}
\multicolumn{1}{|p{0.67in}}{8} & \multicolumn{1}{|p{2.51in}}{Daylight savings} & \multicolumn{1}{|p{3.11in}|}{-1, 0, 1, -1 means library determines DST} & \hline
\end{tabular}
\end{adjustbox}
\end{table}


 %%%%%%%%%%%%  Table No:1 Ends Here %%%%%%%%%%%%%%


\vspace{12pt}
\begin{adjustwidth}{0.03in}{0.03in}
Tuple di atas setara dengan struct $  \_  $time structure. Struktur ini memiliki atribut berikut –\end{adjustwidth}
 \par
\vspace{12pt}
\vspace{12pt}


 %%%%%%%%%%%%  Table No:2 Here %%%%%%%%%%%%%%


\begin{table}[H]
\centering
\begin{adjustbox}{width=\textwidth}
\begin{tabular}{ p{0.67in}p{2.51in}p{3.11in} }
\hhline{---}
\multicolumn{1}{|p{0.67in}}{\textbf{Index}} & \multicolumn{1}{|p{2.51in}}{\textbf{Attributes}} & \multicolumn{1}{|p{3.11in}|}{\textbf{Values}} & \hhline{---}
\multicolumn{1}{|p{0.67in}}{0} & \multicolumn{1}{|p{2.51in}}{tm $  \_  $year} & \multicolumn{1}{|p{3.11in}|}{2008} & \hhline{---}
\multicolumn{1}{|p{0.67in}}{1} & \multicolumn{1}{|p{2.51in}}{tm $  \_  $mon} & \multicolumn{1}{|p{3.11in}|}{1 to 12} & \hhline{---}
\multicolumn{1}{|p{0.67in}}{2} & \multicolumn{1}{|p{2.51in}}{tm $  \_  $mday} & \multicolumn{1}{|p{3.11in}|}{1 to 31} & \hhline{---}
\multicolumn{1}{|p{0.67in}}{3} & \multicolumn{1}{|p{2.51in}}{tm $  \_  $hour} & \multicolumn{1}{|p{3.11in}|}{0 to 23} & \hhline{---}
\multicolumn{1}{|p{0.67in}}{4} & \multicolumn{1}{|p{2.51in}}{tm $  \_  $min} & \multicolumn{1}{|p{3.11in}|}{0 to 59} & \hhline{---}
\multicolumn{1}{|p{0.67in}}{5} & \multicolumn{1}{|p{2.51in}}{tm $  \_  $sec} & \multicolumn{1}{|p{3.11in}|}{0 to 61 (60 or 61 are leap-seconds)} & \hhline{---}
\multicolumn{1}{|p{0.67in}}{6} & \multicolumn{1}{|p{2.51in}}{tm $  \_  $wday} & \multicolumn{1}{|p{3.11in}|}{0 to 6 (0 is Monday)} & \hhline{---}
\multicolumn{1}{|p{0.67in}}{7} & \multicolumn{1}{|p{2.51in}}{tm $  \_  $yday} & \multicolumn{1}{|p{3.11in}|}{1 to 366 (Julian day)} & \hhline{---}
\multicolumn{1}{|p{0.67in}}{8} & \multicolumn{1}{|p{2.51in}}{tm $  \_  $isdst} & \multicolumn{1}{|p{3.11in}|}{-1, 0, 1, -1 means library determines DST} & \hline
\end{tabular}
\end{adjustbox}
\end{table}


 %%%%%%%%%%%%  Table No:2 Ends Here %%%%%%%%%%%%%%


\vspace{20pt}
\vspace{20pt}
\subsection*{Getting current time}
 \par
\begin{adjustwidth}{0.03in}{0.03in}
 \hspace*{0.5in} Untuk menerjemahkan waktu instan dari satu detik sejak nilai floating-point ke waktu tupel, lewati nilai floating-point ke fungsi (mis., Localtime) yang mengembalikan waktu tupel dengan semua sembilan item valid.\end{adjustwidth}
 \par
\vspace{12pt}
\noindent 
 \hspace*{0.5in}  $  \#  $!/usr/bin/python \par
\noindent 
 \hspace*{0.5in} import time; \par
\vspace{12pt}
\noindent 
 \hspace*{0.5in} localtime = time.localtime(time.time()) \par
\noindent 
 \hspace*{0.5in} print "Local current time :", localtime \par
\vspace{12pt}
\begin{adjustwidth}{0.03in}{0.03in}
Ini akan menghasilkan hasil berikut, yang dapat diformat dalam bentuk lain yang sesuai -\end{adjustwidth}
 \par
\noindent 
{\fontsize{9pt}{9pt}\selectfont  \hspace*{0.5in} Local current time : time.struct $  \_  $time(tm $  \_  $year=2013, tm $  \_  $mon=7, } \par
\noindent 
{\fontsize{9pt}{9pt}\selectfont  \hspace*{0.5in} tm $  \_  $mday=17, tm $  \_  $hour=21, tm $  \_  $min=26, tm $  \_  $sec=3, tm $  \_  $wday=2, tm $  \_  $yday=198, tm $  \_  $isdst=0)} \par
\vspace{9pt}
\vspace{20pt}
\vspace{12pt}
\subsection*{Getting formatted time}
 \par
\begin{adjustwidth}{0.03in}{0.03in}
 \hspace*{0.5in} Anda dapat memformat kapan saja sesuai kebutuhan Anda, namun metode sederhana untuk mendapatkan waktu dalam format yang mudah dibaca adalah asctime () -\end{adjustwidth}
 \par
\noindent 
 \hspace*{0.5in}  $  \#  $!/usr/bin/python \par
\noindent 
 \hspace*{0.5in} import time; \par
\vspace{12pt}
\noindent 
 \hspace*{0.5in} localtime = time.asctime( time.localtime(time.time()) ) \par
\noindent 
 \hspace*{0.5in} print "Local current time :", localtime \par
\vspace{12pt}
\begin{adjustwidth}{0.03in}{0.03in}
Ini akan menghasilkan hasil sebagai berikut -\end{adjustwidth}
 \par
\noindent 
 \hspace*{0.5in} Local current time : Tue Jan 13 10:17:09 2009 \par
\vspace{12pt}
\vspace{20pt}
\subsection*{Getting calendar for a month}
 \par
\begin{adjustwidth}{0.03in}{0.03in}
Modul kalender memberikan berbagai macam metode untuk dimainkan dengan kalender tahunan dan bulanan. Di sini, kami mencetak kalender untuk bulan tertentu (Jan 2008) -\end{adjustwidth}
 \par
\noindent 
 \hspace*{0.5in}  $  \#  $!/usr/bin/python \par
\noindent 
 \hspace*{0.5in} import calendar \par
\vspace{12pt}
\noindent 
 \hspace*{0.5in} cal = calendar.month(2008, 1) \par
\noindent 
 \hspace*{0.5in} print "Here is the calendar:" \par
\noindent 
 \hspace*{0.5in} print cal \par
\vspace{12pt}
\begin{adjustwidth}{0.03in}{0.03in}
Ini akan menghasilkan hasil sebagai berikut -\end{adjustwidth}
 \par
\noindent 
{\fontsize{9pt}{9pt}\selectfont  \hspace*{0.5in} Here is the calendar:} \par
\noindent 
{\fontsize{9pt}{9pt}\selectfont ~~~  \hspace*{0.5in}  \hspace*{0.5in} January 2008} \par
\noindent 
{\fontsize{9pt}{9pt}\selectfont  \hspace*{0.5in} Mo Tu We Th Fr Sa Su} \par
\noindent 
{\fontsize{9pt}{9pt}\selectfont ~~~  \hspace*{0.5in}  \hspace*{0.5in} 1~ 2~~3~~4  5  6} \par
\noindent 
{\fontsize{9pt}{9pt}\selectfont   \hspace*{0.5in}  \hspace*{0.5in} 7~ 8~ 9 10 11 12 13} \par
\noindent 
{\fontsize{9pt}{9pt}\selectfont  \hspace*{0.5in}  \hspace*{0.5in} 14 15 16 17 18 19 20} \par
\noindent 
{\fontsize{9pt}{9pt}\selectfont  \hspace*{0.5in}  \hspace*{0.5in} 21 22 23 24 25 26 27} \par
\noindent 
{\fontsize{9pt}{9pt}\selectfont  \hspace*{0.5in}  \hspace*{0.5in} 28 29 30 31} \par
\vspace{9pt}
\vspace{20pt}
\vspace{12pt}
\subsection*{The time Module}
 \par
\begin{adjustwidth}{0.03in}{0.03in}
 \hspace*{0.5in} Ada modul waktu populer yang tersedia dengan Python yang menyediakan fungsi untuk bekerja dengan waktu dan untuk mengkonversi antara representasi. Berikut adalah daftar semua metode yang tersedia -\end{adjustwidth}
 \par


 %%%%%%%%%%%%  Table No:3 Here %%%%%%%%%%%%%%


\begin{table}[H]
\centering
\begin{adjustbox}{width=\textwidth}
\begin{tabular}{ p{0.4in}p{5.9in} }
\hhline{--}
\multicolumn{1}{|p{0.4in}}{\textbf{SN}} & \multicolumn{1}{|p{5.9in}|}{\textbf{Function with Description}} & \hhline{--}
\multicolumn{1}{|p{0.4in}}{1} & \multicolumn{1}{|p{5.9in}|}{\href{https://www.tutorialspoint.com/python/time $  \_  $altzone.htm}{time.altzone}
The offset of the local DST timezone, in seconds west of UTC, if one is defined. This is negative if the local DST timezone is east of UTC (as in Western Europe, including the UK). Only use this if daylight is nonzero.} & \hhline{--}
\multicolumn{1}{|p{0.4in}}{2} & \multicolumn{1}{|p{5.9in}|}{\href{https://www.tutorialspoint.com/python/time $  \_  $asctime.htm}{time.asctime([tupletime])}
Accepts a time-tuple and returns a readable 24-character string such as 'Tue Dec 11 18:07:14 2008'.} & \hhline{--}
\multicolumn{1}{|p{0.4in}}{3} & \multicolumn{1}{|p{5.9in}|}{\href{https://www.tutorialspoint.com/python/time $  \_  $clock.htm}{time.clock( )}
Returns the current CPU time as a floating-point number of seconds. To measure computational costs of different approaches, the value of time.clock is more useful than that of time.time().} & \hhline{--}
\multicolumn{1}{|p{0.4in}}{4} & \multicolumn{1}{|p{5.9in}|}{\href{https://www.tutorialspoint.com/python/time $  \_  $ctime.htm}{time.ctime([secs])}
Like asctime(localtime(secs)) and without arguments is like asctime( )} & \hhline{--}
\multicolumn{1}{|p{0.4in}}{5} & \multicolumn{1}{|p{5.9in}|}{\href{https://www.tutorialspoint.com/python/time $  \_  $gmtime.htm}{time.gmtime([secs])}
Accepts an instant expressed in seconds since the epoch and returns a time-tuple t with the UTC time. Note : t.tm $  \_  $isdst is always 0} & \hhline{--}
\multicolumn{1}{|p{0.4in}}{6} & \multicolumn{1}{|p{5.9in}|}{\href{https://www.tutorialspoint.com/python/time $  \_  $localtime.htm}{time.localtime([secs])}
Accepts an instant expressed in seconds since the epoch and returns a time-tuple t with the local time (t.tm $  \_  $isdst is 0 or 1, depending on whether DST applies to instant secs by local rules).} & \hhline{--}
\multicolumn{1}{|p{0.4in}}{7} & \multicolumn{1}{|p{5.9in}|}{\href{https://www.tutorialspoint.com/python/time $  \_  $mktime.htm}{time.mktime(tupletime)}
Accepts an instant expressed as a time-tuple in local time and returns a floating-point value with the instant expressed in seconds since the epoch.} & \hhline{--}
\multicolumn{1}{|p{0.4in}}{8} & \multicolumn{1}{|p{5.9in}|}{\href{https://www.tutorialspoint.com/python/time $  \_  $sleep.htm}{time.sleep(secs)}
Suspends the calling thread for secs seconds.} & \hhline{--}
\multicolumn{1}{|p{0.4in}}{9} & \multicolumn{1}{|p{5.9in}|}{\href{https://www.tutorialspoint.com/python/time $  \_  $strftime.htm}{time.strftime(fmt[,tupletime])}
Accepts an instant expressed as a time-tuple in local time and returns a string representing the instant as specified by string fmt.} & \hhline{--}
\multicolumn{1}{|p{0.4in}}{10} & \multicolumn{1}{|p{5.9in}|}{\href{https://www.tutorialspoint.com/python/time $  \_  $strptime.htm}{time.strptime(str,fmt=' $  \%  $a  $  \%  $b  $  \%  $d  $  \%  $H: $  \%  $M: $  \%  $S  $  \%  $Y')}
Parses str according to format string fmt and returns the instant in time-tuple format.} & \hhline{--}
\multicolumn{1}{|p{0.4in}}{11} & \multicolumn{1}{|p{5.9in}|}{\href{https://www.tutorialspoint.com/python/time $  \_  $time.htm}{time.time( )}
Returns the current time instant, a floating-point number of seconds since the epoch.} & \hhline{--}
\multicolumn{1}{|p{0.4in}}{12} & \multicolumn{1}{|p{5.9in}|}{\href{https://www.tutorialspoint.com/python/time $  \_  $tzset.htm}{time.tzset()}
Resets the time conversion rules used by the library routines. The environment variable TZ specifies how this is done.} & \hline
\end{tabular}
\end{adjustbox}
\end{table}


 %%%%%%%%%%%%  Table No:3 Ends Here %%%%%%%%%%%%%%


\vspace{12pt}
\begin{adjustwidth}{0.03in}{0.03in}
Mari kita lalui fungsi secara singkat -\end{adjustwidth}
 \par
\begin{adjustwidth}{0.03in}{0.03in}
Ada dua atribut penting yang tersedia dengan modul waktu:\end{adjustwidth}
 \par
\vspace{12pt}


 %%%%%%%%%%%%  Table No:4 Here %%%%%%%%%%%%%%


\begin{table}[H]
\centering
\begin{adjustbox}{width=\textwidth}
\begin{tabular}{ p{0.4in}p{5.89in} }
\hhline{--}
\multicolumn{1}{|p{0.4in}}{\textbf{SN}} & \multicolumn{1}{|p{5.89in}|}{\textbf{Attribute with Description}} & \hhline{--}
\multicolumn{1}{|p{0.4in}}{1} & \multicolumn{1}{|p{5.89in}|}{\textbf{time.timezone}Attribute time.timezone is the offset in seconds of the local time zone (without DST) from UTC (>0 in the Americas; <=0 in most of Europe, Asia, Africa).} & \hhline{--}
\multicolumn{1}{|p{0.4in}}{2} & \multicolumn{1}{|p{5.89in}|}{\textbf{time.tzname}Attribute time.tzname is a pair of locale-dependent strings, which are the names of the local time zone without and with DST, respectively.} & \hline
\end{tabular}
\end{adjustbox}
\end{table}


 %%%%%%%%%%%%  Table No:4 Ends Here %%%%%%%%%%%%%%


\vspace{20pt}
\subsection*{The calendar Module}
 \par
\begin{adjustwidth}{0.03in}{0.03in}
Modul kalender memasok fungsi yang berhubungan dengan kalender, termasuk fungsi untuk mencetak kalender teks untuk bulan atau tahun tertentu.\end{adjustwidth}
 \par
\begin{adjustwidth}{0.03in}{0.03in}
Secara default, kalender mengambil hari Senin sebagai hari pertama minggu dan minggu sebagai yang terakhir. Untuk mengubah ini, fungsi call calendar.setfirstweekday ().\end{adjustwidth}
 \par
\begin{adjustwidth}{0.03in}{0.03in}
Berikut adalah daftar fungsi yang tersedia dengan modul kalender:\end{adjustwidth}
 \par
\vspace{12pt}


 %%%%%%%%%%%%  Table No:5 Here %%%%%%%%%%%%%%


\begin{table}[H]
\centering
\begin{adjustbox}{width=\textwidth}
\begin{tabular}{ p{0.4in}p{5.9in} }
\hhline{--}
\multicolumn{1}{|p{0.4in}}{\textbf{SN}} & \multicolumn{1}{|p{5.9in}|}{\textbf{Function with Description}} & \hhline{--}
\multicolumn{1}{|p{0.4in}}{1} & \multicolumn{1}{|p{5.9in}|}{\textbf{calendar.calendar}\textbf{(}\textbf{year,w}\textbf{=2,l=1,c=6)}Returns a multiline string with a calendar for year year formatted into three columns separated by c spaces. w is the width in characters of each date; each line has length 21*w+18+2*c. l is the number of lines for each week.} & \hhline{--}
\multicolumn{1}{|p{0.4in}}{2} & \multicolumn{1}{|p{5.9in}|}{\textbf{calendar.firstweekday}\textbf{( )}Returns the current setting for the weekday that starts each week. By default, when calendar is first imported, this is 0, meaning Monday.} & \hhline{--}
\multicolumn{1}{|p{0.4in}}{3} & \multicolumn{1}{|p{5.9in}|}{\textbf{calendar.isleap}\textbf{(year)}Returns True if year is a leap year; otherwise, False.} & \hhline{--}
\multicolumn{1}{|p{0.4in}}{4} & \multicolumn{1}{|p{5.9in}|}{\textbf{calendar.leapdays}\textbf{(y1,y2)}Returns the total number of leap days in the years within range(y1,y2).} & \hhline{--}
\multicolumn{1}{|p{0.4in}}{5} & \multicolumn{1}{|p{5.9in}|}{\textbf{calendar.month}\textbf{(}\textbf{year,month,w}\textbf{=2,l=1)}Returns a multiline string with a calendar for month month of year year, one line per week plus two header lines. w is the width in characters of each date; each line has length 7*w+6. l is the number of lines for each week.} & \hhline{--}
\multicolumn{1}{|p{0.4in}}{6} & \multicolumn{1}{|p{5.9in}|}{\textbf{calendar.monthcalendar}\textbf{(}\textbf{year,month}\textbf{)}Returns a list of lists of ints. Each sublist denotes a week. Days outside month month of year year are set to 0; days within the month are set to their day-of-month, 1 and up.} & \hhline{--}
\multicolumn{1}{|p{0.4in}}{7} & \multicolumn{1}{|p{5.9in}|}{\textbf{calendar.monthrange}\textbf{(}\textbf{year,month}\textbf{)}Returns two integers. The first one is the code of the weekday for the first day of the month month in year year; the second one is the number of days in the month. Weekday codes are 0 (Monday) to 6 (Sunday); month numbers are 1 to 12.} & \hhline{--}
\multicolumn{1}{|p{0.4in}}{8} & \multicolumn{1}{|p{5.9in}|}{\textbf{calendar.prcal}\textbf{(}\textbf{year,w}\textbf{=2,l=1,c=6)}Like print calendar.calendar(year,w,l,c).} & \hhline{--}
\multicolumn{1}{|p{0.4in}}{9} & \multicolumn{1}{|p{5.9in}|}{\textbf{calendar.prmonth}\textbf{(}\textbf{year,month,w}\textbf{=2,l=1)}Like print calendar.month(year,month,w,l).} & \hhline{--}
\multicolumn{1}{|p{0.4in}}{10} & \multicolumn{1}{|p{5.9in}|}{\textbf{calendar.setfirstweekday}\textbf{(weekday)}Sets the first day of each week to weekday code weekday. Weekday codes are 0 (Monday) to 6 (Sunday).} & \hhline{--}
\multicolumn{1}{|p{0.4in}}{11} & \multicolumn{1}{|p{5.9in}|}{\textbf{calendar.timegm}\textbf{(}\textbf{tupletime}\textbf{)}The inverse of time.gmtime: accepts a time instant in time-tuple form and returns the same instant as a floating-point number of seconds since the epoch.} & \hhline{--}
\multicolumn{1}{|p{0.4in}}{12} & \multicolumn{1}{|p{5.9in}|}{\textbf{calendar.weekday}\textbf{(}\textbf{year,month,day}\textbf{)}Returns the weekday code for the given date. Weekday codes are 0 (Monday) to 6 (Sunday); month numbers are 1 (January) to 12 (December).} & \hline
\end{tabular}
\end{adjustbox}
\end{table}


 %%%%%%%%%%%%  Table No:5 Ends Here %%%%%%%%%%%%%%


\vspace{20pt}
\subsection*{Other Modules & Functions:}
 \par
\begin{adjustwidth}{0.03in}{0.03in}
Jika Anda tertarik, maka di sini Anda akan menemukan daftar modul dan fungsi penting lainnya untuk bermain dengan tanggal  $  \&  $ waktu dengan Python:\end{adjustwidth}
 \par
\begin{adjustwidth}{0.53in}{0.03in}
\begin{itemize}
\item \href{http://docs.python.org/library/datetime.html}{The $  $datetime $  $Module}
\end{adjustwidth}
 \par
\begin{adjustwidth}{0.53in}{0.03in}
\item \href{http://www.twinsun.com/tz/tz-link.htm}{The $  $pytz $  $Module}
\end{adjustwidth}
 \par
\begin{adjustwidth}{0.53in}{0.03in}
\item \href{http://labix.org/python-dateutil}{The $  $dateutil $  $Module}
\end{itemize}
\end{adjustwidth}
 \par
\end{document}
