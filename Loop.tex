\sloppy
\begin{center}Python Loops\end{center} \par
\vspace{12pt}
\vspace{12pt}
Secara umum, pernyataan pada bahasa pemrograman akan dieksekusi secara berurutan. Pernyataan pertama dalam sebuah fungsi dijalankan pertama, diikuti oleh yang kedua, dan seterusnya. Tetapi akan ada situasi dimana Anda harus menulis banyak kode, dimana kode tersebut sangat banyak. Jika dilakukan secara manual maka Anda hanya akan membuang-buang tenaga dengan menulis beratus-ratus bahkan beribu-ribu kode. Untuk itu Anda perlu menggunakan pengulangan di dalam bahasa pemrograman Python. \par
\noindent 
\vspace{\baselineskip}
\vspace{\baselineskip}
Di dalam bahasa pemrograman Python pengulangan dibagi menjadi 3 bagian, yaitu : \par
\noindent 
While Loop \par
\noindent 
For Loop \par
\noindent 
Nested Loop \par
\noindent 
\vspace{\baselineskip}
\vspace{\baselineskip}
\vspace{12pt}
\noindent 
While Loop \par
\noindent 
Pengulangan While Loop di dalam bahasa pemrograman Python dieksesusi statement berkali-kali selama kondisi bernilai benar atau True. \par
\noindent 
\vspace{\baselineskip}
\vspace{\baselineskip}
Dibawah ini adalah contoh penggunaan pengulangan While Loop. \par
\noindent 
\vspace{\baselineskip}
\vspace{12pt}
\noindent 
Contoh penggunaan While Loop \par
\noindent 
\vspace{\baselineskip}
\vspace{\baselineskip}
count = 0 \par
\noindent 
\vspace{\baselineskip}
while (count < 9): \par
\noindent 
\vspace{\baselineskip}
 $  $  $  $ print ('The count is:', count) \par
\noindent 
\vspace{\baselineskip}
 $  $  $  $ count = count + 1 \par
\noindent 
\vspace{\baselineskip}
\vspace{\baselineskip}
print ("Good bye!") \par
\noindent 
\vspace{\baselineskip}
\vspace{\baselineskip}
\vspace{\baselineskip}
\vspace{12pt}
\noindent 
For Loop \par
\noindent 
Pengulangan For pada Python memiliki kemampuan untuk mengulangi item dari urutan apapun, seperti $  $list atau string. \par
\noindent 
 \vspace{\baselineskip}
\vspace{\baselineskip}
Dibawah ini adalah contoh penggunaan pengulangan While Loop. \par
\noindent 
\vspace{\baselineskip}
\vspace{12pt}
\noindent 
Contoh pengulangan for sederhana \par
\noindent 
\vspace{\baselineskip}
angka = [1,2,3,4,5] \par
\noindent 
\vspace{\baselineskip}
for x in angka: \par
\noindent 
\vspace{\baselineskip}
 $  $  $  $ print(x) \par
\noindent 
\vspace{\baselineskip}
\vspace{\baselineskip}
Contoh pengulangan for \par
\noindent 
\vspace{\baselineskip}
buah = ["nanas", "apel", "jeruk"] \par
\noindent 
\vspace{\baselineskip}
for makanan in buah: \par
\noindent 
\vspace{\baselineskip}
 $  $  $  $ print("Saya suka makan", makanan) \par
\noindent 
\vspace{\baselineskip}
\vspace{\baselineskip}
\vspace{12pt}
\noindent 
Nested Loop \par
\vspace{12pt}
\noindent 
Bahasa pemrograman Python memungkinkan penggunaan satu lingkaran di dalam loop lain. Bagian berikut menunjukkan beberapa contoh untuk menggambarkan konsep tersebut. $  $\vspace{\baselineskip}
\vspace{\baselineskip}
Dibawah ini adalah contoh penggunaan Nested Loop. \par
\noindent 
Contoh penggunaan Nested Loop \par
\noindent 
\vspace{\baselineskip}
\vspace{\baselineskip}
i = 2 \par
\noindent 
\vspace{\baselineskip}
while(i < 100): \par
\noindent 
\vspace{\baselineskip}
 $  $  $  $ j = 2 \par
\noindent 
\vspace{\baselineskip}
 $  $  $  $ while(j <= (i/j)): \par
\noindent 
\vspace{\baselineskip}
 $  $  $  $  $  $  $  $ if not(i $  \%  $j): break \par
\noindent 
\vspace{\baselineskip}
 $  $  $  $  $  $  $  $ j = j + 1 \par
\noindent 
\vspace{\baselineskip}
 $  $  $  $ if (j > i/j) : print i, " is prime" \par
\noindent 
\vspace{\baselineskip}
 $  $  $  $ i = i + 1 \par
\noindent 
\vspace{\baselineskip}
\vspace{\baselineskip}
print "Good bye!" \par
\vspace{12pt}
\vspace{12pt}
Perhatikan contoh berikut ini:\vspace{\baselineskip}
\vspace{\baselineskip}
 \par
\vspace{12pt}
print ("1") \par
print ("2") \par
print ("3") \par
print ("4") \par
print ("5") \par
print ("6") \par
print ("7") \par
print ("8") \par
print ("9") \par
print ("10") \par
\vspace{12pt}
\vspace{\baselineskip}
Contoh program diatas adalah program untuk menampilkan angka 1 sampai dengan 10 tanpa perulangan. Tanpa menggunakan perulangan, programmer harus menuliskan semua statement diatas sehingga source code menjadi lebih banyak dan tidak efisien. Bayangkan kalau programmer disuruh menampilkan angka 1 sampai dengan 1000000 tanpa menggunakan perulangan\vspace{\baselineskip}
\vspace{\baselineskip}
Dengan menggunakan perulangan, source code lebih pendek dan efisien. Perhatikan contoh program untuk mencetak angka 1 sampai dengan 10 dengan menggunakan konsep perulangan di bawah ini.\vspace{\baselineskip}
\vspace{\baselineskip}
 \par
\vspace{12pt}
i = 1 \par
while(i < 11): \par
~~~ print(i) \par
~~~ i = i+1 \par
\vspace{\baselineskip}
Bandingkan kedua program diatas, Mana yang lebih efisien? Mana yang lebih simple?\vspace{\baselineskip}
\vspace{\baselineskip}
Ada 3 macam bentuk perulangan pada Python, yaitu: \par
FOR Loop \par
WHILE Loop \par
dan Loop bersarang (Nested Loop) \par
\vspace{\baselineskip}
Selain membahas 3 bentuk perulangan diatas, tutorial ini juga membahas control perulangan, meliputi: \par
Break Statement \par
Continue Statement \par
dan Pass Statement \par
\vspace{\baselineskip}
\vspace{12pt}
FOR Loop \par
FOR Loop digunakan untuk melakukan perulangan atau iterasi sampai batas atau range yang telah ditentukan.\vspace{\baselineskip}
\vspace{\baselineskip}
Dibawah ini adalah sintak dasar FOR Loop di Python.\vspace{\baselineskip}
\vspace{\baselineskip}
 \par
for iterating $  \_  $var in range: \par
~~ statements(s) \par
\vspace{\baselineskip}
Contoh Program\vspace{\baselineskip}
\vspace{\baselineskip}
Perhatikan contoh program For Loop pada Python:\vspace{\baselineskip}
\vspace{\baselineskip}
Contoh 1\vspace{\baselineskip}
\vspace{\baselineskip}
 \par
Program mencetak angka 1 s/d 10 \par
\vspace{12pt}
i = 10 \par
for i in range(10): \par
~~ print(i+1) \par
~~ i = i+1 \par
\vspace{\baselineskip}
Fungsi $  $range() $  $biasanya digunakan sebagai counter pada perulangan bentuk For. range(10) artinya menampikan perulangan sebanyak 10 elemen.\vspace{\baselineskip}
\vspace{\baselineskip}
Apabila program diatas Anda jalankan, maka akan menampilkan angka 1 sampai dengan 10 seperti output di bawah ini:\vspace{\baselineskip}
\vspace{\baselineskip}
 \par
1 \par
2 \par
3 \par
4 \par
5 \par
6 \par
7 \par
8 \par
9 \par
10 \par
\vspace{\baselineskip}
Contoh 2\vspace{\baselineskip}
\vspace{\baselineskip}
 \par
 Program mencetak angka -1 s/d 8 \par
\vspace{12pt}
i = 10 \par
for i in range(-10, 10, 2):  $  \#  $ range(range awal, range akhir, selisih) \par
~~ print(i) \par
\vspace{12pt}
\vspace{\baselineskip}
Perhatikan pada range(-10, 10, 2) artinya perulangan akan dimulai dari batas awal -10 sampai dengan batas akhir 10 dengan selisih 2.\vspace{\baselineskip}
\vspace{\baselineskip}
Apabila program diatas Anda jalankan, maka akan menampilkan output berikut ini:\vspace{\baselineskip}
\vspace{\baselineskip}
 \par
-10 \par
-8 \par
-6 \par
-4 \par
-2 \par
0 \par
2 \par
4 \par
6 \par
8 \par
\vspace{\baselineskip}
Contoh 3\vspace{\baselineskip}
\vspace{\baselineskip}
 \par
Program menampilkan huruf Belajar Python \par
for~huruf~in 'Belajar Python':    \par
~~ print (huruf) \par
\vspace{\baselineskip}
Apabila program diatas Anda jalankan, maka akan menghasilkan output berikut ini:\vspace{\baselineskip}
\vspace{\baselineskip}
 \par
B \par
e \par
l \par
a \par
j \par
a \par
r \par
  \par
P \par
y \par
t \par
h \par
o \par
n \par
\vspace{12pt}
Contoh 4\vspace{\baselineskip}
\vspace{\baselineskip}
Program berikut akan menampilkan perulangan dari list atau tupple.\vspace{\baselineskip}
\vspace{\baselineskip}
 \par
Program menampilkan huruf Belajar Python \par
\vspace{12pt}
makanan = ['Pizza', 'Nasi Bebek',~ 'Rujak Buah'] \par
for makan in makanan: \par
~~ print ("Makanan Favorit :", makan) \par
\vspace{12pt}
Apabila program diatas Anda jalankan, maka akan menghasilkan output berikut ini:\vspace{\baselineskip}
\vspace{\baselineskip}
 \par
Makanan Favorit : Pizza \par
Makanan Favorit : Nasi Bebek \par
Makanan Favorit : Rujak Buah \par
\vspace{12pt}
\vspace{\baselineskip}
\vspace{12pt}
While Loop \par
While Loop akan menjalankan statemet selama kondisi terpenuhi (atau bernilai true).\vspace{\baselineskip}
\vspace{\baselineskip}
Di bawah ini adalah sintak dasar dari While Loop pada Python\vspace{\baselineskip}
\vspace{\baselineskip}
Contoh Program\vspace{\baselineskip}
\vspace{\baselineskip}
Coba Anda ketik program di bawah ini:\vspace{\baselineskip}
\vspace{\baselineskip}
 \par
Program mencetak angka 1 s/d 10 \par
\vspace{12pt}
i = 1 \par
while(i < 11): \par
 print(i) \par
 i = i+1 \par
\vspace{\baselineskip}
Apabila program diatas Anda jalankan, maka akan menghasilkan output seperti di bawah ini:\vspace{\baselineskip}
\vspace{\baselineskip}
 \par
1 \par
2 \par
3 \par
4 \par
5 \par
6 \par
7 \par
8 \par
9 \par
10 \par
\vspace{12pt}
\vspace{\baselineskip}
\vspace{12pt}
Infinite Loop \par
\vspace{\baselineskip}
Infinite Loop adalah kondisi perulangan, dimana statement akan dijalankan terus menerus tanpa berhenti. Akan berhenti kalau Anda menekan tombol CTRL+C.\vspace{\baselineskip}
\vspace{\baselineskip}
Di bawah ini contoh program Infinite Loop\vspace{\baselineskip}
\vspace{\baselineskip}
 \par
program menampilkan tulisan Python tanpa henti \par
\vspace{12pt}
flag = 1 \par
\vspace{12pt}
while (flag): print ("Python") \par
print ("Good bye!") \par
\vspace{12pt}
\vspace{\baselineskip}
\vspace{12pt}
Nested Loop \par
\vspace{\baselineskip}
Nested Loop secara sederhana adalah perulangan di dalam perulangan.\vspace{\baselineskip}
\vspace{\baselineskip}
Di bawah ini adalah sintak dasar Nested Loop pada Python:\vspace{\baselineskip}
\vspace{\baselineskip}
 \par
for iterating $  \_  $var in sequence: \par
~~ for iterating $  \_  $var in sequence: \par
~~~~~ statements(s) \par
~~ statements(s) \par
\vspace{\baselineskip}
atau yang menggunakan while loop\vspace{\baselineskip}
\vspace{\baselineskip}
 \par
while expression: \par
~~ while expression: \par
~~~~~ statement(s) \par
~~ statement(s) \par
\vspace{\baselineskip}
Contoh Program\vspace{\baselineskip}
\vspace{\baselineskip}
Di bawah ini adalah contoh program implementasi Nested Loop untuk mencetak bilangan prima dari 2 sampai 30.\vspace{\baselineskip}
\vspace{\baselineskip}
 \par
Program menampilkan bilangan prima dari 2 s/d 30 \par
\vspace{12pt}
i = 2 \par
while(i < 30): \par
~~ j = 2 \par
~~ while(j <= (i/j)): \par
~~~~~ if not(i $  \%  $j): break \par
~~~~~ j = j + 1 \par
~~ if (j > i/j) : print (i, " adalah bilangan prima") \par
~~ i = i + 1 \par
\vspace{12pt}
print ("Good bye!") \par
\vspace{12pt}
\vspace{\baselineskip}
Apabila program diatas Anda jalankan, maka akan menampilkan output seperti di bawah ini.\vspace{\baselineskip}
\vspace{\baselineskip}
 \par
2~ adalah bilangan prima \par
3~ adalah bilangan prima \par
5~ adalah bilangan prima \par
7~ adalah bilangan prima \par
11~ adalah bilangan prima \par
13~ adalah bilangan prima \par
17~ adalah bilangan prima \par
19~ adalah bilangan prima \par
23~ adalah bilangan prima \par
29~ adalah bilangan prima \par
\vspace{12pt}
Pengulangan adalah salah satu hal penting yang ada di bahasa pemrograman. Pengulangan digunakan misalnya untuk meng-update $  $nama $  $file $  $yang cukup banyak jumlahnya, atau mengakses piksel satu persatu pada gambar. \par
Python memiliki tiga jenis pengulangan yang wajib Anda cermati untuk membuat sebuah aplikasi dengan Python. Pengulangan yang pertama adalah $  $while. Dengan menggunakan $  $while, Anda dapat membuat kondisi tertentu untuk menghentikan $  $while. Biasanya $  $while $  $digunakan untuk melakukan $  $loopingyang tidak pasti. Coba lihat contoh berikut (Anda dapat menulisnya dalam sebuah $  $file, kemudian eksekusi $  $file $  $tersebut di konsol): \par
i = 0 \par
while True: \par
~~~ if i < 10: \par
~~~~~~~ print "Saat ini i bernilai: ", i \par
~~~~~~~ i = i + 1 \par
~~~ elif i >= 10: \par
~~~~~~~ break \par
\vspace{12pt}
Pada potongan kode diatas, $  $while $  $akan terus berputar selama i masih kurang dari 10. Jika sudah lebih dari 10 maka $  $while $  $akan berhenti. Pengulangan $  $whilejuga biasa digunakan di aplikasi konsol, untuk menahan $  $user $  $mengisikan semua input yang diperlukan dan baru akan berhenti setelah semua input dan proses interaksi berakhir. Jika kode diatas kita jalankan, maka $  $output-nya akan seperti ini: \par
\vspace{12pt}
Saat~ini i bernilai:  0 \par
Saat~ini i bernilai:  1 \par
Saat~ini i bernilai:  2 \par
Saat~ini i bernilai:  3 \par
Saat~ini i bernilai:  4 \par
Saat~ini i bernilai:  5 \par
Saat~ini i bernilai:  6 \par
Saat~ini i bernilai:  7 \par
Saat~ini i bernilai:  8 \par
Saat~ini i bernilai:  9 \par
\vspace{12pt}
Sekarang kita coba gunakan $  $for. Pengulangan $  $for $  $biasa digunakan untuk pengulangan yang sudah jelas banyaknya. Misal, Anda ingin mengulang sebuah pengulangan sampai 10 kali atau mengeluarkan semua hasil $  $query $  $dari $  $databasedi halaman HTML. Berikut ini adalah contoh kode untuk pengulangan $  $for: \par
for i in range(0, 10): \par
~~~ print i \par
Jika dijalankan maka kode diatas akan mengeluarkan $  $output $  $seperti ini: \par
\vspace{12pt}
0 \par
1 \par
2 \par
3 \par
4 \par
5 \par
6 \par
7 \par
8 \par
9 \par
Tidak hanya mengiterasi deretan angka, pengulangan $  $for $  $pun dapat Anda gunakan untuk mengulang sesuatu yang $  $iterable $  $seperti $  $list, $  $tuple, $  $dictionary, dan $  $iterable object $  $lainnya. Berikut ini kita ambil contoh dengan mengulang sebuah $  $list $  $yang berisi karakter anime Dragonball Super: \par
\vspace{12pt}
dragonball $  \_  $super $  \_  $character = ["Son Goku", "Vegeta", "Beerus", "Trunks", "Whiz", "Champa"] \par
for character in dragonball $  \_  $super $  \_  $character: \par
~~~ print character \par
\vspace{12pt}
Jika kita jalankan potongan kode tadi, maka $  $output-nya akan seperti berikut: \par
\vspace{12pt}
Son Goku \par
Vegeta \par
Beerus \par
Trunks \par
Whiz \par
Champa \par
For Loop \par
Seperti pada bahasa pemrograman lainnya, for loop sudah menjadi standar namun berbeda-beda tata cara penulisan nya di setiap pemrograman. \par
\vspace{12pt}
Sekarang kita langsung buat contoh di Python. $  $ \par
\vspace{12pt}
\vspace{12pt}
Contoh iterasi pada String  \par
\vspace{12pt}
for~n in 'Python':   \par
~~~ print 'Huruf :', n \par
\vspace{12pt}
  \par
iterasi pada List biasa \par
\vspace{12pt}
mobil = ['sedan', 'truk', 'angkot']  \par
for p in mobil: \par
~~~ print 'Mobil :', mobil \par
\vspace{12pt}
\vspace{12pt}
iterasi pada list melalui index \par
for i in range(len(mobil)): \par
~~~ print 'Mobil :', mobil[i] \par
\vspace{12pt}
iterasi angka / range \par
\vspace{12pt}
for a in range(1,10): \par
~~~~ print "Angka :", a \par
~~~~ if(a == 5):  $  \#  $ditambah conditional \par
~~~~~~~~ print "Saya dapat angka : ",a \par
\vspace{12pt}
iterasi loop nested \par
for a in range(1,10): \par
~~~ for x in range(11,20): \par
~~~~~~~~b~=~a~* x      \par
~~~~~~~ print "Angka :", b \par
\vspace{12pt}
loop dgn break \par
for letter in 'Python': \par
~~ if letter == 'h': \par
~~~~~ break \par
~~ print 'Current Letter :', letter \par
\vspace{12pt}
print "Good job !!!" \par
\vspace{12pt}
While Loop \par
WHile dipakai untuk looping dimana iterasi akan dilakukan selama kondisi yang diberikan benar. While ini juga bisa di pakai untuk Infinite loop. \par
\vspace{12pt}
Contoh While \par
count = 0 \par
while count < 100: \par
 $  $  $  $  $  $print "Count ke : ", count \par
 $  $  $  $  $  $count = count + 1 \par
\vspace{12pt}
infinite loop \par
''' \par
Set loop ini untuk kondisi dimana suatu syarat tidak pernah TRUE \par
''' \par
\vspace{12pt}
setvar =1 \par
while setvar == 1 \par
 $  $  $  $ input = input $  \_  $raw("Masukan angka :") \par
 $  $  $  $ print "Angka anda : ", input \par
\vspace{12pt}
loop diatas akan berhenti jika anda stop manual misal dgn CTRL+C di terminal \par
''' \par
ELSE statement di while loop. di Python kita bisa set WHile loop lalu dikasih kondisi \par
''' \par
count = 0 \par
while count < 5: \par
 $  $  $  $  $  $print "count : ",count \par
 $  $  $  $  $  $count = count + 1 \par
else: \par
 $  $  $  $ print "Lihat yang masuk sini apa : ",count \par
\vspace{12pt}
while dgn break \par
angka~=~10~~~~~~    \par
while~angka~>~0:~~~~~~~~~~     \par
~~  \par
~~ print 'Angka :', angka \par
~~ angka = angka -1 \par
~~ if angka == 7: \par
~~~~~ break \par
\vspace{\baselineskip}
\vspace{12pt}
\vspace{12pt}
