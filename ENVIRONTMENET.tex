
\sloppy
{\fontsize{14pt}{14pt}\selectfont ENVIRONTMENET SETUP \\} \par
\vspace{14pt}
\noindent 
{\fontsize{14pt}{14pt}\selectfont Download python terbaru \\} \par
\vspace{14pt}
\noindent 
{\fontsize{14pt}{14pt}\selectfont Install dan simpan di data C \\} \par
\vspace{14pt}
\noindent 
{\fontsize{14pt}{14pt}\selectfont Terus test apakah sudah terinstal atau belum, dengan cara buka cmd pada pc anda ketik python apabila masih not responed \\} \par
\vspace{14pt}
\noindent 
{\fontsize{14pt}{14pt}\selectfont Cek di control panel $  \textbackslash  $system and security $  \textbackslash  $system \\} \par
\vspace{14pt}
\noindent 
{\fontsize{14pt}{14pt}\selectfont Terus klik advanced system and security settings \\} \par
\vspace{14pt}
\noindent 
{\fontsize{14pt}{14pt}\selectfont Diteruskan dengan klik environment variables setelah muncul pop up cari path itu adalah nama dari variable nya \\} \par
\vspace{14pt}
\noindent 
{\fontsize{14pt}{14pt}\selectfont Lalu klik path tersebut setelah itu edit \\} \par
\vspace{14pt}
\noindent 
{\fontsize{14pt}{14pt}\selectfont Edit system variable \\} \par
\vspace{14pt}
\noindent 
{\fontsize{14pt}{14pt}\selectfont Jangan lupa copy semua terus paste kedalam notepad untuk mencegah kesalahan system \\} \par
\vspace{14pt}
\noindent 
{\fontsize{14pt}{14pt}\selectfont Setelah terbuka pop up cari path dan tambahkan python yang di download yang di simpan di data c tadi jangan lupa menggunakan tanda petik setelah itu \\} \par
\vspace{14pt}
\noindent 
{\fontsize{14pt}{14pt}\selectfont Klik ok pada system variable \\} \par
\noindent 
{\fontsize{14pt}{14pt}\selectfont Klik ok lagi di environment variable \\} \par
\vspace{14pt}
\noindent 
{\fontsize{14pt}{14pt}\selectfont Klik lagi ok nya pada system properties \\} \par
\vspace{14pt}
\noindent 
{\fontsize{14pt}{14pt}\selectfont Close control panel nya \\} \par
\vspace{14pt}
\noindent 
{\fontsize{14pt}{14pt}\selectfont Lanjutkan di cmd setelah python teks menggunakan petik akan muncul kata kata apabila proses telah berhasil \\} \par
\vspace{14pt}
\noindent 
{\fontsize{14pt}{14pt}\selectfont Cara install di linux atau mac os \\} \par
\vspace{14pt}
\noindent 
{\fontsize{14pt}{14pt}\selectfont Python tersedia di berbagai platform termasuk Linux dan Mac OS X. Mari kita mengerti bagaimana mengatur lingkungan Python kita. \\} \par
\vspace{14pt}
\noindent 
{\fontsize{14pt}{14pt}\selectfont Penyiapan Lingkungan Lokal \\} \par
\noindent 
{\fontsize{14pt}{14pt}\selectfont Buka jendela terminal dan ketik "python" untuk mengetahui apakah sudah terpasang dan versi mana yang terpasang. \\} \par
\vspace{14pt}
\noindent 
{\fontsize{14pt}{14pt}\selectfont Unix (Solaris, Linux, FreeBSD, AIX, HP / UX, SunOS, IRIX, dll.) \\} \par
\noindent 
{\fontsize{14pt}{14pt}\selectfont Menang 9x / NT / 2000 \\} \par
\noindent 
{\fontsize{14pt}{14pt}\selectfont Macintosh (Intel, PPC, 68K) \\} \par
\noindent 
{\fontsize{14pt}{14pt}\selectfont OS / 2 \\} \par
\noindent 
{\fontsize{14pt}{14pt}\selectfont DOS (beberapa versi) \\} \par
\noindent 
{\fontsize{14pt}{14pt}\selectfont PalmOS \\} \par
\noindent 
{\fontsize{14pt}{14pt}\selectfont Ponsel Nokia \\} \par
\noindent 
{\fontsize{14pt}{14pt}\selectfont Windows CE \\} \par
\noindent 
{\fontsize{14pt}{14pt}\selectfont OS Acorn / RISC \\} \par
\noindent 
{\fontsize{14pt}{14pt}\selectfont BeOS \\} \par
\noindent 
{\fontsize{14pt}{14pt}\selectfont Amiga \\} \par
\noindent 
{\fontsize{14pt}{14pt}\selectfont VMS / OpenVMS \\} \par
\noindent 
{\fontsize{14pt}{14pt}\selectfont QNX \\} \par
\noindent 
{\fontsize{14pt}{14pt}\selectfont VxWorks \\} \par
\noindent 
{\fontsize{14pt}{14pt}\selectfont Psion \\} \par
\noindent 
{\fontsize{14pt}{14pt}\selectfont Python juga telah porting ke Jawa. \\} \par
\noindent 
{\fontsize{14pt}{14pt}\selectfont Memasang Python \\} \par
\noindent 
{\fontsize{14pt}{14pt}\selectfont Distribusi Python tersedia untuk berbagai macam platform. Anda hanya perlu mendownload kode biner yang berlaku untuk platform Anda dan menginstal Python. \\} \par
\vspace{14pt}
\noindent 
{\fontsize{14pt}{14pt}\selectfont Jika kode biner untuk platform Anda tidak tersedia, Anda memerlukan kompiler C untuk mengkompilasi kode sumber secara manual. Kompilasi kode sumber menawarkan fleksibilitas lebih dalam hal pilihan fitur yang Anda butuhkan dalam instalasi Anda. \\} \par
\vspace{14pt}
\noindent 
{\fontsize{14pt}{14pt}\selectfont Berikut adalah ikhtisar singkat tentang menginstal Python di berbagai platform - \\} \par
\vspace{14pt}
\noindent 
{\fontsize{14pt}{14pt}\selectfont Instalasi Unix dan Linux \\} \par
\noindent 
{\fontsize{14pt}{14pt}\selectfont Berikut adalah langkah-langkah sederhana untuk menginstal Python di mesin Unix / Linux. \\} \par
\noindent 
{\fontsize{14pt}{14pt}\selectfont Ikuti link untuk mendownload kode sumber zip yang tersedia untuk Unix / Linux. \\} \par
\vspace{14pt}
\noindent 
{\fontsize{14pt}{14pt}\selectfont Download dan ekstrak file. \\} \par
\vspace{14pt}
\noindent 
{\fontsize{14pt}{14pt}\selectfont Mengedit Modul / Setup file jika Anda ingin menyesuaikan beberapa pilihan. \\} \par
\vspace{14pt}
\noindent 
{\fontsize{14pt}{14pt}\selectfont jalankan ./configure script \\} \par
\vspace{14pt}
\noindent 
{\fontsize{14pt}{14pt}\selectfont membuat \\} \par
\vspace{14pt}
\noindent 
{\fontsize{14pt}{14pt}\selectfont buat install \\} \par
\vspace{14pt}
\noindent 
{\fontsize{14pt}{14pt}\selectfont Ini menginstal Python di lokasi standar / usr / local / bin dan pustakanya di / usr / local / lib / pythonXX dimana XX adalah versi Python. \\} \par
\vspace{14pt}
\noindent 
{\fontsize{14pt}{14pt}\selectfont Instalasi Windows \\} \par
\noindent 
{\fontsize{14pt}{14pt}\selectfont Berikut adalah langkah-langkah untuk menginstal Python pada mesin Windows. \\} \par
\noindent 
{\fontsize{14pt}{14pt}\selectfont \vspace{\baselineskip}
Ikuti link untuk berkas installer python-XYZ.msi Windows dimana XYZ adalah versi yang perlu Anda instal. Untuk menggunakan installer python-XYZ.msi ini, sistem Windows harus mendukung Microsoft Installer 2.0. Simpan file installer ke komputer lokal Anda dan kemudian jalankan untuk mengetahui apakah mesin Anda mendukung MSI. Jalankan file yang didownload. Ini membawa wizard install Python, yang sangat mudah digunakan. Hanya menerima pengaturan default, tunggu sampai install selesai, dan selesai. Instalasi Macintosh Mac terbaru datang dengan Python terinstal, tapi mungkin beberapa tahun kedaluwarsa \\} \par
\noindent 
{\fontsize{14pt}{14pt}\selectfont Menyiapkan PATH \\} \par
\noindent 
{\fontsize{14pt}{14pt}\selectfont Program dan file eksekusi lainnya bisa berada di banyak direktori, jadi sistem operasi menyediakan jalur pencarian yang mencantumkan direktori yang dicari OS untuk executable. \\} \par
\vspace{14pt}
\noindent 
{\fontsize{14pt}{14pt}\selectfont Path disimpan dalam variabel lingkungan, yang merupakan string bernama yang dikelola oleh sistem operasi. Variabel ini berisi informasi yang tersedia untuk perintah shell dan program lainnya. \\} \par
\vspace{14pt}
\noindent 
{\fontsize{14pt}{14pt}\selectfont Variabel path dinamakan sebagai PATH di Unix atau Path in Windows (Unix bersifat caseensitive; Windows tidak). \\} \par
\vspace{14pt}
\noindent 
{\fontsize{14pt}{14pt}\selectfont Di Mac OS, installer menangani detail jalur. Untuk meminta juru bahasa Python dari direktori tertentu, Anda harus menambahkan direktori Python ke path Anda. \\} \par
\vspace{14pt}
\noindent 
{\fontsize{14pt}{14pt}\selectfont \vspace{\baselineskip}
Lingkungan Pembangunan Terpadu Anda dapat menjalankan Python dari lingkungan Graphical User Interface (GUI) juga, jika Anda memiliki aplikasi GUI di sistem Anda yang mendukung Python. Unix - IDLE adalah IDE Unix pertama untuk Python. Windows - PythonWin adalah antarmuka Windows pertama untuk Python dan merupakan IDE dengan GUI. Macintosh - Versi Macintosh dari Python beserta IDE IDLE tersedia dari situs utama, dapat didownload sebagai file MacBinary atau BinHex. Jika Anda tidak bisa mengatur lingkungan dengan baik, maka Anda dapat mengambil bantuan dari admin sistem Anda. Pastikan lingkungan Python benar diatur dan berfungsi dengan baik. Catatan - Semua contoh yang diberikan dalam bab berikutnya dijalankan dengan versi 2.4.3 Python yang tersedia pada rasa CentOS di Linux. Kami telah menyiapkan lingkungan Pemrograman Python secara online, sehingga Anda dapat mengeksekusi semua contoh online yang tersedia bersamaan saat Anda belajar teori. Merasa bebas untuk mengubah contoh apapun dan menjalankannya secara online. \\} \par
\vspace{14pt}
\noindent 
{\fontsize{14pt}{14pt}\selectfont \vspace{\baselineskip}
Salah satu hal terpenting yang akan Anda lakukan saat bekerja dengan bahasa pemrograman adalah menyiapkan lingkungan pengembangan yang memungkinkan Anda mengeksekusi kode yang Anda tulis. Tanpa ini, Anda tidak akan pernah dapat memeriksa pekerjaan Anda dan melihat apakah situs atau aplikasi Anda bebas dari kesalahan sintaksis.  $  $ Dengan Python, Anda juga memerlukan sesuatu yang disebut penerjemah yang mengubah kode Anda - yang membentuk keseluruhan aplikasi Anda - untuk sesuatu yang dapat dibaca dan dijalankan komputer. Tanpa penerjemah ini, Anda tidak memiliki cara untuk menjalankan kode Anda.  $  $ Untuk mengonversi kode Anda, Anda harus terlebih dahulu menggunakan shell Python, yang memanggil juru bahasa melalui sesuatu yang disebut "bang".  $  $ Sedangkan untuk membuat aplikasi atau file, ada dua cara untuk melakukan ini. Anda bisa membuat program menggunakan editor teks sederhana seperti WordPad, atau Notepad ++. Anda juga bisa membuat program menggunakan shell Python. Ada kelebihan dan kekurangan masing-masing metode \\} \par
\vspace{14pt}
\noindent 
{\fontsize{14pt}{14pt}\selectfont Tutorial Python dibuat untuk mengajarkan dasar-dasar bahasa pemrograman Python. Akhirnya, Tutorial Python akan menjelaskan bagaimana membangun aplikasi web, namun saat ini, Anda akan mempelajari dasar-dasar Python secara offline. Python bisa bekerja di Server Side (di server hosting website) atau di komputer Anda. Namun, Python tidak benar-benar bahasa pemrograman web. Artinya, banyak program Python tidak pernah dimaksudkan untuk digunakan secara online. Dalam tutorial Python ini, kita hanya akan membahas dasar-dasar Python dan bukan perbedaan keduanya \\} \par
\vspace{14pt}
\noindent 
{\fontsize{14pt}{14pt}\selectfont Python bekerja sama seperti dua kategori sebelumnya, PHP dan ColdFusion karena semuanya adalah bahasa pemrograman sisi server. Anda akan melihat dari tutorial \\} \par
\noindent 
{\fontsize{14pt}{14pt}\selectfont Pemrograman GUI: Python mendukung aplikasi GUI yang dapat dibuat dan dikirimkan ke banyak sistem panggilan, perpustakaan dan sistem windows, seperti Windows MFC, Macintosh, dan sistem X Window dari Unix. \\} \par
\vspace{14pt}
\noindent 
{\fontsize{14pt}{14pt}\selectfont Untuk memulai mengembangkan aplikasi web dengan Flask, saya sarankan belajar pemrograman python terlebih dahulu, supaya tidak terlalu susah ketika menggunakan Flask. Istilahnya kita mau membuat pupuh Sunda, minimal harus mengerti bahasa Sunda dulu kan? Saya di sini menggunakan Python 2.7.7, jadi untuk pengguna Python 3.x, mungkin harus sedikit menyesuaikan diri dengan tutorial ini. Dan satu lagi, mungkin harus membiasakan diri menggunakan terminal atau shell atau command prompt, untuk mempermudah beberapa perintah. Selain itu juga kita butuh internet untuk mengunduh framework Flask dan extensionnya. \\} \par
\noindent 
{\fontsize{14pt}{14pt}\selectfont Instalasi Flask \\} \par
\noindent 
{\fontsize{14pt}{14pt}\selectfont Kemudian kita akan menginstall framework Flask dan beberapa extensionnya. Supaya lebih mudah, kita bisa membuat sebuah virtual environment dan menginstall Flask di sana. Sekarang kita buat satu folder untuk mengerjakan project kita, misalnya kita sebut microblog. Kenapa? karena kita akan membuat microblog. $  $ $  $Nah selanjutnya, kita harus menginstall virtualenv terlebih dahulu. \\} \par
\noindent 
{\fontsize{14pt}{14pt}\selectfont Untuk yang menggunakan Mac OS X, bisa menggunakan:\vspace{\baselineskip}
sudo easy $  \_  $install virtualenv \\} \par
\noindent 
{\fontsize{14pt}{14pt}\selectfont Kalau menggunakan ubuntu, bisa dengan:\vspace{\baselineskip}
sudo apt-get install python-virtualenv \\} \par
\noindent 
{\fontsize{14pt}{14pt}\selectfont Nah, untuk Windows ini agak berbelit-belit. \\} \par
\noindent 
{\fontsize{14pt}{14pt}\selectfont Setelah kita menginstall virtualenv, sekarang kita buka terminal/shell (atau command prompt bagi pengguna Windows), kemudian masuk ke folder microblog yang sudah dibuat tadi, kemudian kita jalankan virtualenv untuk membuat virtual environment di folder tersebut.\vspace{\baselineskip}
virtualenv flask \\} \par
\noindent 
{\fontsize{14pt}{14pt}\selectfont Setelah perintah tersebut dijalankan, maka di dalam folder microblog tadi akan muncul folder flask yang berisi virtual environment python yang siap dijalankan untuk project ini. Selanjutnya, kita akan menginstall Flask dan extension yang akan digunakan dalam project ini. Untuk Linux atau OS X, gunakan perintah berikut ini: \\} \par
\vspace{14pt}
\noindent 
{\fontsize{14pt}{14pt}\selectfont Flask $  \textbackslash  $bin $  \textbackslash  $pip $  \textbackslash  $ install flask \\} \par
\vspace{14pt}
\noindent 
{\fontsize{14pt}{14pt}\selectfont Flask $  \textbackslash  $bin $  \textbackslash  $pip $  \textbackslash  $ install flask-login \\} \par
\vspace{14pt}
\noindent 
{\fontsize{14pt}{14pt}\selectfont Flask $  \textbackslash  $bin $  \textbackslash  $pip $  \textbackslash  $ install flask-openid \\} \par
\vspace{14pt}
\noindent 
{\fontsize{14pt}{14pt}\selectfont Flask $  \textbackslash  $bin $  \textbackslash  $pip $  \textbackslash  $ install flask-mail \\} \par
\vspace{14pt}
\noindent 
{\fontsize{14pt}{14pt}\selectfont Flask $  \textbackslash  $bin $  \textbackslash  $pip $  \textbackslash  $ install flask-sqlalchemy \\} \par
\vspace{14pt}
\noindent 
{\fontsize{14pt}{14pt}\selectfont Flask $  \textbackslash  $bin $  \textbackslash  $pip $  \textbackslash  $ install sqlalchemy \\} \par
\vspace{14pt}
\noindent 
{\fontsize{14pt}{14pt}\selectfont Flask $  \textbackslash  $bin $  \textbackslash  $pip $  \textbackslash  $ install sqlalchemy-migrate \\} \par
\vspace{14pt}
\noindent 
{\fontsize{14pt}{14pt}\selectfont Flask $  \textbackslash  $bin $  \textbackslash  $pip $  \textbackslash  $ install flask-whooshalchemy \\} \par
\vspace{14pt}
\noindent 
{\fontsize{14pt}{14pt}\selectfont Flask $  \textbackslash  $bin $  \textbackslash  $pip $  \textbackslash  $ install flask-wtf \\} \par
\vspace{14pt}
\noindent 
{\fontsize{14pt}{14pt}\selectfont Flask $  \textbackslash  $bin $  \textbackslash  $pip $  \textbackslash  $ install flask-babel \\} \par
\vspace{14pt}
\noindent 
{\fontsize{14pt}{14pt}\selectfont Flask $  \textbackslash  $bin $  \textbackslash  $pip $  \textbackslash  $ install guess $  \_  $language \\} \par
\vspace{14pt}
\noindent 
{\fontsize{14pt}{14pt}\selectfont Flask $  \textbackslash  $bin $  \textbackslash  $pip $  \textbackslash  $ install flipflop \\} \par
\vspace{14pt}
\noindent 
{\fontsize{14pt}{14pt}\selectfont Flask $  \textbackslash  $bin $  \textbackslash  $pip $  \textbackslash  $ install coverage \\} \par
\vspace{14pt}
\noindent 
{\fontsize{14pt}{14pt}\selectfont Kalo untuk windows bisa menggunaka langkah ini \\} \par
\vspace{14pt}
\noindent 
{\fontsize{14pt}{14pt}\selectfont Flask $  \textbackslash  $scripts $  \textbackslash  $pip $  \textbackslash  $ install flask \\} \par
\vspace{14pt}
\noindent 
{\fontsize{14pt}{14pt}\selectfont Flask $  \textbackslash  $scripts $  \textbackslash  $pip $  \textbackslash  $ install flask-login \\} \par
\vspace{14pt}
\noindent 
{\fontsize{14pt}{14pt}\selectfont Flask $  \textbackslash  $ Scripts  $  \textbackslash  $pip $  \textbackslash  $ install flask-openid \\} \par
\vspace{14pt}
\noindent 
{\fontsize{14pt}{14pt}\selectfont Flask $  \textbackslash  $ Scripts  $  \textbackslash  $pip $  \textbackslash  $ install flask-mail \\} \par
\vspace{14pt}
\noindent 
{\fontsize{14pt}{14pt}\selectfont Flask $  \textbackslash  $ Scripts  $  \textbackslash  $pip $  \textbackslash  $ install flask-sqlalchemy \\} \par
\vspace{14pt}
\noindent 
{\fontsize{14pt}{14pt}\selectfont Flask $  \textbackslash  $ Scripts  $  \textbackslash  $pip $  \textbackslash  $ install sqlalchemy \\} \par
\vspace{14pt}
\noindent 
{\fontsize{14pt}{14pt}\selectfont Flask $  \textbackslash  $ Scripts  $  \textbackslash  $pip $  \textbackslash  $ install sqlalchemy-migrate \\} \par
\vspace{14pt}
\noindent 
{\fontsize{14pt}{14pt}\selectfont Flask $  \textbackslash  $ Scripts  $  \textbackslash  $pip $  \textbackslash  $ install flask-whooshalchemy \\} \par
\vspace{14pt}
\noindent 
{\fontsize{14pt}{14pt}\selectfont Flask $  \textbackslash  $ Scripts  $  \textbackslash  $pip $  \textbackslash  $ install flask-wtf \\} \par
\vspace{14pt}
\noindent 
{\fontsize{14pt}{14pt}\selectfont Flask $  \textbackslash  $ Scripts  $  \textbackslash  $pip $  \textbackslash  $ install flask-babel \\} \par
\vspace{14pt}
\noindent 
{\fontsize{14pt}{14pt}\selectfont Flask $  \textbackslash  $ Scripts  $  \textbackslash  $pip $  \textbackslash  $ install guess $  \_  $language \\} \par
\vspace{14pt}
\noindent 
{\fontsize{14pt}{14pt}\selectfont Flask $  \textbackslash  $ Scripts  $  \textbackslash  $pip $  \textbackslash  $ install flipflop \\} \par
\vspace{14pt}
\noindent 
{\fontsize{14pt}{14pt}\selectfont Flask $  \textbackslash  $Scripts $  \textbackslash  $pip $  \textbackslash  $ install coverage \\} \par
\vspace{14pt}
\vspace{14pt}
\noindent 
{\fontsize{14pt}{14pt}\selectfont Setelah selesai menginstal flask bisa membuat ata mengembangkan sebuah aplikasi kita.  \\} \par
\noindent 
{\fontsize{14pt}{14pt}\selectfont Contohnya membuat hello world \\} \par
\noindent 
{\fontsize{14pt}{14pt}\selectfont Klik folder microblog buat didalam folder beberapa folder untuk struktur aplikasi tersebut. \\} \par
\vspace{14pt}
\noindent 
{\fontsize{14pt}{14pt}\selectfont Lalu code program python init dan lain lain untuk menjalankan sebuah program hello world untuk mengisi form nya. \\} \par
\noindent 
{\fontsize{14pt}{14pt}\selectfont From flask import flask \\} \par
\vspace{14pt}
\noindent 
{\fontsize{14pt}{14pt}\selectfont App = flask ( $  \_  $ $  \_  $name) $  \_  $ $  \_  $) \\} \par
\vspace{14pt}
\noindent 
{\fontsize{14pt}{14pt}\selectfont From app import views \\} \par
\vspace{14pt}
\noindent 
{\fontsize{14pt}{14pt}\selectfont Kode diataas ini merupakan objek flask yang di buat lalu views ini menginport program. \\} \par
\vspace{14pt}
\noindent 
{\fontsize{14pt}{14pt}\selectfont From app import app \\} \par
\vspace{14pt}
\noindent 
{\fontsize{14pt}{14pt}\selectfont @app.route (‘/’) \\} \par
\vspace{14pt}
\noindent 
{\fontsize{14pt}{14pt}\selectfont @app.route(‘/indexef index(): \\} \par
\vspace{14pt}
\noindent 
{\fontsize{14pt}{14pt}\selectfont Return  $ " $hello, world! $ " $ \\} \par
\vspace{14pt}
\noindent 
{\fontsize{14pt}{14pt}\selectfont Kita tambah code lagi untuk menjalankan web server python nya \\} \par
\vspace{14pt}
\vspace{14pt}
\noindent 
{\fontsize{14pt}{14pt}\selectfont  $  \#  $!flask/bin/python \\} \par
\noindent 
{\fontsize{14pt}{14pt}\selectfont From app import app \\} \par
\noindent 
{\fontsize{14pt}{14pt}\selectfont App.run(debug=true) \\} \par
\vspace{14pt}
\noindent 
{\fontsize{14pt}{14pt}\selectfont Setelah itu kita jalan kan web server nya hanya di run cmd lalu setelah beres di cmd kita menuju local di halaman browser maka akan muncul hello world! \\} \par
\vspace{14pt}
\vspace{14pt}
