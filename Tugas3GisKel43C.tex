% Tugas 3 GIS Kelompok 4 
% Akbar Pambudi Utomo (1154094)


\section{Basic Syntax}
\subsection{Pengenalan Python}
Python dikembangkan oleh Guido van Rossum (programmer kelahiran belanda) pada tahun 1990 di CWI, Amsterdam sebagai
kelanjutan dari bahasa pemrograman ABC. Python adalah bahasa pemrograman interpretatif yang dianggap mudah dipelajari 
serta berfokus pada keterbacaan kode. Dengan kata lain, Python diklaim sebagai bahasa perograman yang memiliki kode-kode
pemrograman yang sangat jelas, lengkap dan mudah untuk dipahami. Python secara umum berbentuk pemrograman berorientasi
objek, pemrograman imperatif, dan pemrograman fungsional. Python dapat digunakan dalam berbagai pengembangan perangkat 
lunak dan dapat berjalan diberbagai platform sistem operasi.
Pengenal Python adalah nama yang digunakan untuk mengidentifikasi variabel, fungsi, kelas, modul ataupun objek lainnya.
pengenal dimulai dengan huruf \"A\" sampai \"Z\" atau huruf \"a\" sampai \"z\" atau garis \"_\" diikuti oleh nol atau lebih
huruf, garis bawah dan angka (0 sampai 9). Python tidak mengisinkan karakter tanda baca seperti \"@\", \"$\", san \"%\" 
dalam pengenal karena Python adalah bahasa pemrograman yang sensitif.
