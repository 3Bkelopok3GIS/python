% Tugas 3 GIS Kelompok 4 
% Akbar Pambudi Utomo (1154094)
% Pebridayanti Hasibuan (1154118)
% Andi Nurfadillah Ali (1154041)


\section{Basic Syntax}
\subsection{Pengenalan Python}
Python dikembangkan oleh Guido van Rossum (programmer kelahiran belanda) pada tahun 1990 di CWI, Amsterdam sebagai
kelanjutan dari bahasa pemrograman ABC. Python adalah bahasa pemrograman interpretatif yang dianggap mudah dipelajari 
serta berfokus pada keterbacaan kode. Dengan kata lain, Python diklaim sebagai bahasa perograman yang memiliki kode-kode
pemrograman yang sangat jelas, lengkap dan mudah untuk dipahami. Python secara umum berbentuk pemrograman berorientasi
objek, pemrograman imperatif, dan pemrograman fungsional. Python dapat digunakan dalam berbagai pengembangan perangkat 
lunak dan dapat berjalan diberbagai platform sistem operasi.
Pengenal Python adalah nama yang digunakan untuk mengidentifikasi variabel, fungsi, kelas, modul ataupun objek lainnya.
pengenal dimulai dengan huruf \"A\" sampai \"Z\" atau huruf \"a\" sampai \"z\" atau garis \"_\" diikuti oleh nol atau lebih
huruf, garis bawah dan angka (0 sampai 9). Python tidak mengisinkan karakter tanda baca seperti \"@\", \"$\", san \"%\" 
dalam pengenal karena Python adalah bahasa pemrograman yang sensitif.
\end

\subsection {Basic syntax}
Visual Basic adalah salah suatu developement tools untuk membangun aplikasi dalam lingkungan Windows. Dalam pengembangan aplikasi, Visual Basic menggunakan pendekatan Visual untuk merancang user interface dalam bentuk form, sedangkan untuk kodingnya menggunakan dialek bahasa Basic yang cenderung mudah dipelajari. Visual Basic telah menjadi tools yang terkenal bagi para pemula maupun para developer. Dalam lingkungan Window’s User-interface sangat memegang peranan penting,karena dalam pemakaian aplikasi yang kita buat, pemakai senantiasa berinteraksi dengan User-interface tanpa menyadari bahwa dibelakangnya berjalan instruksi-instruksi program yang mendukung tampilan dan proses yang dilakukan.
Pada pemrograman Visual, pengembangan aplikasi dimulai dengan pembentukkan user interface, kemudian mengatur properti dari objek-objek yang digunakan dalam user interface, dan baru dilakukan penulisan kode program untuk menangani kejadian-kejadian (event). Tahap pengembangan aplikasi demikian dikenal dengan istilah pengembangan aplikasi dengan pendekatan Bottom Up.
\subsection {permasalahan yang terdapat di basic syntax}
Permasalahan yang dibahas adalah membuat suatu pengkodean dengan
menggunakan algoritma RC4. Masalah enkripsi data dengan algoritma RC4
muncul ketika proses enkripsi dalam sistem sedang. Pembahasan masalah lebih
ditekankan pada proses indeks kerja algoritma RC4.
Algoritma RC4 mengenkripsi dengan mengombinasikannya dengan
plainteks dengan menggunakan bit-wise Xor (Exclusive-or). RC4 menggunakan
panjang kunci dari 1 sampai 256 byte yang digunakan untuk menginisialisasikan
tabel sepanjang 256 byte. Tabel ini digunakan untuk generasi yang berikut dari
pseudo random yang menggunakan XOR dengan plaintext untuk menghasilkan
ciphertext. Masing - masing elemen dalam tabel saling ditukarkan minimal sekali.
Proses dekripsinya dilakukan dengan cara yang sama (karena Xor merupakan
fungsi simetrik). Untuk menghasilkan keystream, cipher menggunakan state
internal yang meliputi dua bagian :
1. Tahap key scheduling dimana state automaton diberi nilai awal berdasar kan
kunci enkripsi.
State yang diberi nilai awal berupa array yang merepresentasikan suatu
permutasi dengan 256 elemen, jadi hasil dari algoritma KSA adalah
permutasi awal. Array yang mempunyai 256 elemen ini (dengan indeks 0
sampai dengan 255) dinamakan S. Berikut adalah algoritma KSA dalam 
29
bentuk pseudo-code dimana key adalah kunci enkripsi dan keylength adalah besar kunci enkripsi dalam bytes (untuk kunci 128 bit, keylength = 16).
\begin {equation}
for i = 0 to 255
S [i] := i
j := 0
for i = 0 to 255
j := (j + S[i] + key [I mod keylenght] ) mod 256
swap (S[i], S[j])
\end {equation}
2. Tahap pseudo-random generation dimana state automaton beroperasi dan
outputnya menghasilkan keystream. Setiap putaran, bagian keystream sebesar
1 byte (dengan nilai antara 0 sampai dengan 255) dioutput oleh PRGA
berdasarkan state S. Berikut adalah algoritma PRGA dalam bentuk pseudocode:
\begin {equation}
i := 0
j := 0
loop
i := ( i + 1 ) mod 256
j := ( j + S[i] ) mod 256
swap ( S[i], S[j] )
output S[ (S[i] + S[j]) mod 256]
\end {equation}
\end 
