% Tugas 3 GIS Kelompok 4 
% Akbar Pambudi Utomo (1154094)
% Pebridayanti Hasibuan (1154118)
% Andi Nurfadillah Ali (1154041)
% Julham Ramadhana (1154069)
% Hanna Theresia Siregar (1154009)
% Andi Wadi Afriyandika (1154113)


\section{Basic Syntax}
\subsection{Pengenalan Python}
Python dikembangkan oleh Guido van Rossum (programmer kelahiran belanda) pada tahun 1990 di CWI, Amsterdam sebagai
kelanjutan dari bahasa pemrograman ABC. Python adalah bahasa pemrograman interpretatif yang dianggap mudah dipelajari 
serta berfokus pada keterbacaan kode. Dengan kata lain, Python diklaim sebagai bahasa perograman yang memiliki kode-kode
pemrograman yang sangat jelas, lengkap dan mudah untuk dipahami. Python secara umum berbentuk pemrograman berorientasi
objek, pemrograman imperatif, dan pemrograman fungsional. Python dapat digunakan dalam berbagai pengembangan perangkat 
lunak dan dapat berjalan diberbagai platform sistem operasi.
Pengenal Python adalah nama yang digunakan untuk mengidentifikasi variabel, fungsi, kelas, modul ataupun objek lainnya.
pengenal dimulai dengan huruf \"A\" sampai \"Z\" atau huruf \"a\" sampai \"z\" atau garis \"_\" diikuti oleh nol atau lebih
huruf, garis bawah dan angka (0 sampai 9). Python tidak mengisinkan karakter tanda baca seperti \"@\", \"$\", san \"%\" 
dalam pengenal karena Python adalah bahasa pemrograman yang sensitif. 
Bahasa Python memiliki banyak kesamaan dengan Pemrograman Perl, C, dan Java. Namun, ada beberapa perbedaan yang pasti antara 
bahasa.
Python adalah salah satu bahasa pemrograman tingkat tinggi. Yang mana, Python berada diperingkat ke-5 yaitu  pada tahun 2016. Python juga banyak digunakan oleh para programmer. Python dapat pula digunakan untuk membangun sebuah aplikasi desktop, web, mobile dan sebagainya. Dan untuk membangun aplikasi seperti aplikasi berbasi web, maka programmer dapat menggunakan framework ataupun tanpa framework. Sehingga apabila membangun aplikasi web tanpa framework, terdapat juga kekurangan yaitu salah satunya membutuhkan waktu yang relatif lama karena harus membuatnya dari awal. Dan berdasarkan kekurangan tersebut maka dibuatlah framework agar dapat membangun web menjadi lebih cepat, serta terstruktur dan reusable.
Fitur Python sendiri mudah dipelajari karena Python memiliki beberapa kata kunci, struktur sederhana, dan sintaks yang jelas. Hal ini memungkinkan siswa untuk mengambil bahasa dengan cepat. Mudah dibaca karena kode Python lebih jelas dan terlihat oleh mata. Mudah dipelihara karena kode sumber Python cukup mudah untuk dipelihara.
Perpustakaan standar yang luas: sebagian besar perpustakaan Python sangat portabel dan kompatibel dengan platform cross-platform di UNIX, Windows, dan Macintosh. 
Python sendiri tidak mengizinkan karakter tanda baca seperti @, $, dan % dalam pengenal. Python merupakan  bahasa pemrograman yang termasuksensitif. Dengan demikian, Tenaga Kerja adalah dua pengidentifikasi yang berbeda dengan Python. Pengenal Python adalah nama yang digunakan untuk mengidentifikasi variabel, fungsi, kelas, modul atau objek lainnya. Pengenal dimulai dengan huruf A sampai Z atau huruf a sampai z atau garis bawah (_) diikuti oleh nol atau lebih huruf, garis bawah dan angka (0 sampai 9).
\subsection{keunggulan phyton}
\begin{enumerate}
    \item Tidak ada tahapan kompilasi dan penyambungan (link) sehingga kecepatan perubahan pada masa pembuatan sistem aplikasi meningkat. 
    \item Tidak ada deklarasi tipe data yang merumitkan sehingga program menjadi lebih sederhana, singkat, dan fleksible. 
    \item Manajemen memori otomatis yaitu kumpulan sampah memori sehingga dapat menghindari pencacatan kode. 
    \item Tipe data dan operasi tingkat tinggi yaitu kecepatan pembuatan sistem aplikasi menggunakan tipe objek yang telah ada. 
    \item Pemrograman berorientasi objek. 
    \item Pelekatan dan perluasan dalam C.
    \item Terdapat kelas, modul, eksepsi sehingga terdapat dukungan pemrograman skala besar secara modular.
    \item Pemuatan dinamis modul C sehingga ekstensi menjadi sederhana dan berkas biner yang 
    \item Pemuatan kembali secara dinamis modul phyton seperti memodifikasi aplikasi tanpa menghentikannya. 
    \item Model objek universal kelas Satu. Konstruksi pada saat aplikasi berjalan. 
    \item Interaktif, dinamis dan alamiah.\item Akses hingga informasi interpreter. 
    \item Portabilitas secara luas seperti pemrograman antar platform tanpa ports.
    \item Kompilasi untuk portable kode byte sehingga kecepatan eksekusi bertambah dan melindungi kode sumber. 
    \item Antarmuka terpasang untuk pelayanan keluar seperti perangkat Bantu system, GUI, persistence, database, dll. 
\end{enumerate}
\subsection{konvensi penamaan untuk pengenal Python}
\begin{enumerate}
    \item Nama kelas dimulai dengan huruf besar. Semua pengenal lainnya mulai dengan huruf kecil.
    \item Memulai pengenal dengan satu garis bawah terkemuka menunjukkan bahwa pengenal bersifat pribadi.
    \item Memulai pengenal dengan dua garis bawah terkemuka menunjukkan pengenal yang sangat pribadi.
    \item Jika pengenal juga diakhiri dengan dua tanda garis bawah, identifier adalah bahasa yang didefinisikan nama khusus
\end{enumerate}
\subsection{Python Software Foundation}
Python software foundation adalah sebuah organisasi non-profit yang dibentuk sebagai pemegang Hak cipta intelektual Python sejak versi 2.1 dan dengan demikian mencegah Python dimiliki oleh Perusahaan momersial. Salah satu fitur yang tersedia pada Python adalah sebagai bahasa pemrograman dinamis yang dilengkapi dengan manajemen memori otomatis. Seperti halnya pada bahasa pemrograman dinamis lainnya, Python umumnya digunakan sebagai bahasa skrip meskk pada prakteknya penggunaan bahasa ini lebih luas mencakup konteks pemanfaatan yang umumnya tidak dilakukan dengan menggunakan bahasa skrip. Python dapat digunakan untuk berbagai keperluan pengembangan perangkat lunak dan dapat berjalan diberbagai platform sistem operasi.

\subsubsection{Kata-kata yang dicadangkan}
Ini adalah kata-kata yang dicadangkan dan Anda tidak dapat menggunakannya sebagai konstan atau variabel 
atau nama pengenal lainnya. Semua kata kunci Python mengandung huruf kecil saja.
\begin{enumerate}
    \item Dan, exec, Tidak
    \item Menegaskan, akhirnya, atau
    \item Istirahat, untuk, lulus
    \item Kelas, dari, mencetak
    \item Terus, global, menaikkan
    \item def, jika, kembali
    \item del, impor, mencoba
    \item elif, di, sementara
    \item lain, aku s, dengan
    \item kecuali, lambda, menghasilkan
    \end{enumerate}

\subsection{Program Python}
Mari kita jalankan program dalam mode pemrograman yang berbeda. pemrograman mode interaktif memohon interpreter tanpa melewatkan file script sebagai parameter menampilkan prompt berikut :
    \begin{verbatim}
          Python
          
          Python 2.4.3 (#1,Okt 20 2017, 19:14:34)
          [GCC 4.1.2 20080704 (Red Hat 4.1.2.-48)] on Linux2
          Type "help", "copyright", "credits", or "license" for more information.
          
          Ketik teks berikut pada prompt Python dan tekan Enter :
          
          print "Hello, Python!"
          
     \end{verbatim}
Jika anda menjalankan versi baru Python, Anda perlu menggunakan pernyataan cetak dengan tanda kurung seperti pada cetak $("Hallo, Python!");$. Namun dengan versi Python 2.4.3, ini hasilnya sebagai berikut :
Hello, Python!

\subsubsuction{Masukan / Keluaran}
Berikut ini merupakan contoh program masukan ataupun keluaran:
\begin{verbatim}
    Contoh masukan : 
        nama = input("Masukkan nama Anda: ")
    Contoh keluaran :
        print ("Halo", nama, ":)"
       Halo Dunia
Perintah ini biasanya digunakan untuk menguji keberhasilan pemasangan Python dalam komputer.
    print ("Halo dunia!")
Keluaran yang seharusnya ditampilkan adalah seperti di bawah ini.
    Halo dunia!
\end{verbatim}


\subsection{Fitur Python}
Fitur Python meliputi :
    \begin{enumerate}
        \item Mudah dipelajari : Python memiliki beberapa kata kunci, struktur sederhana, dan sintaks yang jelas. Hal ini memungkinkan
        pengguna untuk mengerti bahasa dengan cepat.
        \item Mudah dibaca : kode Python lebih jelas dan terlihat oleh mata.
        \item Mudah dipelihara : kode sumber Python cukup mudah untuk dipelihara.
        \item Perpustakaan standar yang luas : sebagian besar perpustakaan Python sangat portabel dan kompatibel dengan platform cross-           platform di UNIX, Windows, Macintosh.
        \item Mode Interaktif : Python memiliki dukungan untuk mode interaktif yang memungknkan pengujian interaktif dan debugging dari         cuplikan kode.
        \item Portable : Python dapat berjalan di berbagai platform perangkat keras dan memiliki anarmuka yang sama pada semua platform.
        \item Dapat diperpanjang : Anda dapat menambahkan modul tingkat rendah ke penerjemah Python. Modul ini memugkinkan Programmer            untuk menambahkan atau menyeseuaikan alat mereka agar lebih efisien.
     \end{enumerate}

\subsection{Garis dan indentasi}
Python tidak memberikan tanda kurung untuk menunjukkan blok kode untuk definisi dan alur kelas dan fungsi
kontrol. Blok kode dilambangkan dengan garis indentasi, yang ditegakkan secara kaku.
Jumlah ruang dalam indentasi bervariasi, namun semua pernyataan di dalam blok harus
menjulurkan jumlah yang sama, Misalnya :
    \begin{verbatim}
            if True:
                print "True"
            else:
               print "False"
    \end{verbatim}
    
Namun, Blok tersebut menghasilkan kesalahan
    \begin{verbatim}
            if True:
                print "Answer"
                print "True"
            else:
                print "Answer"
               print "False"
    \end{verbatim}
Jadi, dengan Python semua garis kontinu yang menjorok dengan jumlah spasi yang sama akan membentuk satu blok. Contoh berikut memiliki berbagai blok pernyataan
Note : Jangan mencoba memahami logika pada saat ini. Pastikan Anda memahami berbagai blok meskipun tanpa tanda kurung.

\subsection{Deklarasi Tipe Sederhana dan Tipe Casting}
    Hanya tipe numerik dari int dan float yang akan digunakan di sini:int memegang nilai integer dan tidak ada titik desimal: 
    float menyimpan nilai floating point dan memiliki titik desimal.Kedua tipe memiliki representasi internal yang berbeda. 
    Sebelum digunakan variabel alfanumerik harus dinyatakan sebagai tipe tertentu dengan:
\begin{enumerate}
    \item Menetapkannya menjadi konstan seperti dalam a = 0,0
    \item Makin banyak tugas seperti pada a, b, c = 0,0,1,2
    \item Menugaskannya ke nilai ekspresi seperti pada a = b / c, di mana b dan c sebelumnya telah diumumkan
 \end{enumerate}
Variabel dari satu jenis dapat dilemparkan sebagai yang lain dengan menggunakan fungsi int () dan float () \cite{gray2017snake}

\subsection {Basic syntax}
Visual Basic adalah salah suatu developement tools untuk membangun aplikasi dalam lingkungan Windows. Dalam pengembangan aplikasi, Visual Basic menggunakan pendekatan Visual untuk merancang user interface dalam bentuk form, sedangkan untuk kodingnya menggunakan dialek bahasa Basic yang cenderung mudah dipelajari. Visual Basic telah menjadi tools yang terkenal bagi para pemula maupun para developer. Dalam lingkungan Window’s User-interface sangat memegang peranan penting,karena dalam pemakaian aplikasi yang kita buat, pemakai senantiasa berinteraksi dengan User-interface tanpa menyadari bahwa dibelakangnya berjalan instruksi-instruksi program yang mendukung tampilan dan proses yang dilakukan.
Pada pemrograman Visual, pengembangan aplikasi dimulai dengan pembentukkan user interface, kemudian mengatur properti dari objek-objek yang digunakan dalam user interface, dan baru dilakukan penulisan kode program untuk menangani kejadian-kejadian (event). Tahap pengembangan aplikasi demikian dikenal dengan istilah pengembangan aplikasi dengan pendekatan Bottom Up.
\subsection {permasalahan yang terdapat di basic syntax}
Permasalahan yang dibahas adalah membuat suatu pengkodean dengan
menggunakan algoritma RC4. Masalah enkripsi data dengan algoritma RC4
muncul ketika proses enkripsi dalam sistem sedang. Pembahasan masalah lebih
ditekankan pada proses indeks kerja algoritma RC4.
Algoritma RC4 mengenkripsi dengan mengombinasikannya dengan
plainteks dengan menggunakan bit-wise Xor (Exclusive-or). RC4 menggunakan
panjang kunci dari 1 sampai 256 byte yang digunakan untuk menginisialisasikan
tabel sepanjang 256 byte. Tabel ini digunakan untuk generasi yang berikut dari
pseudo random yang menggunakan XOR dengan plaintext untuk menghasilkan
ciphertext. Masing - masing elemen dalam tabel saling ditukarkan minimal sekali.
Proses dekripsinya dilakukan dengan cara yang sama (karena Xor merupakan
fungsi simetrik). Untuk menghasilkan keystream, cipher menggunakan state
internal yang meliputi dua bagian :
1. Tahap key scheduling dimana state automaton diberi nilai awal berdasar kan
kunci enkripsi.
State yang diberi nilai awal berupa array yang merepresentasikan suatu
permutasi dengan 256 elemen, jadi hasil dari algoritma KSA adalah
permutasi awal. Array yang mempunyai 256 elemen ini (dengan indeks 0
sampai dengan 255) dinamakan S. Berikut adalah algoritma KSA dalam 
29
bentuk pseudo-code dimana key adalah kunci enkripsi dan keylength adalah besar kunci enkripsi dalam bytes (untuk kunci 128 bit, keylength = 16).
\begin {equation}
for i = 0 to 255
S [i] := i
j := 0
for i = 0 to 255
j := (j + S[i] + key [I mod keylenght] ) mod 256
swap (S[i], S[j])
\end {equation}
2. Tahap pseudo-random generation dimana state automaton beroperasi dan
outputnya menghasilkan keystream. Setiap putaran, bagian keystream sebesar
1 byte (dengan nilai antara 0 sampai dengan 255) dioutput oleh PRGA
berdasarkan state S. Berikut adalah algoritma PRGA dalam bentuk pseudocode:
\begin {equation}
i := 0
j := 0
loop
i := ( i + 1 ) mod 256
j := ( j + S[i] ) mod 256
swap ( S[i], S[j] )
output S[ (S[i] + S[j]) mod 256]
\end {equation}
\end

\subsection {bentuk-bentuk basic syntax}
Bentuk-bentuk kondisi yang ada pada visual basic adalah :
A.Kondisi Perulangan
Dalam pemrograman ada kalanya kita memerlukan perulangan untuk melakukan suatu perintah yang sama untuk beberapa kali, misalkan pada program untuk mencari data maka diperlukan perulangan untuk mencari data dari record awal sampai record akhir atau sampai data yang dicari ditemukan. 
Perhatikan contoh sederhana yang menunjukkan penggunaan kondisi perulangan dalam program berikut ini :

\begin {equation}
Top of Form
Private Sub Form_Load()
MsgBox \“ini adalah pesan ke 1\”
MsgBox \“ini adalah pesan ke 2\”
MsgBox \“ini adalah pesan ke 3\”
MsgBox \“ini adalah pesan ke 4\”
MsgBox \“ini adalah pesan ke 5\”
End Sub
\end {equation}

Kode program di atas adalah kode program yang digunakan untuk menampilkan pesan sebanyak 5 kali ketika program di load. Bayangkan jika pesan yang ingin ditampilkan bukan 5 kali tetapi 1000 kali, pastinya kita akan kesusahan jika harus menulis kode program \“ MsgBox \“ini adalah pesan ke #/” /“ Sebanyak 1000 kali. Untuk mempersingkat kode program maka sebenarnya kita tidak perlu menulis program sebanyak 1000 baris, kita cukup menulis 3 baris program yang hasilnya akan menampilkan pesan sebanyak 1000 kali yaitu sebagai berikut :

\begin {equation}
For i = 1 To 1000
MsgBox \“ini adalah pesan ke\ ” & i
Next i
Sehingga source kodenya menjadi :
Private Sub Form_Load()
For i = 1 To 1000
MsgBox \“ini adalah pesan ke \” & i
Next i
End Sub
\end {equation}
Dengan menggunakan struktur kondisi perulangan seperti source code di atas maka ketika program di load maka program akan menjalankan perulangan dan menampilkan pesan \“ini adalah pesan ke 1\”,angka 1 pada pesan karena pada perulangan For … Next nilai awalnya adalah 1 kemudian program menjalankan perintah Next i sehingga sekarang nilai i menjadi 2, kemudian program menampilkan pesan \“ini adalah pesan ke 2\”, kemudian begitu seterusnya sampai nilai i = 1000 dan program menampilkan pesan \“ini adalah pesan ke 1000\”, karena nilai i = nilai akhir yaitu 1000 maka program keluar dariperulangan. Dari contoh di atas dapat disimpulkan bahwa sebuah perulangan memiliki kondisi awal dan kondisi akhir, dan perulangan akan berjalan dan berhenti jika kondisi akhir terpenuhi.
1.Macam – Macam Bentuk Perulangan
Dalam visual basic terdapat beberapa macam struktur kondisi perulangan, diantaranya adalah Do … Loop dan For … Next. Untuk lebih jelasnya berikut adalah macam – macam bentuk perulangan dalam visual basic :
A.Do While … Loop
Digunakan sebagai statement kondisi dengan banyak pilihan.
Select Case nilai_angka
\begin {equation}
Case 0 to 50
nilai huruf = \"D\"
Case Is <= 70
nilai huruf = \"C\"
Case Is <= 80
nilai huruf = \"B\"
Else Case
nilai huruf = \"A\"
End select
\end {equation}

Len()
Digunakan untuk mendapatkan informasi panjang sebuah String.
Syntax dasar :
Len(string)

Contoh :
\bin {equation}
Dim str As String
str = Belajar Visual Basic
msgbox str &  =  & Len(str) & karakter
\end {equation}
\end

\subsection {fungsi dari pemograman basix syntax}
fungsi dalam pemrograman C terdiri dari function header dan function body. Berikut adalah semua bagian dari sebuah fungsi: 

Return Type −  Fungsi dapat mengembalikan nilai. Return_type adalah tipe data dari nilai fungsi yang dikembalikan. Beberapa fungsi melakukan operasi yang diinginkan tanpa mengembalikan nilai.
Function Name − Ini adalah nama sebenarnya dari fungsi. Nama fungsi dan daftar parameter bersama merupakan function signature.
Parameter List − Parameter seperti placeholder. Saat sebuah fungsi dipanggil, Anda melewatkan sebuah nilai sebagai parameter. Nilai ini disebut sebagai parameter atau argumen Daftar parameter mengacu pada tipe, urutan, dan jumlah parameter fungsi. Parameter bersifat opsional; Artinya, fungsi tidak mengandung parameter.
Function Body − berisi kumpulan pernyataan yang mendefinisikan fungsi yang dilakukannya.

Memanggil Function
Untuk menggunakan fungsi, Anda harus memanggil fungsi itu untuk melakukan tugas yang ditentukan.

Sekarang, mari tulis contoh di atas dengan bantuan sebuah fungsi
\begin {enumaret}
#include <stdio.h>
 
int getMax( int set[] ) {
 
   int i, max;
    
   max = set[0];
   i = 1;    
   while( i < 5 ) {
   
      if( max <  set[i] ) {
     
         max = set[i];
      }
      i = i + 1;
   }
     
   return max;
}
 
main() {
 
   int set1[5] = {10, 20, 30, 40, 50};
   int set2[5] = {101, 201, 301, 401, 501};
   int max;
 
   /* Process first set of numbers available in set1[] */
   max = getMax(set1);
   printf("Max in first set = %dn", max );
     
   /* Now process second set of numbers available in set2[] */
   max = getMax(set2);
   printf("Max in second set = %dn", max );
}
\end {enumaret}
Outpunya adalah: 
\begin {enumaret}
Max in first set = 50
Max in second set = 501
\end {enumaret}

Function dalam bahasa Java
Jika Anda jelas tentang fungsi dalam pemrograman C, maka mudah untuk memahaminya di bahasa java, java menyebut function sebagai methods tapi sisa konsepnya kurang lebih sama. 
Berikut ini adalah program sama seperti di atas yang ditulis di bahasa java 
\begin {enumaret}
public class DemoJava {
     
   public static void main(String []args) {
  
      int[] set1 = {10, 20, 30, 40, 50};
      int[] set2 = {101, 201, 301, 401, 501};
      int max;
 
      /* Process first set of numbers available in set1[] */
      max = getMax(set1);
      System.out.format(\"Max in first set = %dn\", max );
 
      /* Now process second set of numbers available in set2[] */
      max = getMax(set2);
      System.out.format(\"Max in second set = %dn\", max );
   }
    
   public static int getMax( int set[] ) {
  
      int i, max;
    
      max = set[0];
      i = 1;    
      while( i < 5 ) {
 
         if( max <  set[i] ) {
            max = set[i];
         }
         i = i + 1;
      }
     
      return max;
   }
}
\end {enumaret}
Outputnya adalah: 
\begin {enumaret}
Max in first set = 50
Max in second set = 501
\end {enumaret}

Function dalan bahasa Python
Jika Anda tahu konsep fungsi dalam pemrograman C dan Java, maka Python tidak jauh berbeda. Berikut basic syntax untuk mengunakan fucntion di python 

\begin {enumaret}
def function_name( parameter list ):
   body of the function
   
   return [expression]
\end {enumaret}
Berikut ini adalah program sama seperti di atas yang ditulis di bahasa python 
\begin {enumaret}
def getMax( set ):
   max = set[0]
   i = 1  
   while( i < 5 ):
      if( max <  set[i] ):
         max = set[i]
      i = i + 1
   return max
 
 
set1 = [10, 20, 30, 40, 50]
set2 = [101, 201, 301, 401, 501]
\end {enumaret}
 
# Process first set of numbers available in set1[]
max = getMax(set1)
print \"Max in first set = \"\, max
     
# Now process second set of numbers available in set2[]
max = getMax(set2)
print \"Max in second set = \"\, max
Outputnya adalah: 

Max in first set =  50
Max in second set =  501
\end {enumaret}
