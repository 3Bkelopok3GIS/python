\sloppy
Python Variabel Type  \par
\vspace{12pt}
\noindent 
Variabel tidak lain hanyalah lokasi memori reserved untuk menyimpan nilai. $  $Ini berarti bahwa ketika Anda membuat variabel Anda memesan beberapa ruang di memori. \par
\noindent 
Berdasarkan tipe data sebuah variabel, penafsir mengalokasikan memori dan memutuskan apa yang dapat disimpan dalam memori yang dipesan. $  $Oleh karena itu, dengan menetapkan tipe data yang berbeda ke variabel, Anda dapat menyimpan bilangan bulat, desimal atau karakter dalam variabel ini. \par
\noindent 
Variabel adalah lokasi memori yang dicadangkan untuk menyimpan nilai-nilai. Ini berarti bahwa ketika Anda membuat sebuah variabel Anda memesan beberapa ruang di memori. Variabel menyimpan data yang dilakukan selama program dieksekusi, yang natinya isi dari variabel tersebut dapat diubah oleh operasi - operasi tertentu pada program yang menggunakan variabel.\vspace{\baselineskip}
\vspace{\baselineskip}
Variabel dapat menyimpan berbagai macam $  $tipe data. Di dalam pemrograman Python, variabel mempunyai sifat yang dinamis, artinya variabel Python tidak perlu didekralasikan tipe data tertentu dan variabel Python dapat diubah saat program dijalankan.\vspace{\baselineskip}
\vspace{\baselineskip}
Penulisan variabel Python sendiri juga memiliki aturan tertentu, yaitu :\vspace{\baselineskip}
1. Karakter pertama harus berupa huruf atau garis bawah/underscore $  $ $  \_  $\vspace{\baselineskip}
2. Karakter selanjutnya dapat berupa huruf, garis bawah/underscore $  $ $  \_  $ $  $atau angka\vspace{\baselineskip}
3. Karakter pada nama variabel bersifat sensitif (case-sensitif). Artinya huruf kecil dan huruf besar dibedakan. Sebagai contoh, variabel $  $namaDepan $  $dan $  $namadepan $  $adalah variabel yang berbeda.\vspace{\baselineskip}
\vspace{\baselineskip}
Untuk mulai membuat variabel di Python caranya sangat mudah, Anda cukup menuliskan variabel lalu mengisinya dengan suatu nilai dengan cara menambahkan tanda sama dengan $  $= $  $diikuti dengan nilai yang ingin dimasukan. \par
\noindent 
Menetapkan Nilai ke Variabel \par
\noindent 
Variabel Python tidak memerlukan deklarasi eksplisit untuk memesan ruang memori. $  $Deklarasi terjadi secara otomatis saat Anda menetapkan nilai ke variabel. $  $Tanda sama (=) digunakan untuk menetapkan nilai pada variabel. \par
\noindent 
Operand di sebelah kiri = operator adalah nama variabel dan operan di sebelah kanan = operator adalah nilai yang tersimpan dalam variabel. $  $Misalnya  \par
\vspace{12pt}
\noindent 
counter~=~100~~~~~~~    $  \#  $ An integer assignment \par
\noindent 
miles~~~=~1000.0~~~~    $  \#  $ A floating point \par
\noindent 
name~~~~=~"John"~~~~    $  \#  $ A string \par
\vspace{12pt}
\noindent 
print counter \par
\noindent 
print miles \par
\noindent 
print name \par
\noindent 
Di sini, 100, 1000.0 dan "John" adalah nilai yang diberikan untuk $  $melawan $  $, $  $mil $  $, dan $  $variabel $  $nama $  $masing-masing. $  $Ini menghasilkan hasil sebagai berikut - \par
\noindent 
100 \par
\noindent 
1000.0 \par
\noindent 
John \par
\noindent 
Beberapa Tugas \par
\noindent 
Python memungkinkan Anda untuk menetapkan nilai tunggal ke beberapa variabel secara bersamaan. $  $Misalnya - \par
\noindent 
a = b = c = 1 \par
\noindent 
Di sini, sebuah objek bilangan bulat dibuat dengan nilai 1, dan ketiga variabel ditugaskan ke lokasi memori yang sama. $  $Anda juga dapat menetapkan beberapa objek ke beberapa variabel. $  $Misalnya - \par
\noindent 
a,b,c = 1,2,"john" \par
\noindent 
Di sini, dua objek bilangan bulat dengan nilai 1 dan 2 masing-masing diberikan pada variabel a dan b masing-masing, dan satu objek string dengan nilai "john" diberikan ke variabel c. \par
\noindent 
Tipe data standar \par
\noindent 
Data yang tersimpan dalam memori bisa bermacam-macam. $  $Misalnya, usia seseorang disimpan sebagai nilai numerik dan alamatnya disimpan sebagai karakter alfanumerik. $  $Python memiliki berbagai jenis data standar yang digunakan untuk menentukan operasi yang mungkin dilakukan pada mereka dan metode penyimpanan untuk masing-masing metode. \par
\noindent 
Python memiliki lima tipe data standar - \par
\noindent 
Angka \par
\noindent 
Tali \par
\noindent 
Daftar \par
\noindent 
Tuple \par
\noindent 
Kamus \par
\noindent 
Nomor Python \par
\noindent 
Nomor tipe data menyimpan nilai numerik. $  $Nomor objek dibuat saat Anda memberikan nilai pada mereka. $  $Misalnya - \par
\noindent 
var1 = 1 \par
\noindent 
var2 = 10 \par
\noindent 
Anda juga dapat menghapus referensi ke objek nomor dengan menggunakan del statement. $  $Sintaks dari pernyataan del adalah - \par
\noindent 
del var1[,var2[,var3[....,varN]]]] \par
\noindent 
Anda dapat menghapus satu objek atau beberapa objek dengan menggunakan pernyataan del. $  $Misalnya - \par
\noindent 
del var \par
\noindent 
del var $  \_  $a, var $  \_  $b \par
\noindent 
Python mendukung empat jenis numerik yang berbeda - \par
\noindent 
int (bilangan bulat yang ditandatangani) \par
\noindent 
Panjang (bilangan bulat panjang, mereka juga bisa diwakili dalam oktal dan heksadesimal) \par
\noindent 
float (floating point real value) \par
\noindent 
kompleks (bilangan kompleks) \par
\noindent 
Python memungkinkan Anda untuk menggunakan huruf kecil l dengan panjang, tapi disarankan agar Anda hanya menggunakan huruf besar L untuk menghindari kebingungan dengan nomor 1. Python menampilkan bilangan bulat panjang dengan huruf besar L. \par
\noindent 
Sebuah bilangan kompleks terdiri dari sepasang bilangan floating-point yang diinisialisasi langsung yang dinotasikan dengan x + yj, di mana x dan y adalah bilangan real dan j adalah unit imajiner. \par
\noindent 
String Python \par
\noindent 
String dengan Python diidentifikasi sebagai kumpulan karakter bersebelahan yang ditunjukkan dalam tanda petik. $  $Python memungkinkan untuk kedua pasang tanda kutip tunggal atau ganda. $  $Subset string dapat diambil dengan menggunakan operator slice ([] dan [:]) dengan indeks mulai dari 0 pada awal string dan bekerja dengan cara mereka dari -1 di akhir. \par
\noindent 
Tanda plus (+) adalah operator concatenation string dan tanda bintang (*) adalah operator pengulangan. $  $Misalnya  \par
\vspace{12pt}
\vspace{12pt}
\noindent 
str = 'Hello World!' \par
\vspace{12pt}
\noindent 
print str~~~~~~~~~  $  \#  $ Prints complete string \par
\noindent 
print str[0]~~~~~~  $  \#  $ Prints first character of the string \par
\noindent 
print str[2:5]~~~~  $  \#  $ Prints characters starting from 3rd to 5th \par
\noindent 
print str[2:]~~~~~  $  \#  $ Prints string starting from 3rd character \par
\noindent 
print str~*~2~~~    $  \#  $ Prints string two times \par
\noindent 
print str + "TEST"  $  \#  $ Prints concatenated string \par
\noindent 
Ini akan menghasilkan hasil sebagai berikut - \par
\noindent 
Hello World! \par
\noindent 
H \par
\noindent 
llo \par
\noindent 
llo World! \par
\noindent 
Hello World!Hello World! \par
\noindent 
Hello World!TEST \par
\noindent 
Daftar Python \par
\noindent 
Daftar adalah jenis data majemuk Python yang paling serbaguna. $  $Daftar berisi item yang dipisahkan dengan tanda koma dan dilampirkan dalam tanda kurung siku ([]). $  $Sampai batas tertentu, daftar serupa dengan array di C. Salah satu perbedaan di antara keduanya adalah bahwa semua item yang termasuk dalam daftar dapat terdiri dari tipe data yang berbeda. \par
\noindent 
Nilai yang tersimpan dalam daftar dapat diakses menggunakan operator slice ([] dan [:]) dengan indeks mulai dari 0 di awal daftar dan bekerja dengan cara mereka untuk mengakhiri -1. $  $Tanda plus (+) adalah daftar operator concatenation, dan asterisk (*) adalah operator pengulangan. $  $Misalnya - \par
\noindent 
 $  \#  $!/usr/bin/python \par
\vspace{12pt}
\noindent 
list = [ 'abcd', 786 , 2.23, 'john', 70.2 ] \par
\noindent 
tinylist = [123, 'john'] \par
\vspace{12pt}
\noindent 
print~list~~~~~~~~   $  \#  $ Prints complete list \par
\noindent 
print list[0]~~~~~~  $  \#  $ Prints first element of the list \par
\noindent 
print list[1:3]~~~~  $  \#  $ Prints elements starting from 2nd till 3rd  \par
\noindent 
print list[2:]~~~~~  $  \#  $ Prints elements starting from 3rd element \par
\noindent 
print tinylist * 2~  $  \#  $ Prints list two times \par
\noindent 
print list + tinylist  $  \#  $ Prints concatenated lists \par
\noindent 
Ini menghasilkan hasil sebagai berikut - \par
\noindent 
['abcd', 786, 2.23, 'john', 70.200000000000003] \par
\noindent 
abcd \par
\noindent 
[786, 2.23] \par
\noindent 
[2.23, 'john', 70.200000000000003] \par
\noindent 
[123, 'john', 123, 'john'] \par
\noindent 
['abcd', 786, 2.23, 'john', 70.200000000000003, 123, 'john'] \par
\noindent 
Tupel Python \par
\noindent 
Sebuah tupel adalah jenis data urutan lain yang serupa dengan daftar. $  $Sebuah tupel terdiri dari sejumlah nilai yang dipisahkan dengan koma. $  $Tidak seperti daftar, bagaimanapun, tupel tertutup dalam tanda kurung. \par
\noindent 
Perbedaan utama antara daftar dan tupel adalah: Daftar tertutup dalam tanda kurung ([]) dan elemen dan ukurannya dapat diubah, sementara tupel dilampirkan dalam tanda kurung (()) dan tidak dapat diperbarui. $  $Tupel bisa dianggap sebagai $  $daftar $  $hanya-baca $  $. $  $Misalnya - \par
\noindent 
 $  \#  $!/usr/bin/python \par
\vspace{12pt}
\noindent 
tuple = ( 'abcd',~786 , 2.23, 'john', 70.2  ) \par
\noindent 
tinytuple = (123, 'john') \par
\vspace{12pt}
\noindent 
print~tuple~~~~~~~~~   $  \#  $ Prints complete list \par
\noindent 
print tuple[0]~~~~~~~  $  \#  $ Prints first element of the list \par
\noindent 
print tuple[1:3]~~~~~  $  \#  $ Prints elements starting from 2nd till 3rd  \par
\noindent 
print tuple[2:]~~~~~~  $  \#  $ Prints elements starting from 3rd element \par
\noindent 
print tinytuple~*~2    $  \#  $ Prints list two times \par
\noindent 
print tuple + tinytuple  $  \#  $ Prints concatenated lists \par
\noindent 
Ini menghasilkan hasil sebagai berikut - \par
\noindent 
('abcd', 786, 2.23, 'john', 70.200000000000003) \par
\noindent 
abcd \par
\noindent 
(786, 2.23) \par
\noindent 
(2.23, 'john', 70.200000000000003) \par
\noindent 
(123, 'john', 123, 'john') \par
\noindent 
('abcd', 786, 2.23, 'john', 70.200000000000003, 123, 'john') \par
\noindent 
Kode berikut tidak valid dengan tupel, karena kami mencoba memperbarui tupel, yang tidak diizinkan. $  $Kasus serupa dimungkinkan dengan daftar - \par
\vspace{12pt}
\noindent 
tuple = ( 'abcd',~786 , 2.23, 'john', 70.2  ) \par
\noindent 
list = [ 'abcd', 786 ,~2.23, 'john', 70.2  ] \par
\noindent 
tuple[2]~=~1000~    $  \#  $ Invalid syntax with tuple \par
\noindent 
list[2]~=~1000~~    $  \#  $ Valid syntax with list \par
\noindent 
Kamus Python \par
\noindent 
Kamus Python adalah jenis tipe tabel hash. $  $Mereka bekerja seperti array asosiatif atau hash yang ditemukan di Perl dan terdiri dari pasangan kunci-nilai. $  $Kunci kamus bisa hampir sama dengan tipe Python, tapi biasanya angka atau string. $  $Nilai, di sisi lain, bisa menjadi objek Python yang sewenang-wenang. \par
\noindent 
Kamus ditutupi oleh kurung kurawal ( $  \{  $ $  \}  $) dan nilai dapat diberikan dan diakses menggunakan kawat gigi persegi ([]). $  $Misalnya - \par
\noindent 
 $  \#  $!/usr/bin/python \par
\vspace{12pt}
\noindent 
dict =  $  \{  $ $  \}  $ \par
\noindent 
dict['one'] = "This is one" \par
\noindent 
dict[2]~~~~ = "This is two" \par
\vspace{12pt}
\noindent 
tinydict =  $  \{  $'name': 'john','code':6734, 'dept': 'sales' $  \}  $ \par
\vspace{12pt}
\vspace{12pt}
\noindent 
print dict['one']~~ ~~~  $  \#  $ Prints value for 'one' key \par
\noindent 
print dict[2]~~~~~~~~~~  $  \#  $ Prints value for 2 key \par
\noindent 
print tinydict~~~~~~~~~  $  \#  $ Prints complete dictionary \par
\noindent 
print tinydict.keys()~~  $  \#  $ Prints all the keys \par
\noindent 
print tinydict.values()  $  \#  $ Prints all the values \par
\noindent 
Ini menghasilkan hasil sebagai berikut - \par
\noindent 
This is one \par
\noindent 
This is two \par
\noindent 
 $  \{  $'dept': 'sales', 'code': 6734, 'name': 'john' $  \}  $ \par
\noindent 
['dept', 'code', 'name'] \par
\noindent 
['sales', 6734, 'john'] \par
\noindent 
Kamus tidak memiliki konsep keteraturan antar elemen. $  $Tidak benar mengatakan bahwa unsur-unsurnya "rusak"; $  $Mereka hanya unordered. \par
\noindent 
Konversi Tipe Data \par
\noindent 
Terkadang, Anda mungkin perlu melakukan konversi antara jenis built-in. $  $Untuk mengonversi antar jenis, Anda cukup menggunakan nama jenis sebagai fungsi. \par
\noindent 
Ada beberapa fungsi built-in untuk melakukan konversi dari satu tipe data ke tipe data yang lain. $  $Fungsi ini mengembalikan objek baru yang mewakili nilai yang dikonversi. \par
\noindent 
Pembagian nilai $  $a $  $dan $  $b $  $menghasilkan $  $3 $  $(integer). Mengapa demikian? \par
\noindent 
Karena nilai $  $a $  $dan $  $b $  $bertipe integer, maka hasilnya pun berupa integer. \par
\noindent 
Bagaimana agar hasilnya ada komanya? \par
\noindent 
Tentu kita harus merubah tipe variabel $  $a $  $dan $  $b $  $menjadi bilangan pecahan (float) dulu, baru setelah itu dibagi. \par
\noindent 
a = 10 \par
\noindent 
b = 3 \par
\noindent 
c = float(a) / float(b)  $  \#  $output: 3.3333333333333335 \par
\vspace{12pt}
\noindent 
print c \par
\noindent 
Fungsi $  $float() $  $akan mengubah nilai $  $a $  $menjadi $  $10.0 $  $dan $  $b $  $menjadi $  $3.0. \par
\noindent 
Fungsi-fungsi untuk mengubah tipe data: \par
\noindent 
int() $  $untuk mengubah menjadi integer; \par
\noindent 
long() $  $untuk mengubah menjadi integer panjang; \par
\noindent 
float() $  $untuk mengubah menjadi float; \par
\noindent 
bool() $  $untuk mengubah menjadi boolean; \par
\noindent 
chr() $  $untuk mengubah menjadi karakter; \par
\noindent 
str() $  $untuk mengubah menjadi string. \par
\noindent 
bin() $  $untuk mengubah menjadi bilangan Biner. \par
\noindent 
hex() $  $untuk mengubah menjadi bilangan Heksadesimal. \par
\noindent 
oct() $  $untuk mengubah menjadi bilangan okta. \par
\vspace{12pt}
