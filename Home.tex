
\sloppy
{\fontsize{14pt}{14pt}\selectfont HOME \\} \par
\noindent 
{\fontsize{14pt}{14pt}\selectfont Pemrograman python adalah bahasa pemrograman terpopuler di tahun 2016 menurut tiobe. Python juga memiliki sintak atau aturan penulisan code pemrograman. Salah satu bagian Home merupakan halaman pengantar untuk mempelajari python . sebelum ketahapan yang baru selain home ini pembaca memerlukan pengertian yang lain yaitu seperti enverinmoment setup, syntax dan lain lain, awal untuk penulis jelakan yaitu pengertian tentang class pada python untuk mengantarkan logika dan pengetahuan apaitu class. \\} \par
\vspace{14pt}
\noindent 
{\fontsize{14pt}{14pt}\selectfont Nah class itu merupakan class yang didalam nya mempunyai metode yang sesuai dengan fungsinya, contoh di kehidupan nyata seperti kelas belajar sebagai class nya da isi dari kelas itu seperti bangku, sepidol, dan lain lain itu merupakan metode dari class dan metode itu berfungsi seperti fungsi itu sendiri seperti spidol untuk menulis itu contoh nya di sesuaikan dengan fungsi. \\} \par
\vspace{14pt}
\noindent 
{\fontsize{14pt}{14pt}\selectfont Pembuatan class pada python  \\} \par
\noindent 
{\fontsize{14pt}{14pt}\selectfont Kita awali dengan sebuah kata kunci. Yaitu  $ " $class $ " $ yang kemudian di ikuti dengan  $ " $nama class nya $ " $ terkahir membuat kurung buka dan tutup serta membuat tanda titik dua  $ " $() $ " $ dan  $ " $: $ " $ kalo synax nya seperti ini. \\} \par
\vspace{14pt}
\noindent 
{\fontsize{14pt}{14pt}\selectfont namaClass() \\} \par
\vspace{14pt}
\noindent 
{\fontsize{14pt}{14pt}\selectfont untuk memanggil metode kita cukup menggunakan memanggil class yang kemudian di ikuti dengan pemanggilan nama metode yang tersedia di dalam class tersebut dengan di pisahkan oleh tanda titik seperti \\} \par
\vspace{14pt}
\noindent 
{\fontsize{14pt}{14pt}\selectfont namClass().namaMetode() \\} \par
\vspace{14pt}
\noindent 
{\fontsize{14pt}{14pt}\selectfont ingin lebih mudah kita tampung classnya dulu ke variable  \\} \par
\vspace{14pt}
\noindent 
{\fontsize{14pt}{14pt}\selectfont penampung = namaClass() \\} \par
\noindent 
{\fontsize{14pt}{14pt}\selectfont penampung.namaMetode() \\} \par
\vspace{14pt}
\noindent 
{\fontsize{14pt}{14pt}\selectfont di dalam sebuah class yang dibuat biasanya terdapat init itu disediakan langsung oleh python nya jadi seperti ini \\} \par
\vspace{14pt}
\noindent 
{\fontsize{14pt}{14pt}\selectfont class namaClass(): \\} \par
\noindent 
{\fontsize{14pt}{14pt}\selectfont def $  \_  $init $  \_  $(self,parameter): \\} \par
\noindent 
{\fontsize{14pt}{14pt}\selectfont itu code program yang pertama kali kalian buat \\} \par
\noindent 
{\fontsize{14pt}{14pt}\selectfont def metode 1 (self,parameter): \\} \par
\noindent 
{\fontsize{14pt}{14pt}\selectfont isi metode \\} \par
\noindent 
{\fontsize{14pt}{14pt}\selectfont def metode 2 (self): \\} \par
\noindent 
{\fontsize{14pt}{14pt}\selectfont isi metode \\} \par
\noindent 
{\fontsize{14pt}{14pt}\selectfont nah seperti itu lah kurang lebih. \\} \par
\vspace{14pt}
\noindent 
{\fontsize{14pt}{14pt}\selectfont Setelah menjelaskan class kita akan menjelaskan variable seperti tadi penampung class, nah variable bisa di artikan sebagai huruf atau kata tujuan nya untuk mempermudah proses penulisan sebuah program. \\} \par
\noindent 
{\fontsize{14pt}{14pt}\selectfont Contoh dalam kehidupan nyata seperti gelas atau ember, ambil contoh ember kita ketahui bahwa ember bisa di isi dengan air, pasir, tanah dan lain lain, nah ember itu sebagai variable da nisi variable nya itu adalah air, tanah, pasir dan lain lain. \\} \par
\vspace{14pt}
\noindent 
{\fontsize{14pt}{14pt}\selectfont Contoh variable seperti ini \\} \par
\noindent 
{\fontsize{14pt}{14pt}\selectfont Variable=  $ " $ini string atau teks $ " $ \\} \par
\noindent 
{\fontsize{14pt}{14pt}\selectfont Variable2=12 \\} \par
\vspace{14pt}
\noindent 
{\fontsize{14pt}{14pt}\selectfont Print( $ " $nilai isi dari variable1 adalah: $ " $,variable1) \\} \par
\noindent 
{\fontsize{14pt}{14pt}\selectfont Print( $ " $nilai atau isi dari variable2 adalah: $ " $,variable2) \\} \par
\noindent 
{\fontsize{14pt}{14pt}\selectfont Nah seperti itu contohnya \\} \par
\vspace{14pt}
\noindent 
{\fontsize{14pt}{14pt}\selectfont Ada beberapa hal yang harus di ketahui seperti input, data operation danlain lain seperti ini \\} \par
\vspace{14pt}
\noindent 
{\fontsize{14pt}{14pt}\selectfont Input \\} \par
\noindent 
{\fontsize{14pt}{14pt}\selectfont Tahapan ini merupakan proses memasukan data ke dalam proses komputer melalui peralatan $  $input. $  $ Pada bahasa Python, untuk menerima masukan dari pengguna yaitu dengan menggunakan $  $method input() $  $dan $  $raw $  \_  $input(). \\} \par
\noindent 
{\fontsize{14pt}{14pt}\selectfont Data \\} \par
\noindent 
{\fontsize{14pt}{14pt}\selectfont Data adalah bahan mentah yang akan diolah menjadi informasi sehingga dapat berguna dan dimanfaatkan oleh pengguna. Data dapat berupa variabel, konstanta, atau yang berisi bilangan, kalimat, dan lainnya. Tipe data berupa string, number, list, tuple, dan lainnya. \\} \par
\vspace{14pt}
\noindent 
{\fontsize{14pt}{14pt}\selectfont Operation \\} \par
\noindent 
{\fontsize{14pt}{14pt}\selectfont Operation $  $adalah yang akan mengubah suatu nilai menjadi nilai lain. Yang termasuk $  $operationatau yang biasa disebut dengan operator adalah operator aritmatika, operator assignment, dan lainnya. \\} \par
\vspace{14pt}
\noindent 
{\fontsize{14pt}{14pt}\selectfont Output \\} \par
\noindent 
{\fontsize{14pt}{14pt}\selectfont Output $  $adalah menuliskan informasi yang ditampilkan dilayar, $  $disk, atau ke salah satu unit I/O. Pada Python 2.0, untuk menampilkan $  $output $  $dengan menulis sintax print. Sedangkan pada Python 3.0 dengan menggunakan fungsi print(). \\} \par
\vspace{14pt}
\noindent 
{\fontsize{14pt}{14pt}\selectfont Conditional \\} \par
\noindent 
{\fontsize{14pt}{14pt}\selectfont Merupakan jumlah perintah yang akan dijalankan jika kondisi tertentu sudah terpenuhi. Jika $  $username $  $dan $  $password $  $yang dimasukan benar, maka akan menampilkan halaman utama. Hal ini bisa disebut $  $conditional. $  $Pada $  $conditional, $  $Python menggunakan pernyataan if, else, dan elif. \\} \par
\vspace{14pt}
\noindent 
{\fontsize{14pt}{14pt}\selectfont Looping \\} \par
\noindent 
{\fontsize{14pt}{14pt}\selectfont Perintah yang akan berjalan beberapa kali, selama kondisi yang ditentukan atau kondisi yang terpenuhi. Pada $  $looping $  $ini, Python menggunakan pernyataan for dan while untuk melakukan perulangan. \\} \par
\vspace{14pt}
\noindent 
{\fontsize{14pt}{14pt}\selectfont Subroutine \\} \par
\noindent 
{\fontsize{14pt}{14pt}\selectfont Perintah yang bisa dijalankan dengan cara memanggil namanya. Sering disebut sebagai $  $functionatau $  $method. Pada bahasa pemrograman Python, untuk menggunakan $  $function $  $atau $  $method $  $yaitu dengan menggunakan pernyatan $  $def nama $  \_  $function(). \\} \par
\vspace{14pt}
\noindent 
{\fontsize{14pt}{14pt}\selectfont Fungsi def dalam python \\} \par
\noindent 
{\fontsize{14pt}{14pt}\selectfont Penggunaan fungsi tanpa parameter \\} \par
\noindent 
{\fontsize{14pt}{14pt}\selectfont Command=fungsi() \\} \par
\noindent 
{\fontsize{14pt}{14pt}\selectfont Deklarasi command= def fungsi() \\} \par
\vspace{14pt}
\noindent 
{\fontsize{14pt}{14pt}\selectfont Pemanggilan fungsi, parameter sesuai dengan kata kunci seperti tadi class \\} \par
\noindent 
{\fontsize{14pt}{14pt}\selectfont Command= fungsi(arg=1, arg2=2) \\} \par
\noindent 
{\fontsize{14pt}{14pt}\selectfont Deklarasi command – def fungsi (arg2,arg2) \\} \par
\vspace{14pt}
\noindent 
{\fontsize{14pt}{14pt}\selectfont Pemanggilan fungsi, parameter sesuai dengan posisi \\} \par
\noindent 
{\fontsize{14pt}{14pt}\selectfont Command= fungsi() \\} \par
\noindent 
{\fontsize{14pt}{14pt}\selectfont Deklarasi command – def fungsi (x) \\} \par
\vspace{14pt}
\noindent 
{\fontsize{14pt}{14pt}\selectfont Pemanggilan fungsi parameter sesuai dengan argument posisional tuple \\} \par
\noindent 
{\fontsize{14pt}{14pt}\selectfont Command= fungsi((1,2).(1,3)) \\} \par
\noindent 
{\fontsize{14pt}{14pt}\selectfont Deklarasi command – def fungsi (*args) \\} \par
\vspace{14pt}
\noindent 
{\fontsize{14pt}{14pt}\selectfont Pemanggilan fungsi, parameters ssesuia argument kata kunci dictionary \\} \par
\noindent 
{\fontsize{14pt}{14pt}\selectfont Command= fungsi (bahasa= ‘python’,versi=’2.2’) \\} \par
\noindent 
{\fontsize{14pt}{14pt}\selectfont Deklarasi command = def fungsi (**args) \\} \par
\vspace{14pt}
\noindent 
{\fontsize{14pt}{14pt}\selectfont Itu adalah cara memanggil dalam code python pemrograman jadi ada nenerapa fungsi yang dibutuhkan penulisan dengan tepat maka sebab itu dengan sebab itu buaat lah penulisan yang mudah. Karena pemrograma python sangan sensitive bila ada kesalahan sedikit di penulisan atau symbol yang tertinggal. \\} \par
\vspace{14pt}
\vspace{14pt}
\noindent 
{\fontsize{14pt}{14pt}\selectfont Sejarah \\} \par
\noindent 
{\fontsize{14pt}{14pt}\selectfont Bahasa pemrograman Python adalah bahasa yang dibuat oleh seorang keturunan Belanda yaitu Guido van Rossum. Awalnya, pembuatan bahasa pemrograman ini adalah untuk membuat skrip bahasa tingkat tinggi pada sebuah sistem operasi yang terdistribusi Amoeba. Python telah digunakan oleh beberapa pengembang dan bahkan digunakan oleh beberapa perusahaan untuk pembuatan perangkat lunak komersial. \\} \par
\noindent 
{\fontsize{14pt}{14pt}\selectfont Pemrograman bahasa python ini adalah pemrogram gratis atau freeware, sehingga dapat dikembangkan, dan tidak ada batasan dalam penyalinannya dan mendistribusikan. \\} \par
\vspace{14pt}
\noindent 
{\fontsize{14pt}{14pt}\selectfont Dukungan Komunitas yang Aktif \\} \par
\noindent 
{\fontsize{14pt}{14pt}\selectfont Python adalah salah satu pemrograman yang terus berkembang dan bertahan dikarenakan dukungan komunitas yang aktif diseluruh dunia. Banyak forum-forum ataupun blogger-blogger yang sering membagi pengalaman dalam menggunakan python. Hal ini memudahkan bagi pengguna pemula maupun pengembang untuk bertanya dan sharing tentang ilmu pemrograman ini. \\} \par
\noindent 
{\fontsize{14pt}{14pt}\selectfont Kelebihan dan Kekurangan \\} \par
\vspace{14pt}
\noindent 
{\fontsize{14pt}{14pt}\selectfont Kelebihan : \\} \par
\noindent 
{\fontsize{14pt}{14pt}\selectfont 1. Tidak ada tahapan kompilasi dan penyambungan (link) sehingga kecepatan perubahan pada masa pembuatan sistem aplikasi meningkat. \\} \par
\vspace{14pt}
\noindent 
{\fontsize{14pt}{14pt}\selectfont 2. Tidak ada deklarasi tipe data yang merumitkan sehingga program menjadi lebih sederhana, singkat, dan fleksible. \\} \par
\vspace{14pt}
\noindent 
{\fontsize{14pt}{14pt}\selectfont 3. Manajemen memori otomatis yaitu kumpulan sampah memori sehingga dapat menghindari pencacatan kode. \\} \par
\noindent 
{\fontsize{14pt}{14pt}\selectfont 4. Tipe data dan operasi tingkat tinggi yaitu kecepatan pembuatan sistem aplikasi menggunakan tipe objek yang telah ada. \\} \par
\noindent 
{\fontsize{14pt}{14pt}\selectfont 5. Pemrograman berorientasi objek. \\} \par
\noindent 
{\fontsize{14pt}{14pt}\selectfont 6. Pelekatan dan perluasan dalam C. \\} \par
\noindent 
{\fontsize{14pt}{14pt}\selectfont 7. Terdapat kelas, modul, eksepsi sehingga terdapat dukungan pemrograman skala besar secara modular. \\} \par
\noindent 
{\fontsize{14pt}{14pt}\selectfont 8. Pemuatan dinamis modul C sehingga ekstensi menjadi sederhana dan berkas biner yang kecil \\} \par
\noindent 
{\fontsize{14pt}{14pt}\selectfont 9. Pemuatan kembali secara dinamis modul phyton seperti memodifikasi aplikasi tanpa menghentikannya. \\} \par
\noindent 
{\fontsize{14pt}{14pt}\selectfont 10. Model objek universal kelas Satu. \\} \par
\noindent 
{\fontsize{14pt}{14pt}\selectfont 11. Konstruksi pada saat aplikasi berjalan. \\} \par
\noindent 
{\fontsize{14pt}{14pt}\selectfont 12. Interaktif, dinamis dan alamiah. \\} \par
\noindent 
{\fontsize{14pt}{14pt}\selectfont 13. Akses hingga informasi interpreter. \\} \par
\noindent 
{\fontsize{14pt}{14pt}\selectfont 14. Portabilitas secara luas seperti pemrograman antar platform tanpa ports. \\} \par
\noindent 
{\fontsize{14pt}{14pt}\selectfont 15. Kompilasi untuk portable kode byte sehingga kecepatan eksekusi bertambah dan melindungi kode sumber. \\} \par
\noindent 
{\fontsize{14pt}{14pt}\selectfont 16. Antarmuka terpasang untuk pelayanan keluar seperti perangkat Bantu system, GUI, persistence, database, dll. \\} \par
\vspace{14pt}
\noindent 
{\fontsize{14pt}{14pt}\selectfont Kekurangan : \\} \par
\vspace{14pt}
\noindent 
{\fontsize{14pt}{14pt}\selectfont 1. Beberapa penugasan terdapat diluar dari jangkauan python, seperti bahasa pemrograman dinamis lainnya, python tidak secepat atau efisien sebagai statis, tidak seperti bahasa pemrograman kompilasi seperti bahasa C. \\} \par
\noindent 
{\fontsize{14pt}{14pt}\selectfont 2. Disebabkan python merupakan interpreter, python bukan merupakan perangkat bantu terbaik untuk pengantar komponen performa kritis. \\} \par
\noindent 
{\fontsize{14pt}{14pt}\selectfont 3. Python tidak dapat digunakan sebagai dasar bahasa pemrograman implementasi untuk beberapa komponen, tetapi dapat bekerja dengan baik sebagai bagian depan skrip antarmuka untuk mereka. \\} \par
\noindent 
{\fontsize{14pt}{14pt}\selectfont 4. Python memberikan efisiensi dan fleksibilitas tradeoff by dengan tidak memberikannya secara menyeluruh. Python menyediakan bahasa pemrograman optimasi untuk kegunaan, bersama dengan perangkat bantu yang dibutuhkan untuk diintegrasikan dengan bahasa pemrograman lainnya. \\} \par
\noindent 
{\fontsize{14pt}{14pt}\selectfont 5. Banyak terdapat referensi lama terutama dari pencarian google, python adalah pemrograman yang sangat lambat. Namun belum lama ini ditemukan bahwa Google, Youtube, DropBox dan beberapa software sistem banyak menggunakan Python. \\} \par
\noindent 
{\fontsize{14pt}{14pt}\selectfont 6. Kini Python menjadi salah satu bahasa pemrograman yang populer digunakan oleh pengembangan $  $web, aplikasi $  $web, aplikasi perkantoran, simulasi, dan masih banyak lagi. $  $ Hal ini disebabkan karena Python bahasa pemrograman yang dinamis dan mudah dipahami. \\} \par
\noindent 
{\fontsize{14pt}{14pt}\selectfont 7. Selain itu, sekarang telah tersedia berbagai situs kursus yang bagus untuk mempelajari bahasa pemrograman Python ini sehingga pembaca maupun developer pemula yang akan mempelajari bahasa ini akan menjadi lebih mudah karena dapat berlatih dimanapun dan kapanpun selama terhubung dengan Internet. \\} \par
\noindent 
{\fontsize{14pt}{14pt}\selectfont 8. Menariknya, berbagai situs kursus gratis ini menawarkan metode pembelajaran yang interaktif sehingga mudah dimengerti oleh pesertanya. \\} \par
\noindent 
{\fontsize{14pt}{14pt}\selectfont Python merupakan pemograman yang tidak pernah di compile secara full. Jika kamu sudah menyelesaikan programnya dan kamu ingin mengirim ke teman atau di bagikan ke internet maka teman atau orang lain dapat mengubah kode di program kamu karena program di buka di notepad, python akan tetap berbentuk kode yang sama tidak acak acakan sehingga orang lain dapat memahami pemograman yang kamu buat. \\} \par
\vspace{14pt}
\noindent 
{\fontsize{14pt}{14pt}\selectfont Python \\} \par
\vspace{14pt}
\noindent 
{\fontsize{14pt}{14pt}\selectfont Python 2.4.3 ( $  \#  $1, Nov 11 2010, 13:34:43) \\} \par
\noindent 
{\fontsize{14pt}{14pt}\selectfont [GCC 4.1.2 20080704 (Red Hat 4.1.2-48)] on linux2 \\} \par
\noindent 
{\fontsize{14pt}{14pt}\selectfont Type "help", "copyright", "credits" or "license" for more information. \\} \par
\vspace{14pt}
\noindent 
{\fontsize{14pt}{14pt}\selectfont Ketik teks berikut pada prompt Python dan tekan Enter: \\} \par
\vspace{14pt}
\noindent 
{\fontsize{14pt}{14pt}\selectfont print "Hello, Python!" \\} \par
\vspace{14pt}
\noindent 
{\fontsize{14pt}{14pt}\selectfont Jika Anda menjalankan versi baru Python, Anda perlu menggunakan pernyataan cetak dengan tanda kurung seperti pada cetak ("Halo, Python!") ;. Namun dengan versi Python 2.4.3, ini menghasilkan hasil sebagai berikut: \\} \par
\vspace{14pt}
\noindent 
{\fontsize{14pt}{14pt}\selectfont Hello, Pyhton! \\} \par
\vspace{14pt}
\noindent 
{\fontsize{14pt}{14pt}\selectfont Pemrograman Mode Script \\} \par
\noindent 
{\fontsize{14pt}{14pt}\selectfont Memohon interpreter dengan parameter script memulai eksekusi script dan berlanjut sampai script selesai. Saat skrip selesai, juru bahasa tidak lagi aktif. \\} \par
\vspace{14pt}
\noindent 
{\fontsize{14pt}{14pt}\selectfont Mari kita tuliskan program Python sederhana dalam sebuah naskah. File Python memiliki ekstensi .py. Ketik kode sumber berikut di file \\} \par
\vspace{14pt}
\noindent 
{\fontsize{14pt}{14pt}\selectfont Objek Dengan Python, seperti semua bahasa berorientasi objek, ada kumpulan kode dan data yang disebut objek, yang biasanya mewakili potongan dalam model konseptual suatu sistem. \\} \par
\vspace{14pt}
\noindent 
{\fontsize{14pt}{14pt}\selectfont Objek dengan Python dibuat (yaitu, instantiated) dari template yang disebut kelas (yang akan dibahas kemudian, sebanyak bahasa dapat digunakan tanpa memahami kelas). Mereka memiliki atribut, yang mewakili berbagai potongan kode dan data yang membentuk objek. Untuk mengakses atribut, seseorang menuliskan nama objek yang diikuti oleh suatu periode (selanjutnya disebut titik), diikuti dengan nama atribut. \\} \par
\vspace{14pt}
\noindent 
{\fontsize{14pt}{14pt}\selectfont Contohnya adalah atribut 'atas' dari string, yang mengacu pada kode yang mengembalikan salinan string di mana semua huruf adalah huruf besar. Untuk mendapatkan ini, perlu untuk memiliki cara untuk merujuk ke objek (dalam contoh berikut, jalan adalah string literal yang membangun objek). \\} \par
\vspace{14pt}
\vspace{14pt}
\noindent 
{\fontsize{14pt}{14pt}\selectfont Paradigma : Multi-paradigm: object-oriented, imperative, functional, procedural, reflective\vspace{\baselineskip}
Muncul Tahun : 1991\vspace{\baselineskip}
Perancang : Guido van Rossum\vspace{\baselineskip}
Pengembang : Python Software Foundation\vspace{\baselineskip}
Rilis terbaru : 3.2.3 / 11 April 2012; 46 hari lalu 2.7.3 / 11 April 2012; 46 hari lalu /\vspace{\baselineskip}
Sistem pengetikan: duck, dynamic, strong\vspace{\baselineskip}
Implementasi : CPython, IronPython, Jython, Python for S60, PyPy\vspace{\baselineskip}
Dialek : Cython, RPython, Stackless Python\vspace{\baselineskip}
Terpengaruh oleh : ABC,ALGOL 68,C,C++,Dylan Haskell,Icon,Java,Lisp,Modula-3,Perl\vspace{\baselineskip}
Mempengaruhi : Boo, Cobra, D, Falcon, Groovy, JavaScript, Ruby\vspace{\baselineskip}
Sistem operasi : Cross-platform\vspace{\baselineskip}
Lisensi : Python Software Foundation License,GNU GPL\vspace{\baselineskip}
Situs web : python.org \\} \par
\vspace{14pt}
\noindent 
{\fontsize{14pt}{14pt}\selectfont Kesimpulan  \\} \par
\noindent 
{\fontsize{14pt}{14pt}\selectfont Kenapa~ python banyak digunakan sehingga terpopuler di tahun 2016, karena python salah satu pemrograman opensource sehingga dapat dipahami apabila ingin di pahami, apa yang menjadi python melambung tinggi sampai saat ini, diantaranya mempunyai kepustkaan yang luas untuk memduahkan programmer selain itu ada module yang di sediakan oleh python nya langsung. Apabila pembaca tau tentang instagram atau django itu adalah situs asuhan dari mark zuckerburg instagram dan django adalah web yg menyediakan berbagai hal kalian ketahui django adalah server kapastitas besar dan instagram kalian ketahu sendiri kan itu dibuat dengan pemrograman python menurut disca7x.blogspot.co.id \\} \par
\vspace{14pt}
\vspace{14pt}
\vspace{14pt}
\vspace{12pt}
