
\section{Pengertian Jaringan}
  Jaringan yaitu sekumpulan komputer yang dihubungkan dengan kabel sehingga komputer yang satu dengan komputer yang lainnya dapat saling komunikasi, bertukar informasi sharing file, printer, dan sebagainya.
  Networking merupakan salah satu cabang ilmu dunia Teknik Informatika yang membahas tentang komunikasi antar komputer. Materi networking yang di berikan di sekolah atau di perkuliahan saat ini sepertinya belum cukup memadai dari yang diharapkan. Bagi mereka yang sangat ingin mendalami tentang ilmu networking bisa mempelajarinya dari artikel-artikel di internet, dan biasanya ketika kita menemukan artikel tentang materi networking yang ingin dipelajari sering sekali ditemukan kata-kata atau istilah-istilah yang belum dimengerti, biasanya kita akan mencari kata-kata tersebut dengan mengetikkan keywordnya di mesin pencari Google. lalu kita akan belajar memahami kata tersebut, setelah kita mengerti kita akan kembali mempelajari materi yang tadi. cara ini tentu tidak efektif. maka dari sebaiknya sebelum kita mempelajari mengenai networking kita pelajari dulu dari yang paling dasar, yaitu istilah-istilah dalam networking.
  Networking sangat dibutuhkan ,terutama pada zaman yang semakin lama semakin canggih seperti ini ,karena jaringan itu tentu sangat penting untuk berlangsungnya hubungan atau komunikasi antar komputer. misalnya saja untuk berbagi atau sharing printer , tidak mungkin setiap komputer memiliki printer satu-satu makannya dibuatlah jaringan komputer itu untuk berbagi penggunaan printer secara bersama-sama dan juga berfungsi untuk sharing internet ,satu komputer (server) dapat ip address dari isp ,lalu si server itu membagikan koneksi internet ke client-client dikantornya. Jaringan dibagi menjadi 2 yaitu:
  \begin{enumerate}
    \item Standalone
    \item Network
  \end{enumerate}

\section{Jenis-Jenis Jaringan berdasarkan jangkuan}
  \subsection{Local Area Networking (LAN)}
    Yaitu Jaringan yang dibatasi oleh area yang relative kecil, umumnya dibatasi oleh area lingkungan seperti sebuah perkantoran di sebuah gedung, atau sebuah sekolah, dan biasanya tidak jauh dari sekitar 1 km persegi.
  \subsection{Metropolitan Area Networking (MAN)}
    Yaitu Jaringan yang lebih luas dari LAN, MAN biasanya meliputi area yang lebih besar seperti area propinsi, antar gedung. Mengapa MAN itu dikatakan lebih luas dari LAN?, Yah, karena jaringan MAN itu terhubung dari beberapa jaringan LAN yang dihubungkan melalui switch lagi.
  \subsection{Wide Area Networking (WAN)}
    Yaitu Jaringan yang lingkupnya biasanya sudah menggunakan sarana Satelit ataupun kabel bawah laut sebagai contoh keseluruhan jaringan BANK BNI yang ada di Indonesia ataupun yang ada di Negara-negara lain. Menggunakan sarana WAN, Sebuah Bank yang ada di Bandung bisa menghubungi kantor cabangnya yang ada di Hongkong, hanya dalam beberapa menit. Biasanya WAN agak rumit dan sangat kompleks, menggunakan banyak sarana untuk menghubungkan antara LAN dan WAN ke dalam Komunikasi Global seperti Internet.

\section{Manfaat Jaringan Komputer}
  Berbicara mengenai manfaat dari jaringan komputer. Terdapat banyak sekali manfaat jaringan komputer, antara lain :
    \begin{enumerate}
      \item Dengan jaringan komputer, kita bisa mengakses file yang kita miliki sekaligus file orang lain yang telah diseberluaskan melalui suatu jaringan, semisal jaringan internet.
      \item Melalui jaringan komputer, kita bisa melakukan proses pengiriman data secara cepat dan efisien.
      \item Jaringan komputer membantu seseorang berhubungan dengan orang lain dari berbagai negara dengan mudah.
      \item Selain itu, pengguna juga dapat mengirim teks, gambar, audio, maupun video secara real time dengan bantuan jaringan komputer.
      \item Kita dapat mengakses berita atau informasi dengan sangat mudah melalui internet dikarenakan internet merupakan salah satu contoh jaringan komputer.
    \end{enumerate}

\section{Macam-Macam Jaringan Komputer}
  Umumnya jaringan komputer di kelompokkan menjadi 5 kategori, yaitu berdasarkan jangkauan geografis, distribusi sumber informasi/ data, media transmisi data, peranan dan hubungan tiap komputer dalam memproses data, dan berdasarkan jenis topologi yang digunakan. Berikut penjabaran lengkapnya :
  \subsection{A. Berdasarkan Jangkauan Geografis}
    \subsubsection{LAN}
      Local Area Network atau yang sering disingkat dengan LAN merupakan jaringan yang hanya mencakup wilayah kecil saja, semisal warnet, kantor, atau sekolah. Umumnya jaringan LAN luas areanya tidak jauh dari 1 km persegi.
    Biasanya jaringan LAN menggunakan teknologi IEEE 802.3 Ethernet yang mempunyai kecepatan transfer data sekitar 10, 100, bahkan 1000 MB/s. Selain menggunakan teknologi Ethernet, tak sedikit juga yang menggunakan teknologi nirkabel seperti Wi-fi untuk jaringan LAN.
    Keuntungan dari penggunaan Jenis Jaringan Komputer LAN seperti lebih irit dalam pengeluaran biaya operasional, lebih irit dalam penggunaan kabel, transfer data antar node dan komputer labih cepat karena mencakup wilayah yang sempit atau lokal, dan tidak memerlukan operator telekomunikasi untuk membuat sebuah jaringan LAN.
    Kerugian dari penggunaan Jenis Jaringan LAN adalah cakupan wilayah jaringan lebih sempit sehingga untuk berkomunikasi ke luar jaringan menjadi lebih sulit dan area cakupan transfer data tidak begitu luas.
    Berbeda dengan Jaringan Area Luas atau Wide Area Network(WAN), maka LAN mempunyai karakteristik sebagai berikut:
    \begin{
  \item Mempunyai pesat data yang lebih tinggi.
  \item Meliputi wilayah geografi yang lebih sempit.
  \item Tidak membutuhkan jalur telekomunikasi yang disewa dari dari operator telekomunikasi.
    \subsubsection{MAN}
      Metropolitan Area Network atau MAN merupakan jaringan yang mencakup suatu kota dengan dibekali kecepatan transfer data yang tinggi. Bisa dibilang, jaringan MAN merupakan gabungan dari beberapa jaringan LAN.
    Jangakauan dari jaringan MAN berkisar 10-50 km. MAN hanya memiliki satu atau dua kabel dan tidak dilengkapi dengan elemen switching yang berfungsi membuat rancangan menjadi lebih simple.
    Keuntungan dari Jenis Jaringan Komputer MAN ini diantaranya adalah cakupan wilayah jaringan lebih luas sehingga untuk berkomunikasi menjadi lebih efisien, mempermudah dalam hal berbisnis, dan juga keamanan dalam jaringan menjadi lebih baik.
    Kerugian dari Jenis Jaringan Komputer MAN seperti lebih banyak menggunakan biaya operasional, dapat menjadi target operasi oleh para Cracker untuk mengambil keuntungan pribadi, dan untuk memperbaiki jaringan MAN diperlukan waktu yang cukup lama.
    \subsubsection{WAN}
      Wide Area Network atau WAN merupakan jaringan yang jangkauannya mencakup daerah geografis yang luas, semisal sebuah negara bahkan benua. WAN umumnya digunakan untuk menghubungkan dua atau lebih jaringan lokal sehingga pengguna dapat berkomunikasi dengan pengguna lain meskipun berada di lokasi yang berbebeda.
    Keuntungan Jenis Jaringan Komputer WAN seperti cakupan wilayah jaringannya lebih luas dari Jenis Jaringan Komputer LAN dan MAN, tukar-menukar informasi menjadi lebih rahasia dan terarah karena untuk berkomunikasi dari suatu negara dengan negara yang lainnya memerlukan keamanan yang lebih, dan juga lebih mudah dalam mengembangkan serta mempermudah dalam hal bisnis.
    Kerugian dari Jenis Jaringan WAN seperti biaya operasional yang dibutuhkan menjadi lebih banyak, sangat rentan terhadap bahaya pencurian data-data penting, perawatan untuk jaringan WAN menjadi lebih berat.

B. Berdasarkan Distribusi Sumber Informasi/ Data

\begin{enumerate}
\item Jaringan Terpusat
Yang dimaksud jaringan terpusat adalah jaringan yang terdiri dari komputer client dan komputer server dimana komputer client bertugas sebagai perantara dalam mengakses sumber informasi/ data yang berasal dari komputer server. Dalam jaringan terpusat, terdapat istilah dumb terminal (terminal bisu), dimana terminal ini tidak memiliki alat pemroses data.
  
\item Jaringan Terdistribusi\end{enumerate}
 \par
Jaringan ini merupakan hasil perpaduan dari beberapa jaringan terpusat sehingga memungkinkan beberapa komputer server dan client yang saling terhubung membentuk suatu sistem jaringan tertentu. \par
\vspace{12pt}
\noindent
C. Berdasarkan Media Transmisi Data yang Digunakan \par
\noindent
\begin{enumerate}
\item Jaringan Berkabel (Wired Network) \par
Media transmisi data yang digunakan dalam jaringan ini berupa kabel. Kabel tersebut digunakan untuk menghubungkan satu komputer dengan komputer lainnya agar bisa saling bertukar informasi/ data atau terhubung dengan internet. Salah satu media transmisi yang digunakan dalam wired network adalah kabel UTP. \par
\vspace{12pt}
\noindent
\item Jaringan Nirkabel (Wireless Network)\end{enumerate}
 \par
Dalam jaringan ini diperlukan gelombang elektromagnetik sebagai media transmisi datanya. Berbeda dengan jaringan berkabel (wired network), jaringan ini tidak menggunakan kabel untuk bertukar informasi/ data dengan komputer lain melainkan menggunakan gelombang elektromagnetik untuk mengirimkan sinyal informasi/ data antar komputer satu dengan komputer lainnya. Wireless adapter, salah satu media transmisi yang digunakan dalam wireless network. \par
\vspace{12pt}
\noindent
D. Berdasarkan Peranan dan Hubungan Tiap Komputer dalam Memproses Data \par
\vspace{12pt}
\noindent
\begin{enumerate}
\item Jaringan Client-Server \par
Jaringan ini terdiri dari satu atau lebih komputer server dan komputer client. Biasanya terdiri dari satu komputer server dan beberapa komputer client. Komputer server bertugas menyediakan sumber daya data, sedangkan komputer client hanya dapat menggunakan sumber daya data tersebut. \par
\vspace{12pt}
\noindent
\item Jaringan Peer to Peer\end{enumerate}
 \par
Dalam jaringan ini, masing-masing komputer, baik itu komputer server maupun komputer client mempunyai kedudukan yang sama. Jadi, komputer server dapat menjadi komputer client, dan sebaliknya komputer client juga dapat menjadi komputer server. \par
\vspace{12pt}
\noindent
E. Berdasarkan Topologi Jaringan yang Digunakan \par
\noindent
Network Topology/ Topologi jaringan \par
\vspace{12pt}
Topologi jaringan adalah bentuk perancangan baik secara fisik maupun secara logik yang digunakan untuk membangun sebuah jaringan komputer. rancangan ini sangat erat kaitannya dengan metode access dan media pengiriman yang digunakan. Topologi yang ada sangatlah tergantung dengan letak geofrapis dari masing-masing terminal, kualitas kontrol yang dibutuhkan dalam komunikasi ataupun penyampaian pesan, serta kecepatan dari pengiriman data. \par
\noindent
\textbf{Apa saja alat-alat penting dalam networking itu ?} \par
\noindent
Macam-macam alat jaringan adalah :  \par
\noindent
\begin{enumerate}
\item ROUTER  \par
Router adalah sebuah alat yang mengirimkan paket data melalui sebuah jaringan atau Internet menuju tujuannya, alat ini sangatlah penting untuk meneruskan jaringan satu ke jaringan lainnya yang berbeda kelas/subnet/ip. melalui sebuah proses yang dikenal sebagai routing. Proses routing terjadi pada lapisan 3 (Lapisan jaringan seperti Internet Protocol) dari stack protokol tujuh-lapis OSI. \par
\vspace{12pt}
Router berfungsi sebagai penghubung antar dua atau lebih jaringan untuk meneruskan data dari satu jaringan ke jaringan lainnya. Router berbeda dengan switch. Switch merupakan penghubung beberapa alat untuk membentuk suatu Local Area Network (LAN). \par
\vspace{12pt}
\noindent
\item SWITCH \par
Switch adalah perangkat jaringan komputer yang bekerja di OSI Layer 2, Data Link Layer. Switch kerjanya sebagai penyambung atau concentrator dalam Jaringan komputer. Switch mengenal MAC Adressing shingga dia bisa memilah paket data mana yang akan di teruskan/dilanjutkan ke mana. \par
\vspace{12pt}
\noindent
\item ACCESS POINT\end{enumerate}
 \par
Access point adalah perangkat yang digunakan sebagai pembuat koneksi wireless pada jaringan komputer. Fungsi Access point diantaranya: Sebagai perangkat jaringan yang berfungsi membuat jaringan komputer tanpa kabel, atau biasa disebut WI-FI (Wireless Fidelity) \par
Belajar Network Programming pada python, melalui fungsi-fungsi TCP/IP, SOCKET, dll. Pada latihan ini, kita akan mencoba mengirim data dari server menuju klien dengan menggunakan Socket pada python. \par
\noindent
\begin{enumerate}
\item server.py\end{enumerate}
 \par
\vspace{12pt}
\noindent
Penjelasan fungsi-fungsi tsb akan dijelaskan dibawah ini: \par
\noindent
\begin{itemize}
\item socket.socket(): Membuat socket baru menggunakan alamat yang sudah ada, tipe socket, dan nomor protocol. \par
\noindent
\item socket.bind(address): Menyalin/mengikat socket ke alamat yang ada. \par
\noindent
\item socket.listen(backlog): Menunggu koneksi yang sudah dibuat dari socket tersebut. backlog merupakan sebuah argumen yang menyatakan batas maximal nomor antrian koneksi dan paling tidak sampai dengan 0; nilai maximum tergantung dari sistem(biasanya 5), dan nilai minimumnya harus mencapai 0. \par
\noindent
\item socket.accept(): Nilai yang dikembalikan atau diberikan adalah sepasang(conn, address) dimana conn adalah socket baru yaitu sebuah objek yang biasa digunakan untuk mengirim dan menerima data dari koneksi tersebut dan address adalah alamat yang terikat ke socket pada akhir koneksi. \par
\noindent
\item socket.send(bytes[, flags]): Mengiri data ke socket. Socket harus terkoneksi oleh remote Socket. mengembalikan angkat dari bytes yang terkirim. Aplikasi yang bertugas untuk mengecek semua data harus terkirim; hanya jika data ditransimisikan, aplikasi membutuhkan usaha untuk mengirimkan data yang tersisa. \par
\noindent
\item socket.close(): Menandakan bahwa socket telah ditutup. semua dari operasi-operasi pada objek socket akan gagal. Remote End tidak akan menerima data lagi (sampai data telah dibersihkan). Socket-socket secara otomatis tertutup ketika dilakukan garbage-collected, tetapi lebih baik untuk close() mereka secara eksplisit.\end{itemize}
 \par
\vspace{12pt}
Sebelumnya pesan diatas tidak akan muncul sebelum kita menjalankan script client.py pada tab terminal lain. \par
maka setiap kali kita menjalankan script client.py akan terus mengirimkan pesan kepada server maupun client. \par
\noindent
\vspace{12pt}
