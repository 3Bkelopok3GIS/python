Kelebihan dan Kekurangan
Kelebihan :
•	Tidak ada tahapan kompilasi dan penyambungan (link) sehingga kecepatan perubahan pada masa pembuatan sistem aplikasi meningkat.
•	Tidak ada deklarasi tipe data yang merumitkan sehingga program menjadi lebih sederhana, singkat, dan fleksible.
•	Manajemen memori otomatis yaitu kumpulan sampah memori sehingga dapat menghindari pencacatan kode.
•	Tipe data dan operasi tingkat tinggi yaitu kecepatan pembuatan sistem aplikasi menggunakan tipe objek yang telah ada.
•	Pemrograman berorientasi objek.
•	Pelekatan dan perluasan dalam C.
•	Terdapat kelas, modul, eksepsi sehingga terdapat dukungan pemrograman skala besar secara modular.
•	Pemuatan dinamis modul C sehingga ekstensi menjadi sederhana dan berkas biner yang kecil
•	Pemuatan kembali secara dinamis modul phyton seperti memodifikasi aplikasi tanpa menghentikannya.
•	Model objek universal kelas Satu.
•	Konstruksi pada saat aplikasi berjalan.
•	Interaktif, dinamis dan alamiah.
•	Akses hingga informasi interpreter.
•	Portabilitas secara luas seperti pemrograman antar platform tanpa ports.
•	Kompilasi untuk portable kode byte sehingga kecepatan eksekusi bertambah dan melindungi kode sumber.
•	Antarmuka terpasang untuk pelayanan keluar seperti perangkat Bantu system, GUI, persistence, database, dll.

Kekurangan :
•	Beberapa penugasan terdapat diluar dari jangkauan python, seperti bahasa pemrograman dinamis lainnya, python tidak secepat atau efisien sebagai statis, tidak seperti bahasa pemrograman kompilasi seperti bahasa C.
•	Disebabkan python merupakan interpreter, python bukan merupakan perangkat bantu terbaik untuk pengantar komponen performa kritis.
•	Python tidak dapat digunakan sebagai dasar bahasa pemrograman implementasi untuk beberapa komponen, tetapi dapat bekerja dengan baik sebagai bagian depan skrip antarmuka untuk mereka.
•	Python memberikan efisiensi dan fleksibilitas tradeoff by dengan tidak memberikannya secara menyeluruh. Python menyediakan bahasa pemrograman optimasi untuk kegunaan, bersama dengan perangkat bantu yang dibutuhkan untuk diintegrasikan dengan bahasa pemrograman lainnya.
Banyak terdapat referensi lama terutama dari pencarian google, python adalah pemrograman yang sangat lambat. Namun belum lama ini ditemukan bahwa Google, Youtube, DropBox dan beberapa software sistem banyak menggunakan Python. Bahkan terakhir Google merilis big Data Processing API enginenya (MapReduce) di Java dan Python (Link). Meski yang “katanya” Python adalah pemrograman yang lambat dari beberapa bechmark, tetapi tidak begitu terbukti mempengaruhi kemudahan dalam penggunaannya.
