% Kelas D4 TI 3B Kelompok 3
% Diana Satima Gistivani 1154018
% M. Amran Hakim Siregar 1154106
% Indah Rahmawati 1154070


\section{Python XML Processing}
  XML adalah bahasa open source portable yang mungkinkan pemrogram mengemangkan aplikasi yang dapat dibaca oleh aplikasi lain, bahasa markup untuk keperluan umum yang disarankan oleh W3C untuk membuat dokumen markup keperluan pertukaran data antar sistem yang beraneka ragam. XML merupakan kelanjutan dari HTML (HyperText Markup Language) yang merupakan bahasa standar untuk melacak Internet.
terlepas dari sistem operasi dan bahasa pengembangnya .
\subsection{Apa itu XML}
  Extensible Markup Languange (XML) adalah bahasa markup seperti HTML atau SGML. 
Ini direkomendasikan oleh World Wide Web Consortium dan tersedia sebagai standar terbuka.
XML sangat berguna untuk mencatat data berukuran kecil dan menengah tanpa memerlukan tulang punggung berbasis SQL. 


\vspace{12pt}
\noindent 
\textbf{2.1 Arsitektur }\textbf{Parsing}\textbf{ XML dan API} \par
\noindent 
 \hspace*{0.5in} Perpustakaan standar Python menyediakan seperangkat antarmuka minimal tapi berguna untuk bekerja dengan XML.  \par
\noindent 
 \hspace*{0.5in} Dua API yang paling dasar dan umum digunakan untuk data XML adalah antarmuka SAX dan DOM. \par
\noindent 
 \hspace*{0.5in} API sederhana untuk XML (SAX): mendaftarkan panggilan kemali untuk acara yang diminati dan kemudian membiarkan parser berjalan melalui dokumen. Ini berguna bila dokumen berukuran besar atau memiliki keterbatasan memori, ini memparsing file tidak pernah tersimpan dalam memori. \par
\noindent 
 \hspace*{0.5in} API Document Objek Model (DOM): ini adalah rekomendasi World Wide Web Consortium dimana keseluruhan file dibaca ke memori dan disimpan dalam bentuk hierarkies (tree-based) untuk mewakili semua fitur dokumen XML.  \par
\noindent 
 \hspace*{0.5in} SAX jelas tidak bisa memproses informasi secepat DOM saat bisa bekerjadengan file besar. Di sisi lain, menggunakan DOM secara eklusifenar-benar dapat membunuh sumber daya, terutama jika digunakan pada banyak file kecil. \par
\noindent 
~~~~~~~~~~~ SAX hanya bisa dibaca sementara DOM mengizinkan perubahan pada file XML. Kedua API yang berbeda ini saling melengkapi satu sama lain, tidak ada alasan mengapa tidak dapat menggunakannya untuk proyek besar. \par
\vspace{12pt}
\noindent 
Contoh: \par
\noindent 
<collection shelf="New Arrivals"> \par
\noindent 
<movie title="Enemy Behind"> \par
\noindent 
~~ <type>War, Thriller</type> \par
\noindent 
~~ <format>DVD</format> \par
\noindent 
~~ <year>2003</year> \par
\noindent 
~~ <rating>PG</rating> \par
\noindent 
~~ <stars>10</stars> \par
\noindent 
~~ <description>Talk about a US-Japan war</description> \par
\noindent 
</movie> \par
\noindent 
<movie title="Transformers"> \par
\noindent 
~~ <type>Anime, Science Fiction</type> \par
\noindent 
~~ <format>DVD</format> \par
\noindent 
~~ <year>1989</year> \par
\noindent 
~~ <rating>R</rating> \par
\noindent 
~~ <stars>8</stars> \par
\noindent 
~~ <description>A schientific fiction</description> \par
\noindent 
</movie> \par
\noindent 
~~ <movie title="Trigun"> \par
\noindent 
~~ <type>Anime, Action</type> \par
\noindent 
~~ <format>DVD</format> \par
\noindent 
~~ <episodes>4</episodes> \par
\noindent 
~~ <rating>PG</rating> \par
\noindent 
~~ <stars>10</stars> \par
\noindent 
~~ <description>Vash the Stampede!</description> \par
\noindent 
</movie> \par
\noindent 
<movie title="Ishtar"> \par
\noindent 
~~ <type>Comedy</type> \par
\noindent 
~~ <format>VHS</format> \par
\noindent 
~~ <rating>PG</rating> \par
\noindent 
~~ <stars>2</stars> \par
\noindent 
~~ <description>Viewable boredom</description> \par
\noindent 
</movie> \par
\noindent 
</collection> \par

\subsection{Parsing XML dengan API SAX}
  SAX adalah antarmuka standar untuk parsing XML berbasis event. Parsing XML dengan SAX umumnya mengharuskan untuk membuat dengan subclassing xml.sax.
  ControlHandler menangani tag dan atribut tertentu dari XML. Objek ControlHandler menyediakan metode untuk menangani berbagai aktivitas parsing. Parsing memanggil metode ControlHandler saat memparsing file XML.
  Metode startDocument dan endDocument disebut awal dan akhir setiap elemen. Jika parsing tidak dalam mode namespace, metode startElement (tag attribute) dan endElement (tag) dipanggil. Jika tidak, metode yang sesuai startElemenNS dan endElemenNS dipanggil. 
\begin{itemize}
Berikut ini metode penting untuk memahami sebelum melanjutkan ke materi berikutnya : 
\item Metode berikut membuat objek parsing baru dan mengembalikannya. Objek parsing diuat akan menjadi tipe parsing pertama yang ditemukan sistem.
\selectfont xml.sax.make $  \_  $parser([parser $  \_  $list])}
\end{itemize}

\begin{itemize}
Berikut adalah detail parameternya : 
Parser $  \_  $list : pilihan argumen yang terdiri dari daftar parsing untuk digunakan yang semuanya harus menerapkan metode \textit{make $  \_  $parse}
\end{itemize}

\noindent 
Metode\par
Metode berikut membuat parsing SAX dan menggunakannya untuk mengurai dokumen \par
\vspace{10pt}
{\fontsize{10pt}{10pt}\selectfont xml.sax.parser(xmlfile, contenthandler[, errorhandler])} \par
\vspace{10pt}
Berikut adalah detail dari parameternya: \par
\noindent 
\begin{itemize}
\item \textit{Xmlfile } \par
Ini adalah nama file XML yang bisa dibaca. \par
\noindent 
\item \textit{ContentHandler } \par
Ini harus menjadi objek \textit{ContenHandler} \par
\noindent 
\item \textit{ErrorHandler} \par
Jika ditentukan, e\textit{rrorhandler} harus menjadi objek \textit{ErrorHandler} SAX \par
\noindent 
\item Metode\textit{ parseString}\end{itemize}
 \par
Membuat parsing SAX dan mengurai string XML yang ditentukan : \par
\vspace{12pt}
{\fontsize{10pt}{10pt}\selectfont xml.sax.parsertring(xmlstring,contenthandler[, errorhandler])} \par
\vspace{12pt}
Brikut ini adalah detail nama dar parameter : \par
\noindent 
{XMLstring} \par
Nama dari string yang bisa dibaca \par
\noindent 
{ContentHandler} \par
Menjadi objek ContenHandler \par
\noindent 
{ErrorHandler} \par
Menjadi objek ErorHandler SAX \par
\vspace{12pt}
\noindent 
Contoh : \par
\noindent 
 $  \#  $!/usr/bin/python \par
\vspace{12pt}
\noindent 
import xml.sax \par
\vspace{12pt}
\noindent 
class MovieHandler( xml.sax.ContentHandler ): \par
\noindent 
~~ def  $  \_  $ $  \_  $init $  \_  $ $  \_  $(self): \par
\noindent 
~~~~~ self.CurrentData = "" \par
\noindent 
~~~~~ self.type = "" \par
\noindent 
~~~~~ self.format = "" \par
\noindent 
~~~~~ self.year = "" \par
\noindent 
~~~~~ self.rating = "" \par
\noindent 
~~~~~ self.stars = "" \par
\noindent 
~~~~~ self.description = "" \par
\vspace{12pt}
\noindent 
~~  $  \#  $ Call when an element starts \par
\noindent 
~~ def startElement(self, tag, attributes): \par
\noindent 
~~~~~ self.CurrentData = tag \par
\noindent 
~~~~~ if tag == "movie": \par
\noindent 
~~~~~~~~ print "*****Movie*****" \par
\noindent 
~~~~~~~~ title = attributes["title"] \par
\noindent 
~~~~~~~~ print "Title:", title \par
\vspace{12pt}
\noindent 
~~  $  \#  $ Call when an elements ends \par
\noindent 
~~ def endElement(self, tag): \par
\noindent 
~~~~~ if self.CurrentData == "type": \par
\noindent 
~~~~~~~~ print "Type:", self.type \par
\noindent 
~~~~~ elif self.CurrentData == "format": \par
\noindent 
~~~~~~~~ print "Format:", self.format \par
\noindent 
~~~~~ elif self.CurrentData == "year": \par
\noindent 
~~~~~~~~ print "Year:", self.year \par
\noindent 
~~~~~ elif self.CurrentData == "rating": \par
\noindent 
~~~~~~~~ print "Rating:", self.rating \par
\noindent 
~~~~~ elif self.CurrentData == "stars": \par
\noindent 
~~~~~~~~ print "Stars:", self.stars \par
\noindent 
~~~~~ elif self.CurrentData == "description": \par
\noindent 
~~~~~~~~ print "Description:", self.description \par
\noindent 
~~~~~ self.CurrentData = "" \par
\vspace{12pt}
\noindent 
~~  $  \#  $ Call when a character is read \par
\noindent 
~~ def characters(self, content): \par
\noindent 
~~~~~ if self.CurrentData == "type": \par
\noindent 
~~~~~~~~ self.type = content \par
\noindent 
~~~~~ elif self.CurrentData == "format": \par
\noindent 
~~~~~~~~ self.format = content \par
\noindent 
~~~~~ elif self.CurrentData == "year": \par
\noindent 
~~~~~~~~ self.year = content \par
\noindent 
~~~~~ elif self.CurrentData == "rating": \par
\noindent 
~~~~~~~~ self.rating = content \par
\noindent 
~~~~~ elif self.CurrentData == "stars": \par
\noindent 
~~~~~~~~ self.stars = content \par
\noindent 
~~~~~ elif self.CurrentData == "description": \par
\noindent 
~~~~~~~~ self.description = content \par
\noindent 
~  \par
\noindent 
if (  $  \_  $ $  \_  $name $  \_  $ $  \_  $ == " $  \_  $ $  \_  $main $  \_  $ $  \_  $"): \par
\noindent 
~~  \par
\noindent 
~~  $  \#  $ create an XMLReader \par
\noindent 
~~ parser = xml.sax.make $  \_  $parser() \par
\noindent 
~~  $  \#  $ turn off namepsaces \par
\noindent 
~~ parser.setFeature(xml.sax.handler.feature $  \_  $namespaces, 0) \par
\vspace{12pt}
\noindent 
~~  $  \#  $ override the default ContextHandler \par
\noindent 
~~ Handler = MovieHandler() \par
\noindent 
~~ parser.setContentHandler( Handler ) \par
\noindent 
~~  \par
\noindent 
~~ parser.parse("movies.xml") \par
\vspace{12pt}
\vspace{12pt}
\noindent 
Ini akan menghasilkan hasil sebagai berikut: \par
\noindent 
{\fontsize{10pt}{10pt}\selectfont *****Movie******} \par
\noindent 
{\fontsize{10pt}{10pt}\selectfont *****Movie*****} \par
\noindent 
{\fontsize{10pt}{10pt}\selectfont Title: Enemy Behind} \par
\noindent 
{\fontsize{10pt}{10pt}\selectfont Type: War, Thriller} \par
\noindent 
{\fontsize{10pt}{10pt}\selectfont Format: DVD} \par
\noindent 
{\fontsize{10pt}{10pt}\selectfont Year: 2003} \par
\noindent 
{\fontsize{10pt}{10pt}\selectfont Rating: PG} \par
\noindent 
{\fontsize{10pt}{10pt}\selectfont Stars: 10} \par
\noindent 
{\fontsize{10pt}{10pt}\selectfont Description: Talk about a US-Japan war} \par
\noindent 
{\fontsize{10pt}{10pt}\selectfont *****Movie*****} \par
\noindent 
{\fontsize{10pt}{10pt}\selectfont Title: Transformers} \par
\noindent 
{\fontsize{10pt}{10pt}\selectfont Type: Anime, Science Fiction} \par
\noindent 
{\fontsize{10pt}{10pt}\selectfont Format: DVD} \par
\noindent 
{\fontsize{10pt}{10pt}\selectfont Year: 1989} \par
\noindent 
{\fontsize{10pt}{10pt}\selectfont Rating: R} \par
\noindent 
{\fontsize{10pt}{10pt}\selectfont Stars: 8} \par
\noindent 
{\fontsize{10pt}{10pt}\selectfont Description: A schientific fiction} \par
\noindent 
{\fontsize{10pt}{10pt}\selectfont *****Movie*****} \par
\noindent 
{\fontsize{10pt}{10pt}\selectfont Title: Trigun} \par
\noindent 
{\fontsize{10pt}{10pt}\selectfont Type: Anime, Action} \par
\noindent 
{\fontsize{10pt}{10pt}\selectfont Format: DVD} \par
\noindent 
{\fontsize{10pt}{10pt}\selectfont Rating: PG} \par
\noindent 
{\fontsize{10pt}{10pt}\selectfont Stars: 10} \par
\noindent 
{\fontsize{10pt}{10pt}\selectfont Description: Vash the Stampede!} \par
\noindent 
{\fontsize{10pt}{10pt}\selectfont *****Movie*****} \par
\noindent 
{\fontsize{10pt}{10pt}\selectfont Title: Ishtar} \par
\noindent 
{\fontsize{10pt}{10pt}\selectfont Type: Comedy} \par
\noindent 
{\fontsize{10pt}{10pt}\selectfont Format: VHS} \par
\noindent 
{\fontsize{10pt}{10pt}\selectfont Rating: PG} \par
\noindent 
{\fontsize{10pt}{10pt}\selectfont Stars: 2} \par
\noindent 
\vspace{10pt}
\noindent 
\textbf{2.3 Parsing XML dengan API DOM} \par
\noindent 
 \hspace*{0.5in} Document Ovject Model (DOM) adalah API lintas bahasa dari World Wide Web Consortium (W3C) untuk mengakses dan memodifikasi dokumen XML. \par
\noindent 
 \hspace*{0.5in} DOM sangat berguna untuk aplikasi akses acak. SAX hanya memungkinkan melihat satu bit dokumen sekaligus. Jika melihat satu elemen SAX, tidak memiliki akses ke yang lain. \par
\noindent 
 \hspace*{0.5in} Berikut adalah cara termudah untuk memuat dokumen XML dengan cepat dan membuat objek minidom menggunakan modul xml.dom. Objek minidom menyediakan metode parsing sederhana yang dengan cepat memuat pohon DOM dari file XML. \par
\noindent 
 \hspace*{0.5in} Contoh~frase memanggil fungsi  parsing (file [,parsing]) dari objek minidokumen untuk mengurai file XML yang ditunjuk oleh file ke objek pohon DOM. \par
\noindent 
 $  \#  $!/usr/bin/python \par
\vspace{12pt}
\noindent 
from xml.dom.minidom import parse \par
\noindent 
import xml.dom.minidom \par
\vspace{12pt}
\noindent 
 $  \#  $ Open XML document using minidom parser \par
\noindent 
DOMTree = xml.dom.minidom.parse("movies.xml") \par
\noindent 
collection = DOMTree.documentElement \par
\noindent 
if collection.hasAttribute("shelf"): \par
\noindent 
~~ print "Root element :  $  \%  $s"  $  \%  $ collection.getAttribute("shelf") \par
\vspace{12pt}
\noindent 
 $  \#  $ Get all the movies in the collection \par
\noindent 
movies = collection.getElementsByTagName("movie") \par
\vspace{12pt}
\noindent 
 $  \#  $ Print detail of each movie. \par
\noindent 
for movie in movies: \par
\noindent 
~~ print "*****Movie*****" \par
\noindent 
~~ if movie.hasAttribute("title"): \par
\noindent 
~~~~~ print "Title:  $  \%  $s"  $  \%  $ movie.getAttribute("title") \par
\vspace{12pt}
\noindent 
~~ type = movie.getElementsByTagName('type')[0] \par
\noindent 
~~ print "Type:  $  \%  $s"  $  \%  $ type.childNodes[0].data \par
\noindent 
~~ format = movie.getElementsByTagName('format')[0] \par
\noindent 
~~ print "Format:  $  \%  $s"  $  \%  $ format.childNodes[0].data \par
\noindent 
~~ rating = movie.getElementsByTagName('rating')[0] \par
\noindent 
~~ print "Rating:  $  \%  $s"  $  \%  $ rating.childNodes[0].data \par
\noindent 
~~ description = movie.getElementsByTagName('description')[0] \par
\noindent 
~~ print "Description:  $  \%  $s"  $  \%  $ description.childNodes[0].data \par
\vspace{12pt}
\noindent 
Ini akan menghasilkan hasil sebagai berikut : \par
\noindent 
Root element : New Arrivals \par
\noindent 
*****Movie***** \par
\noindent 
Title: Enemy Behind \par
\noindent 
Type: War, Thriller \par
\noindent 
Format: DVD \par
\noindent 
Rating: PG \par
\noindent 
Description: Talk about a US-Japan war \par
\noindent 
*****Movie***** \par
\noindent 
Title: Transformers \par
\noindent 
Type: Anime, Science Fiction \par
\noindent 
Format: DVD \par
\noindent 
Rating: R \par
\noindent 
Description: A schientific fiction \par
\noindent 
*****Movie***** \par
\noindent 
Title: Trigun \par
\noindent 
Type: Anime, Action \par
\noindent 
Format: DVD \par
\noindent 
Rating: PG \par
\noindent 
Description: Vash the Stampede! \par
\noindent 
*****Movie***** \par
\noindent 
Title: Ishtar \par
\noindent 
Type: Comedy \par
\noindent 
Format: VHS \par
\noindent 
Rating: PG \par
\noindent 
Description: Viewable boredom \par
\vspace{12pt}
\noindent 
\textbf{2.4}\textbf{ Membangun Parsing Document XML menggunakan Python} \par
\noindent 
 \hspace*{0.5in} Python mendukung untuk bekerja dengan berbagai bentuk markup data terstruktur. Selain mengurai xml.etree. \textit{ElementTree} mendukung pembuatan dokumen XML yang terbentuk dengan baik dari objek elemen yang dibangun dalam aplikasi. Kelas elemen digunakakan saat sebuah dokumen diurai untuk mengetahui bagaimana menghasilkan bentuk serial dari isinya kemudian dapat ditulis ke sebuah file.  \par
\vspace{12pt}
\noindent 
 \hspace*{0.5in} Untuk membuat instance elemeb gunakan fungsi elemen contructor dan \textit{SubElemen()} pabrik. \par
\noindent 
Import xml.etree.ElementTree as xml \par
\vspace{12pt}
\noindent 
{\fontsize{10pt}{10pt}\selectfont filename =  $ " $/home/abc/Desktop/test $  \_  $xml.xml $ " $} \par
\noindent 
{\fontsize{10pt}{10pt}\selectfont toot = xml.Element( $ " $Users $ " $)} \par
\noindent 
{\fontsize{10pt}{10pt}\selectfont userelement = xml.Element( $ " $user $ " $)} \par
\noindent 
{\fontsize{10pt}{10pt}\selectfont root.append(userelement)} \par
\noindent 
\vspace{10pt}
\noindent 
Bila menjalankan ini, akan menghasilkan sebagai berikut : \par
\noindent 
{\fontsize{10pt}{10pt}\selectfont <Users>} \par
\noindent 
{\fontsize{10pt}{10pt}\selectfont  \hspace*{0.5in} <user>} \par
\noindent 
{\fontsize{10pt}{10pt}\selectfont  \hspace*{0.5in} <user>} \par
\noindent 
{\fontsize{10pt}{10pt}\selectfont </Users>} \par
\vspace{10pt}
\vspace{10pt}
\vspace{10pt}
\noindent 
Tambahkan anak-anak pegguna \par
\vspace{10pt}
\noindent 
{\fontsize{10pt}{10pt}\selectfont Uid = xml.SubElement(userelement,  $ " $uid $ " $)} \par
\noindent 
{\fontsize{10pt}{10pt}\selectfont Uid.text =  $ " $1 $ " $} \par
\vspace{10pt}
\noindent 
{\fontsize{10pt}{10pt}\selectfont FirstName = xml.SubElement(userelement,  $ " $FirstName $ " $)} \par
\noindent 
{\fontsize{10pt}{10pt}\selectfont FirstName.text =  $ " $testuser $ " $} \par
\vspace{10pt}
\noindent 
{\fontsize{10pt}{10pt}\selectfont LastName = xml.SubElement(userelement,  $ " $LastName $ " $} \par
\noindent 
{\fontsize{10pt}{10pt}\selectfont LastName.text =  $ " $testuser $ " $} \par
\vspace{10pt}
\noindent 
{\fontsize{10pt}{10pt}\selectfont Email = xml.SubElement(userelement,  $ " $Email $ " $)} \par
\noindent 
{\fontsize{10pt}{10pt}\selectfont Email.text = {mailto:testuser@test.com}{testuser@test.com}
} \par
\vspace{10pt}
\noindent 
{\fontsize{10pt}{10pt}\selectfont state = xml.SubElement(userelemet,  $ " $state $ " $)} \par
\noindent 
{\fontsize{10pt}{10pt}\selectfont state.text =  $ " $xyz $ " $} \par
\vspace{10pt}
\noindent 
{\fontsize{10pt}{10pt}\selectfont location = xml.SubElement(userelement,  $ " $location)} \par
\noindent 
{\fontsize{10pt}{10pt}\selectfont location.text = abc} \par
\vspace{10pt}
\noindent 
{\fontsize{10pt}{10pt}\selectfont tree = xml.ElementTree(root)} \par
\noindent 
{\fontsize{10pt}{10pt}\selectfont with open(filename,  $ " $w $ " $) as fh:} \par
\noindent 
{\fontsize{10pt}{10pt}\selectfont tree.write(fh)} \par
\vspace{10pt}
\noindent 
 \hspace*{0.5in} Pertama buat elemen root dengan mengunakan fungsi \textit{ElementTree}. Kemudian membuat elemen pegguna dan menambahkannya ke root. Selanjutnya membuat \textit{SubElement }dengan melewatkan elemen pengguna (userelement) ke \textit{SubElemen} beserta namanya seperto  $ " $FirstName $ " $. Kemudian untuk setiap \textit{SubElement} tetapkan properti teks untuk memberi nilai. Di akhir, membuat \textit{ElementTree} dan menggunakannya untuk menulis XML ke file. \par
\noindent 
 \hspace*{0.5in} Jika menjalankan ini akan menjadi sebagai berikut : \par
\noindent 
 {\fontsize{10pt}{10pt}\selectfont <users>} \par
\noindent 
{\fontsize{10pt}{10pt}\selectfont  \hspace*{0.5in} <user>} \par
\noindent 
{\fontsize{10pt}{10pt}\selectfont  \hspace*{0.5in}  \hspace*{0.5in} <uid>1</uid>} \par
\noindent 
{\fontsize{10pt}{10pt}\selectfont  \hspace*{0.5in}  \hspace*{0.5in} <FirstName>testuser</FirstName>} \par
\noindent 
{\fontsize{10pt}{10pt}\selectfont  \hspace*{0.5in}  \hspace*{0.5in} <LastName>testuser</LastName>} \par
\noindent 
{\fontsize{10pt}{10pt}\selectfont  \hspace*{0.5in}  \hspace*{0.5in} <state>xyz</state>} \par
\noindent 
{\fontsize{10pt}{10pt}\selectfont  \hspace*{0.5in}  \hspace*{0.5in} <location>abc</location>} \par
\noindent 
{\fontsize{10pt}{10pt}\selectfont  \hspace*{0.5in} </user>} \par
\noindent 
{\fontsize{10pt}{10pt}\selectfont </Users>} \par
\vspace{10pt}
\noindent 
Parsing XML Documen : \par
\vspace{12pt}
\noindent 
{\fontsize{10pt}{10pt}\selectfont import xml.etree.ElementTree as ET} \par
\noindent 
{\fontsize{10pt}{10pt}\selectfont tree = ET.parse(‘Your $  \_  $XML $  \_  $file $  \_  $path’)} \par
\noindent 
{\fontsize{10pt}{10pt}\selectfont root = tree.getroot()} \par
\noindent 
{\fontsize{10pt}{10pt}\selectfont 

 %%%%%%%%%%%%  Start New Page here %%%%%%%%%%%%%%


\newpage

}\vspace{10pt}
\vspace{10pt}
\noindent 
Disini \textit{getroot()} akan mengembalikan elemen dari dokumen XML \par
\vspace{10pt}
\noindent 
{\fontsize{10pt}{10pt}\selectfont <Users version= $ " $1.0 $ " $ languange= $ " $SPA $ " $>} \par
\noindent 
{\fontsize{10pt}{10pt}\selectfont  \hspace*{0.5in} <user>} \par
\noindent 
{\fontsize{10pt}{10pt}\selectfont  \hspace*{0.5in}  \hspace*{0.5in} <uid>1</uid>} \par
\noindent 
{\fontsize{10pt}{10pt}\selectfont  \hspace*{0.5in}  \hspace*{0.5in} <FirstName>testuser</FirstName>} \par
\noindent 
{\fontsize{10pt}{10pt}\selectfont  \hspace*{0.5in}  \hspace*{0.5in} <LastName>testuser</LastName>} \par
\noindent 
{\fontsize{10pt}{10pt}\selectfont  \hspace*{0.5in}  \hspace*{0.5in} <Email>testuser@tes.com/Email>} \par
\noindent 
{\fontsize{10pt}{10pt}\selectfont  \hspace*{0.5in}  \hspace*{0.5in} <state>xyz</state>} \par
\noindent 
{\fontsize{10pt}{10pt}\selectfont  \hspace*{0.5in}  \hspace*{0.5in} <location>abc</location>} \par
\noindent 
{\fontsize{10pt}{10pt}\selectfont  \hspace*{0.5in} </user>} \par
\noindent 
{\fontsize{10pt}{10pt}\selectfont </Users>} \par
\vspace{12pt}
