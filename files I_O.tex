%Kelompok 1 D4 TI 3D
%Wahyu Maruti Adjie_1154034
%Muhammad Nur Ikhsan_1154087
%Emy Safitri_1154102
%Andi Ikram Maulana_1154065
%Ilman Mubarik Sidiq_1154114

<<<<<<< master

\section{python files I/O} 

\subsection{Print}
Untuk menghasilkan output  pada phyton dapat menggunakan pernyataan cetak, di mana Anda bisa melewati nol atau lebih banyak ekspresi yang dipisahkan dengan koma. Fungsi tersebut dapat mengubah data yang diberikan ke string dan menulis hasilnya ke output.
contoh:  print ("Wahyu Maruti Adjie adalah mahasiswa yang hebat")

\subsection{Membaca input keyboard}
Dalam Python2 terdapat  dua fungsi built-in yang berfungsi untuk membaca data dari input standar, yang secara default berasal dari keyboard, dua fungsi tersebut yaitu input() dan raw_input(). Dalam Python3, fungsi raw_input() tidak dipakai. Dan input() bertugas menerjemahkan data dari keyboard sebagai string.

\subsection{fuction python}
Fungsi input([prompt]) memiliki fungsi yang sama dengan raw_input, yakni berfungsi untuk membaca masukan, tetapi dalam melakukan pembacaan masukan keduanya punya fungsi yang bedanya . 

\subsection{raw_input()} 
Segala inputan yang diterima oleh fungsi raw_input() di anggap sebagai inputan string.

\subsection{input()}
 fungsi input() akan mengambil bilangan, sehingga inputan dapat di olah oleh operasi aritmatika.
contoh : >>> x = input("something:")
something:10

>>> x
'10'

\subsection{mode}
Kita bisa menentukan mode saat membuka file. Dalam mode, kami menentukan apakah kita ingin membaca 'r', menulis 'w' atau menambahkan 'a' ke file. Kita juga menentukan apakah kita ingin membuka file dalam mode teks atau mode biner. Defaultnya adalah membaca dalam mode teks. 

\subsection{ASCII}
Tidak seperti bahasa lain, karakter 'a' tidak menyiratkan angka 97 sampai dikodekan menggunakan ASCII (atau pengkodean setara lainnya). Apalagi, pengkodean default bergantung pada platform. Di jendela, itu adalah 'cp1252' tapi 'utf-8' di Linux. Jadi, kita juga tidak harus bergantung pada pengkodean default atau kode kita akan berperilaku berbeda di berbagai platform.

\subsection{close}
Ketika kita selesai dengan operasi ke file, kita perlu menutupinya dengan benar. Menutup file akan membebaskan sumber daya yang terkait dengan file dan dilakukan dengan menggunakan metode close (). Python memiliki pengumpul sampah untuk membersihkan benda yang tidak difermentasi tapi, kita tidak boleh bergantung padanya untuk menutup file.

\subsection{try}
Metode ini tidak sepenuhnya aman. Jika pengecualian terjadi saat kita melakukan operasi dengan file, kode keluar tanpa menutup file. Cara yang lebih aman adalah dengan menggunakan try. Dengan cara ini, kita dijamin bahwa file tersebut benar tertutup bahkan jika pengecualian dinaikkan, menyebabkan aliran program berhenti. Cara terbaik untuk melakukannya adalah dengan menggunakan pernyataan. Ini memastikan file ditutup saat blok di dalam dengan keluar. Kita tidak perlu secara eksplisit memanggil metode close (). Hal itu dilakukan secara internal.

\subsection{string}
Untuk menulis ke file kita perlu membukanya dalam mode write 'w', tambahkan 'a' atau exclusive creation 'x'. Kita harus berhati-hati dengan mode 'w' karena akan menimpa file jika sudah ada. Semua data sebelumnya terhapus. Menulis string atau urutan byte (untuk file biner) dilakukan dengan menggunakan metode write (). Metode ini mengembalikan jumlah karakter yang ditulis ke file. 

\subsection{file}
Program ini akan membuat file baru bernama 'test.txt' jika tidak ada. Jika memang ada, itu akan ditimpa. Kita harus menyertakan karakter newline sendiri untuk membedakan garis yang berbeda.Untuk membaca isi sebuah file, kita harus membuka file dalam mode baca. Ada berbagai metode yang tersedia untuk tujuan ini. Kita bisa menggunakan metode read (size) untuk membaca dalam jumlah ukuran data. Jika parameter ukuran tidak ditentukan, bunyinya dan kembali ke akhir file. 

\subsection{method}
Kita dapat melihat, metode read () mengembalikan baris baru sebagai.Begitu akhir file tercapai, kita mendapatkan string kosong untuk dibaca lebih lanjut. Kita bisa mengubah kursor file kita saat ini (posisi) dengan menggunakan metode seek (). Demikian pula metode tell () mengembalikan posisi kita saat ini (dalam jumlah byte). Kita bisa membaca file line-by-line menggunakan for loop. 

\subsection{line}
Baris dalam file itu sendiri memiliki karakter baris baru. Terlebih lagi, print () parameter akhir untuk menghindari dua baris baru saat mencetak. Sebagai alternatif, kita dapat menggunakan metode readline () untuk membaca setiap baris file. Metode ini membaca sebuah file sampai newline, termasuk newline character.Terakhir, metode readlines () mengembalikan daftar baris yang tersisa dari keseluruhan file. 
\subsection{objek}
Objek file menyediakan seperangkat metode akses untuk membuat hidup kita lebih mudah. Kita akan melihat bagaimana menggunakan metode read () dan write () untuk membaca dan menulis file. Metode tulis () Metode write () menulis string apapun ke file yang terbuka. Penting untuk dicatat bahwa string Python dapat memiliki data biner dan bukan hanya teks. 
\subsection{openfiles}
Untuk membuka file teks yang Anda gunakan, well, open () function. Sepertinya masuk akal. Anda melewatkan parameter tertentu untuk membuka () untuk memberitahukannya di mana file harus dibuka - 'r' untuk dibaca saja, 'w' untuk tulisan saja (jika ada file lama, akan dituliskan), 'a 'Untuk menambahkan (menambahkan sesuatu ke akhir file) dan' r + 'untuk membaca dan menulis. 

\subsection{format file}
Teksnya benar-benar tidak diformat, tapi jika Anda melewati keluaran openfile.read () untuk mencetak (dengan mengetikkan print openfile.read () akan diformat dengan baik. Carilah dan Anda Temukan Apakah Anda mencoba mengetik di print openfile.read ()? Apakah itu gagal? Kemungkinan besar, dan alasannya adalah karena 'kursor' telah mengubah tempatnya. Kursor Kursor apa Nah, kursor yang sebenarnya tidak bisa kamu lihat, tapi tetap kursor. Kursor tak terlihat ini memberitahukan fungsi baca (dan banyak fungsi I / O lainnya) dari mana mulai. Untuk mengatur di mana kursor berada, Anda menggunakan fungsi seek ().

\subsection{Readlines}
Readlines () sama seperti readline (), namun readlines () membaca semua baris dari kursor dan seterusnya, dan mengembalikan sebuah daftar, dengan setiap elemen daftar memegang satu baris kode. Gunakan dengan fileobjectname.readlines (). 

\subsection{cursor}
menulis dari mana kursor berada, dan menimpa teks di depannya - seperti di MS Word, di mana Anda menekan 'insert' dan menulis di atas teks lama. Untuk memanfaatkan fungsi yang paling penting ini, letakkan string di antara tanda kurung untuk ditulis mis.


\subsection{pickles}
Pickles, dengan Python, adalah objek yang disimpan ke sebuah file. Objek dalam kasus ini bisa berupa variabel, instance dari kelas, atau daftar, kamus, atau tupel. Hal lain juga bisa acar, tapi dengan batas. Objek kemudian dapat dipulihkan, atau tidak dicemari, nanti. 



>>>>>>> master

