Python Functions

A function is a block of organized, reusable code that is used to perform a single, related action. Functions provide better modularity for your application and a high degree of code reusing.
As you already know, Python gives you many built-in functions like print(), etc. but you can also create your own functions. These functions are called user-defined functions.

Defining a Function
You can define functions to provide the required functionality. Here are simple rules to define a function in Python.
•	Function blocks begin with the keyword def followed by the function name and parentheses ( ( ) ).
•	Any input parameters or arguments should be placed within these parentheses. You can also define parameters inside these parentheses.
•	The first statement of a function can be an optional statement - the documentation string of the function or docstring.
•	The code block within every function starts with a colon (:) and is indented.
•	The statement return [expression] exits a function, optionally passing back an expression to the caller. A return statement with no arguments is the same as return None.


Syntax
	def functionname( parameters ):
   		"function_docstring"
   		function_suite
   		return [expression]
By default, parameters have a positional behavior and you need to inform them in the same order that they were defined.

Example
The following function takes a string as input parameter and prints it on standard screen.
	def printme( str ):
   		"This prints a passed string into this function"
   		print str
   		return

Calling a Function
Defining a function only gives it a name, specifies the parameters that are to be included in the function and structures the blocks of code.
Once the basic structure of a function is finalized, you can execute it by calling it from another function or directly from the Python prompt. Following is the example to call printme() function −
	#!/usr/bin/python

	# Function definition is here
	def printme( str ):
   		"This prints a passed string into this function"
   	print str
   	return;

	# Now you can call printme function
	printme("I'm first call to user defined function!")
	printme("Again second call to the same function")
When the above code is executed, it produces the following result −
	I'm first call to user defined function!
	Again second call to the same function

Pass by reference vs value
All parameters (arguments) in the Python language are passed by reference. It means if you change what a parameter refers to within a function, the change also reflects back in the calling function. For example −
	#!/usr/bin/python

	# Function definition is here
	def changeme( mylist ):
	   "This changes a passed list into this function"
   		mylist.append([1,2,3,4]);
   	print "Values inside the function: ", mylist
   	return
	
	# Now you can call changeme function
	mylist = [10,20,30];
	changeme( mylist );
	print "Values outside the function: ", mylist
Here, we are maintaining reference of the passed object and appending values in the same object. So, this would produce the following result −
	Values inside the function:  [10, 20, 30, [1, 2, 3, 4]]
	Values outside the function:  [10, 20, 30, [1, 2, 3, 4]]
There is one more example where argument is being passed by reference and the reference is being overwritten inside the called function.
	#!/usr/bin/python

	# Function definition is here
	def changeme( mylist ):
	   "This changes a passed list into this function"
	   mylist = [1,2,3,4]; # This would assig new reference in mylist
	   print "Values inside the function: ", mylist
	   return

	# Now you can call changeme function
	mylist = [10,20,30];
	changeme( mylist );
	print "Values outside the function: ", mylist
The parameter mylist is local to the function changeme. Changing mylist within the function does not affect mylist. The function accomplishes nothing and finally this would produce the following result:
	Values inside the function:  [1, 2, 3, 4]
	Values outside the function:  [10, 20, 30]

Function Arguments
You can call a function by using the following types of formal arguments:
•	Required arguments
•	Keyword arguments
•	Default arguments
•	Variable-length arguments

Required arguments
Required arguments are the arguments passed to a function in correct positional order. Here, the number of arguments in the function call should match exactly with the function definition.
To call the function printme(), you definitely need to pass one argument, otherwise it gives a syntax error as follows −
	#!/usr/bin/python

	# Function definition is here
	def printme( str ):
   		"This prints a passed string into this function"
   	print str
   	return;

	# Now you can call printme function
	printme()
When the above code is executed, it produces the following result:
	Traceback (most recent call last):
	  File "test.py", line 11, in <module>
	    printme();
	TypeError: printme() takes exactly 1 argument (0 given)
Keyword arguments
Keyword arguments are related to the function calls. When you use keyword arguments in a function call, the caller identifies the arguments by the parameter name.
This allows you to skip arguments or place them out of order because the Python interpreter is able to use the keywords provided to match the values with parameters. You can also make keyword calls to the printme() function in the following ways −
	#!/usr/bin/python

	# Function definition is here
	def printme( str ):
	   "This prints a passed string into this function"
	   print str
	   return;

	# Now you can call printme function
	printme( str = "My string")
When the above code is executed, it produces the following result −
	My string

The following example gives more clear picture. Note that the order of parameters does not matter.
	#!/usr/bin/python

	# Function definition is here
	def printinfo( name, age ):
	   "This prints a passed info into this function"
	   print "Name: ", name
	   print "Age ", age
	   return;

	# Now you can call printinfo function
	printinfo( age=50, name="miki" )
When the above code is executed, it produces the following result −
	Name:  miki
	Age  50

Default arguments
A default argument is an argument that assumes a default value if a value is not provided in the function call for that argument. The following example gives an idea on default arguments, it prints default age if it is not passed −
	#!/usr/bin/python

	# Function definition is here
	def printinfo( name, age = 35 ):
	   "This prints a passed info into this function"
	   print "Name: ", name
	   print "Age ", age
	   return;

	# Now you can call printinfo function
	printinfo( age=50, name="miki" )
	printinfo( name="miki" )
When the above code is executed, it produces the following result −
	Name:  miki
	Age  50
	Name:  miki
	Age  35

Variable-length arguments
You may need to process a function for more arguments than you specified while defining the function. These arguments are called variable-lengtharguments and are not named in the function definition, unlike required and default arguments.
Syntax for a function with non-keyword variable arguments is this −
	def functionname([formal_args,] *var_args_tuple ):
	   "function_docstring"
	   function_suite
	   return [expression]
An asterisk (*) is placed before the variable name that holds the values of all nonkeyword variable arguments. This tuple remains empty if no additional arguments are specified during the function call. Following is a simple example −
	#!/usr/bin/python

	# Function definition is here
	def printinfo( arg1, *vartuple ):
	   "This prints a variable passed arguments"
	   print "Output is: "
	   print arg1
	   for var in vartuple:
	      print var
	   return;
	
	# Now you can call printinfo function
	printinfo( 10 )
	printinfo( 70, 60, 50 )
When the above code is executed, it produces the following result −
	Output is:
	10
	Output is:
	70
	60
	50

The Anonymous Functions
These functions are called anonymous because they are not declared in the standard manner by using the def keyword. You can use the lambda keyword to create small anonymous functions.
•	Lambda forms can take any number of arguments but return just one value in the form of an expression. They cannot contain commands or multiple expressions.
•	An anonymous function cannot be a direct call to print because lambda requires an expression
•	Lambda functions have their own local namespace and cannot access variables other than those in their parameter list and those in the global namespace.
•	Although it appears that lambda's are a one-line version of a function, they are not equivalent to inline statements in C or C++, whose purpose is by passing function stack allocation during invocation for performance reasons.

Syntax
The syntax of lambda functions contains only a single statement, which is as follows −
	lambda [arg1 [,arg2,.....argn]]:expression

Following is the example to show how lambda form of function works −
	#!/usr/bin/python

	# Function definition is here
	sum = lambda arg1, arg2: arg1 + arg2;

 

	# Now you can call sum as a function
	print "Value of total : ", sum( 10, 20 )
	print "Value of total : ", sum( 20, 20 )
When the above code is executed, it produces the following result −
	Value of total :  30
	Value of total :  40

The return Statement
The statement return [expression] exits a function, optionally passing back an expression to the caller. A return statement with no arguments is the same as return None.
All the above examples are not returning any value. You can return a value from a function as follows −
	#!/usr/bin/python

	# Function definition is here
	def sum( arg1, arg2 ):
	   # Add both the parameters and return them."
	   total = arg1 + arg2
	   print "Inside the function : ", total
	   return total;

	# Now you can call sum function
	total = sum( 10, 20 );
	print "Outside the function : ", total 
When the above code is executed, it produces the following result −
	Inside the function :  30
	Outside the function :  30

Scope of Variables
All variables in a program may not be accessible at all locations in that program. This depends on where you have declared a variable.
The scope of a variable determines the portion of the program where you can access a particular identifier. There are two basic scopes of variables in Python −
•	Global variables
•	Local variables

Global vs. Local variables
Variables that are defined inside a function body have a local scope, and those defined outside have a global scope.
This means that local variables can be accessed only inside the function in which they are declared, whereas global variables can be accessed throughout the program body by all functions. When you call a function, the variables declared inside it are brought into scope. Following is a simple example −
	#!/usr/bin/python

	total = 0; # This is global variable.
	# Function definition is here
	def sum( arg1, arg2 ):
	   # Add both the parameters and return them."
	   total = arg1 + arg2; # Here total is local variable.
	   print "Inside the function local total : ", total
	   return total;
	
	# Now you can call sum function
	sum( 10, 20 );
	print "Outside the function global total : ", total 
When the above code is executed, it produces the following result −
	Inside the function local total :  30
	Outside the function global total :  0
	

