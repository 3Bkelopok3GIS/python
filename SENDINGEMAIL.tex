\section {Server Pada Mail Server dan Penjelasannya} 
Pada mail server terdapat 2 server yang berbeda yaitu :  
\begin{enumerate}
\item Outgoing Server (Sending email) : Protocol server yang menangani adalah SMTP (Simple Mail Transfer Protocol) pada port 25. 
\item Incoming Server (Receiving email) : Protocol server yang menangani adalah POP3 (Post Office Protocol) pada port 110 atau IMAP (Internet Message Access Protocol) pada port 143.
 \end{enumerate}
Penjelasan dari Server yang menangani outgoing email dan incoming email sebagai berikut 
\begin{enumerate}
\item SMTP Server : Saat anda mengirimkan email maka email anda akan ditangani SMTP Server dan akan dikirim ke SMTP Server tujuan, baik secara langsung maupun melalui beberapa SMTP Server dijalurnya. Apabila server tujuan terkoneksi maka email akan dikirim, namun apabila tidak terjadi koneksi maka akan dimasukan ke dalam queue dan di resend setiap 15 menit, apabila dalam 5 hari tidak ada perubahan maka akan diberikan undeliver notice ke inbox pengirim. 
\item POP3 Server : Jika menggunakan POP3 Server, apabila kita akan membaca email maka email pada server di download sehingga email hanya akan ada pada mesin yang mendownload email tersebut (kita hanya bisa membaca email tersebut pada device yang mendownload email tersebut). 
\item IMAP Server : Jika menggunakan IMAP Server, email dapat dibuka kembali lewat device yang berbeda. Fungsinya adalah mengelola email yang disimpan di server, kemudian email tersebut di ambil oleh client, selain itu IMAP juga meneruskan packet data. Kemampuan ini jauh lebih baik daripada POP (Post Office Protocol) yang hanya memperbolehkan kita mengambil/download semua pesan yang ada tanpa kecuali. IMAP adalah suatu  protokol yang umum digunakan untuk pengiriman surat elektronik atau email di Internet. Protokol ini gunakan untuk mengirimkan data dari komputer pengirim surat elektronik ke server surat elektronik penerima. Untuk menggunakan SMTP bisa dari Microsoft Outlook. biasanya untuk menggunakan SMTP di perlukan settingan :
\end{enumerate} 
\begin{enumerate}
\item Email Address : contoh; 
\begin{enumerate}
\item anda@domainanda.com 
\item Incoming Mail (POP3, IMAP or HTTP) server : mail.domainanda.com 
\item Outgoing (SMTP) server : mail.domainanda.com 
\item Account Name : anda@domainanda.com 
\item Password : password yang telah anda buat sebelumnya
\end{enumerate} 
Pada ilustrasi diatas Siti memiliki alamat email siti@a.id menulis email nya di komputer menggunakan Thunderbird atau Evolution. Pada kolom To: dia ketikkan alamat tujuan yaitu hendra@b.id. Setelah siti menekan tombol send, email yang dikirim langsung menuju ke mesin SMTP server milik ISP 1 yang bernama smtp.a.id
Pada server smtp.a.id menerima email dari siti (siti@a.id) yang ditujukan kepada hendra (hendra@b.id). Server mengecek smtp.a.id mencek alamat email tujuan yaitu hendra.@b.id. Mesin server smtp.a.id membutuhkan informasi ke server mana email untuk mesin.b.id harus ditujukan. Untuk memperoleh informasi tersebut tentang domain b.id. 
Kemudian pada mesin Name Server ns.b.id memberitahukan mesin smtp.a.id bahwa semua email yang ditujukan kepada b.id harus dikirim kepada mesin smtp.b.id.Setelah memperoleh jawaban dari ns.c.id bahwa email harus dikirm ke mesin smtp.b.id maka mesin smtp.a.id berusaha untuk menghubungi mesin smtp.b.id.Setelah mesin smtp.b.id berhasil dihubungi, mesin smtp.a.id mengirimkan teks email dari Siti (siti@a.id) yang ditujukan kepada Hendra(henra@b.id) ke mesin smtp.b.id 
Hendra (hendra@b.id)yang sedang menjalankan perangkat lunak pembaca email dan mengambil email tersebut dari eail server smtp.b.id barulah email dari Siti (siti@a.id) dapat diunduh melalui PC hendra dan di tampilkan isi emailnya. 
E-mail disampaikan oleh mail client (MUA, mail user agent) ke mail server (MSA, mail submission agent) menggunakan SMTP pada port 587 atau menggunakan traditional port 25. Dari sini, MSA mengirim mail tersebut ke mail transfer agent miliknya (MTA, mail transfer agent). MTA batas harus menemukan host target, dengan menggunakan DNS untuk mencari mail exchange record (MX record) untuk domain penerima. MX record yang kembali berisi nama dari host target. MTA selanjutnya menghubungkan ke exchange server sebagai SMTP client. Ketika MX target menerima pesan yang masuk, akan ditangani oleh mail delivery agent (MDA) untuk pengiriman pesan secara local. 
Analisis: Saat PC siti diberi perintah mengirim email ke PC Hendra, kemudian email tersebut terlebih dahulu masuk ke server network dimana dia berada server 1(smtp.a.id), disini server dapat melakukan kegiatan sniffing, Pada server sebelumnya sudah saling terkoneksi dan mendapat authentifikasi dari antar server untuk meneruskan paket email yang akan dikirim protokol yang bekerja pada tahap ini adalah SMTP, kemudian email masuk pada server2 (smtp.b.id).Untuk selanjutnya email dikirim ke PC Hendra (PC Destination) pada tahap ini protokol yang bekerja adalah protokol IMAP. Sehingga dari ilustrasi yang diberikan dapat menggambarkan proses pengiriman email,dan apa saja yang terjadidalam prosesnya. 
Pada proses pengiriman email terjadi kegiatan sniffing yang dilakukan oleh server. Sniffing adalah kegiatan pengendusan traffic data packet pada suatu jaringan. 
Selain itu Prinsip kerja dan Porses Pengiriman Email, email juga dibedakan berdasarkan format isinya, yakni sebagai berikut
\begin{itemize} 
\item Plain Text Email adalah jenis email yang sisanya diformat menggunakan sistem America Standart Code for Information Interchange (ASCII). Tulisan yang dibuat dengan format ini tidak dapat dimodifikasi seperti warna, ukuran jenis font dan lain sebagainya, Tidak ada pengolahan atau penambahan aksesoris. 
\item HTML Email adalah bahasa standar yang digunakan untuk mengatur tampilan informasi di Internet. Email yang menggunakan format ini umumnya dapat disesuaikan dengan selera pengirimnya, Dengan begitu email tersebut dapat ditambahkan macam-macam aksesoris seperti; penggantian jenis font, warna font dan juga besaran font pada tiap bagian surat.
\end{itemize}
\subsection {Apa Itu Port ?} 
Port adalah socket atau jack koneksi yang terletak di luar unit sistem sebagai tempat kabel - kabel yang berbeda ditancapkan. Port berfungsi untuk mentransmisikan data. Berikut macam - macam port : 
\begin{enumerate}
\item Port Serial adalah port seri merupakan sebuah port pada personal computer yang berfungsi untuk mentransmisikan satu bit informasi pada satu satuan waktu. 
\item Port Pararel adalah sebuah port pada personal computer yang berfungsi sebagai alat komunikasi komputer (motherboard) dengan perangkat luar yang bersifat paralel. 
\item Port SCSI (Scuzzy) adalah port berkinerja tinggi yang didefinisikan oleh American National Standart Institute yang digunakan untuk menangani perangkat input/output atau perangkat media penyimpanan. 
\item Port USB adalah suatu teknologi yang memungkinkan untuk menghubungkan alat eksternal (peripheral) seperti scenner, printer, mouse, dan perangkat lainnya ke komputer. \par
\end{enumerate}
\subsection {Cara Kerja Mail Server (singkat)} 
Cara kerja mail server mempunyai berbagai macam versi penjelasan mengenai cara kerjanya, dalam artikel ini saya akan menjelaskan 2 versi cara kerja mail server yang sudah saya rangkum dari berbagai sumber. Sebenarnya cara kerja antara versi 1 dan 2 mempunyai inti yang sama, hanya saja penjelasannya yang beda, silahkan anda pilih yang mana.\par  
\begin{itemize}
\item Cara Kerja Mail Server Versi 1 :
Proses pengiriman e-mail malalui tahapan yang sedikit panjang. Saat e-mail di kirim, maka e-mail tersebut disimpan pada mail server menjadi satu file berdasarkan tujuan e-mail. File ini berisi informasi sumber dan tujuan, serta dilengkapi tanggal dan waktu pengiriman. Pada saat user membaca e-mail berarti user telah mengakses server e-mail dan membaca file yang tersimpan dalam server yang di tampilkan melalui browser user 
\item Cara Kerja Mail Server Versi 2:
Cara kerja ini saya ambil dari Xmodulo, sebelum memahami proses cara kerja mail server sebaiknya anda mengenal terlebih dahulu singkatan - singkatan dari MUA, MTA, MDA dll. Berikut penjelasannya : 
\begin{enumerate}
\item Mail User Agent (MUA) : MUA adalah komponen yang berinteraksi dengan pengguna akhir secara langsung. Contoh dari MUA yaitu Thunderbird, MS Outlook, Zimbra Desktop. Interface webmail seperti Gmail. 
\item Mail Transfer Agent (MTA) : MTA bertanggung jawab untuk mentransfer email dari mail server mengirimkan sampai ke server penerima email. Contoh MTA yaitu sendmail dan postfix 
\item Mail Delivery Agent (MDA) : Dalam surat server tujuan, MTA lokal menerima email masuk dari MTA yang paling jauh. Email tersebut kemudian dikirimkan ke kotak surat pengguna dengan MDA. 
\item POP / IMAP : POP dan IMAP adalah protokol yang digunakan untuk mengambil email dari kotak surat penerima server untuk penerima MUA. 
\item Mail Exchanger Record (MX) : Record MX adalah entri DNS untuk mail server. Catatan ini menunjuk ke alamat IP ke arah mana email harus ditembak. MX record terendah selalu menang, yaitu mendapat prioritas tertinggi. Sebagai contoh, MX 10 adalah lebih baik daripada MX 20, alamat IP dari MX record dapat bervariasi berdasarkan desain dan konfigurasi persyaratan. seperti yang akan dibahas nanti dalam artikel.
Ketika pengirim mengklik tombol kirim, SMTP (MTA) memastikan ujung ke ujung pengiriman email dari pengirim-sisi server ke server tujuan. Setelah mencapai server tujuan, MTA lokal ke server tujuan menerima email, dan di pindahkan ke MDA setempat. MDA kemudian menulis email ke kotak pesan penerima. Ketika penerima memeriksa email, mereka diambil oleh MUA dengan menggunakan protokol seperti POP atau IMAP. 
\end{enumerate}

