%%%%%%%%%%%%%%%%%%%%%%%%%%%%%%%%%%%%%%%%%%%%%%%%%%%
%% LaTeX book template                           %%
%% Author:  Amber Jain (http://amberj.devio.us/) %%
%% License: ISC license                          %%
%%%%%%%%%%%%%%%%%%%%%%%%%%%%%%%%%%%%%%%%%%%%%%%%%%%

\documentclass[a4paper,11pt]{book}
\usepackage[T1]{fontenc}
\usepackage[utf8]{inputenc}
\usepackage{lmodern}
%%%%%%%%%%%%%%%%%%%%%%%%%%%%%%%%%%%%%%%%%%%%%%%%%%%%%%%%%
% Source: http://en.wikibooks.org/wiki/LaTeX/Hyperlinks %
%%%%%%%%%%%%%%%%%%%%%%%%%%%%%%%%%%%%%%%%%%%%%%%%%%%%%%%%%
\usepackage{hyperref}
\usepackage{graphicx}
\usepackage[english]{babel}

%%%%%%%%%%%%%%%%%%%%%%%%%%%%%%%%%%%%%%%%%%%%%%%%%%%%%%%%%%%%%%%%%%%%%%%%%%%%%%%%
% 'dedication' environment: To add a dedication paragraph at the start of book %
% Source: http://www.tug.org/pipermail/texhax/2010-June/015184.html            %
%%%%%%%%%%%%%%%%%%%%%%%%%%%%%%%%%%%%%%%%%%%%%%%%%%%%%%%%%%%%%%%%%%%%%%%%%%%%%%%%
\newenvironment{dedication}
{
   \cleardoublepage
   \thispagestyle{empty}
   \vspace*{\stretch{1}}
   \hfill\begin{minipage}[t]{0.66\textwidth}
   \raggedright
}
{
   \end{minipage}
   \vspace*{\stretch{3}}
   \clearpage
}

%%%%%%%%%%%%%%%%%%%%%%%%%%%%%%%%%%%%%%%%%%%%%%%%
% Chapter quote at the start of chapter        %
% Source: http://tex.stackexchange.com/a/53380 %
%%%%%%%%%%%%%%%%%%%%%%%%%%%%%%%%%%%%%%%%%%%%%%%%
\makeatletter
\renewcommand{\@chapapp}{}% Not necessary...
\newenvironment{chapquote}[2][2em]
  {\setlength{\@tempdima}{#1}%
   \def\chapquote@author{#2}%
   \parshape 1 \@tempdima \dimexpr\textwidth-2\@tempdima\relax%
   \itshape}
  {\par\normalfont\hfill--\ \chapquote@author\hspace*{\@tempdima}\par\bigskip}
\makeatother

%%%%%%%%%%%%%%%%%%%%%%%%%%%%%%%%%%%%%%%%%%%%%%%%%%%
% First page of book which contains 'stuff' like: %
%  - Book title, subtitle                         %
%  - Book author name                             %
%%%%%%%%%%%%%%%%%%%%%%%%%%%%%%%%%%%%%%%%%%%%%%%%%%%

% Book's title and subtitle
\title{\Huge \textbf{Web Service}  \footnote{D4TI3D.} \\ \huge HOME \footnote{1144010.}}
% Author
\author{\textsc{Faisal Akbar Ramadhan}
}


\begin{document}
\maketitle

%%%%%%%%%%%
% Preface %
%%%%%%%%%%%
\chapter*{HOME}
1.	Pengenalan Web Service
Definisi Web Service
Web service adalah suatu sistem perangkat lunak yang dirancang untuk mendukung interoperabilitas dan interaksi antar sistem pada suatu jaringan. Web service digunakan sebagai suatu fasilitas yang disediakan oleh suatu web site untuk menyediakan layanan (dalam bentuk informasi) kepada sistem lain, sehingga sistem lain dapat berinteraksi dengan sistem tersebut melalui layanan-layanan (service) yang disediakan oleh suatu sistem yang menyediakan web service. Web service menyimpan data informasi dalam format XML, sehingga data ini dapat diakses oleh sistem lain walaupun berbeda platform, sistem operasi, maupun bahasa compiler.
Web service bertujuan untuk meningkatkan kolaborasi antar pemrogram dan perusahaan, yang memungkinkan sebuah fungsi di dalam Web Service dapat dipinjam oleh aplikasi lain tanpa perlu mengetahui detil pemrograman yang terdapat di dalamnya.
Beberapa alasan mengapa digunakannya web service  adalah sebagai berikut:
1. Web service dapat digunakan untuk mentransformasikan satu atau beberapa bisnis logic atau class dan objek yang terpisah dalam satu ruang lingkup yang menjadi satu, sehingga tingkat keamanan dapat ditangani dengan baik.
2. Web service memiliki kemudahan dalam proses deployment-nya, karena tidak memerlukan registrasi khusus ke dalam suatu sistem operasi. Web service cukup di-upload ke web server dan siap diakses oleh pihak-pihak yang telah diberikan otorisasi.
3. Web service berjalan di port  80 yang merupakan protokol standar HTTP, dengan demikian web service tidak memerlukan konfigurasi khusus di sisi firewall.
Arsitektur Web Service
Web service memiliki tiga entitas dalam arsitekturnya, yaitu:
1. Service Requester (peminta layanan)
2. Service Provider (penyedia layanan)
4. Service Registry (daftar layanan)
• Service Provider: Berfungsi untuk menyediakan layanan/service dan mengolah sebuah registry agar layanan-layanan tersebut dapat tersedia.
• Service Registry: Berfungsi sebagai lokasi central yang mendeskripsikan semua layanan/service yang telah di-register.
• Service Requestor: Peminta layanan yang mencari dan menemukan layanan yang dibutuhkan serta menggunakan layanan tersebut.
Operasi-Operasi Web Service
Secara umum, web service memiliki tiga operasi yang terlibat di dalamnya, yaitu:
1. Publish/Unpublish: Menerbitkan/menghapus layanan ke dalam atau dari registry.
2. Find: Service requestor mencari dan menemukan layanan yang dibutuhkan.
3. Bind: Service requestor setelah menemukan layanan yang dicarinya, kemudian melakukan binding ke service provider untuk melakukan interaksi dan mengakses layanan/service yang disediakan oleh service provider.
Komponen-Komponen Web Service
Web service secara keseluruhan memiliki empat layer komponen seperti pada gambar di atas, yaitu:
1. Layer 1: Protokol internet standar seperti HTTP, TCP/IP
2. Layer 2: Simple Object Access Protocol (SOAP), merupakan protokol akses objek berbasis XML yang digunakan untuk proses pertukaran data/informasi antar layanan.
3. Layer 3: Web Service Definition Language (WSDL), merupakan suatu standar bahasa dalam format XML yang berfungsi untuk mendeskripsikan seluruh layanan yang tersedia.
\section*{Web Service}
: Merupakan istilah yang mengacu pada aplikasi virtual atau terdistribusi atauproses yang menggunakan internet untuk menghubungkan aktivitas atau komponen perangkatlunak.
—Web Service merupakan arsitektur komputasi yang terdistribusi. Arsitektur ini  bertujuan untuk memungkinkan bermacam-macam aplikasi untuk saling komunikasi

Keuntungan yang didapat dalam menggunakan web service adalah semua aplikasi diduniadapat berkomunikasi satu dan lainnya . Komunikasi antar aplikasi ini tidak memiliki batasantempat, sistem operasi, bahasa pemrograman, protokol dan lain sebagainya.

Untuk berkomunikasi dengan Web Service komputer klien akan
mengirimkan pesan SOAP yang Mengandung pemanggilan pada  sebuah
method beserta parameter yang di butuhkan (oleh method tersebut).
Sebagai tambahan, pesan SOAP dapat juga mengandung sejumlah item
Header yang menjelaskan kebutuhan klien lebih lanjut. 

arsitektur web service
—Web service memiliki tiga entitas dalam arsitekturnya, yaitu:
—1.  Service Requester (peminta layanan)
—2.  Service Provider (penyedia layanan)
—3.  Service Registry (daftar layanan)
—Service Provider: Berfungsi untuk menyediakan layanan/service dan mengolah sebuahregistry agar layanan-layanan tersebut dapat tersedia.
—Service Registry: Berfungsi sebagai lokasi central yang mendeskripsikan semualayanan/service yang telah di-register.
—Service Requestor: Peminta layanan yang mencari dan menemukan layanan yangdibutuhkan serta menggunakan layanan tersebut.
operasi-operasi web service
—Secara umum, web service memiliki tiga operasi yang terlibat di dalamnya, yaitu:
—Publish/Unpublish: Menerbitkan/menghapus layanan ke dalam atau dari registry.
—Find: Service requestor mencari dan menemukan layanan yang dibutuhkan.
—Bind: Service requestor setelah menemukan layanan yang dicarinya, kemudian melakukanbinding ke service provider untuk melakukan interaksi dan mengakses layanan/service yangdisediakan oleh service provider.
komponen utama web service
—SOAP (Simple Object Access Protocol)
  SOAP merupakan spesifikasi yang mendefenisikan grammar XML untuk pesan yang akandikirimkan dan juga jawaban dari pesan tersebut. Tujuan dari SOAP adalah untukmendeskripsikan format sebuah pesan yang tidak bergantung pada perangkat keras danperangkat lunak apapun, melainkan SOAP dapat membawa pesan dari sebuah platform keplatform lainnya tanpa adanya ambiguitas. SOAP biasanya terdiri dari dua bagian : header yang membawa instruksi pemprosesan dan body yang mengandung informasi yang ingindisampaikan.
—Extensible Markup Language (XML)— Merupakan bahasa dimana semua web service dibangun. XML merupakan alat untuk membangun dokumen self-describing. Dalam XML kita dapat membuat sendiri tag-tag dan komponen grammar lainnya.Grammar-grammar ini di deskripsikan dalam skema XML (XML schema) yang menentukan tags yang di izinkan (untuk digunakan) dan hubungan antar element yang didefenisikan oleh tags tersebut.
Untuk pembahasan lebih lanjut, berikut saya coba jelaskan mengenai apa itu Web Service,  kegunaannya untuk apa.
Web service menurut W3.org mendefinisikan web service sebagai “sebuah software aplikasi yang dapat teridentifikasi oleh URI dan memiliki interface yang didefiniskan, dideskripsikan, dan dimengerti oleh XML dan juga mendukung interaksi langsung dengan software aplikasi yang lain dengan menggunakan message berbasis XML melalui protokol internet”.
Web service adalah sebuah sofware aplikasi yang tidak terpengaruh oleh platform, ia akan menyediakan method-method yang dapat diakses oleh network. Ia juga akan menggunakan XML untuk pertukaran data, khususnya pada dua entities bisnis yang berbeda.
Definisi lain : Web service  adalah sistem software yang dirancang untuk mendukung interopabilitas mesin-ke-mesin yang dapat berinteraksi melalui jaringan.  Web service memiliki antarmuka yang dijelaskan dalam format mesin-processable (khusus WSDL). Sistem lain berinteraksi dengan  Web service dalam cara ditentukan oleh deskripsi dengan menggunakan pesan SOAP, biasanya disampaikan menggunakan HTTP dengan serialisasi XML dalam hubungannya dengan Web lainnya yang terkait standar.


\section*{Manfaat}
Web service dapat digunakan sebagai salah satu alternatif dalam pengembangan aplikasi  N-tier, dimana dipisahkan antara server database, aplikasi dan client. Beberapa keuntungan lain yang didapat dari penerapan web service yaitu:
Dengan format XML yang telah menjadi salah satu standar pertukaran data, penggunaan web service akan banyak memudahkan untuk pertukaran data dalam berbagai sistem dengan berbeda platform. Apabila kita membuat web service dengan teknologi Java, maka fungsi-fungsi yang ada dalam web service tersebut dapat kita baca dengan menggunakan sistem lain yang berbeda sama sekali dari Java, misalkan menggunakan .Net ataupun PHP.
Web service di support oleh pemain utama dalam dunia TI seperti Microsoft (NET), SUN (Open Net Environment  – ONE), IBM (Web Service Conceptual Architecture  – WSCA), W3C (Web Service Workshop), Oracle (Web Service Broker), Hewlett-Packard (Web Service Platform).
Dalam penerapan  N-tier, untuk layer bisnis atau  apllication logic  dapat diterapkan dengan web service, sehingga di sisi client kita tidak direpotkan dengan instalasi layer bisnis seperti  halnya  dll, corba, atau jenis  yang lain. Dengan web service,  method atau function yang telah kita buat dapat dipergunakan berulang kali bahkan untuk keperluan aplikasi yang  berbeda (reusable function). Penerapan lebih jauh dari web service adalah  Service Oriented Architecture (SOA) dengan web service sebagai dasarnya.
Web service dibangun berdasarkan  text base document dengan format XML, sehingga untuk komunikasi data relatif lebih ringan dibandingkan dengan aplikasi yang mengakses langsung database melalui suatu jaringan. Apabila kita menerapkan web service untuk aplikasi yang menggunakan  desktop application  based, kita tidak perlu melakukan instalasi konektor database seperti misalnya menggunakan ODBC, OLEDB, ataupun jenis  data provider lain. Dengan jumlah client yang cukup banyak, tentunya akan sangat merepotkan apabila kita harus melakukan instalasi satu persatu untuk konektor database. Dengan menggunakan web service kita cukup menambahkan  web service reference  di client, sedangkan untuk koneksi databasenya hanya perlu dilakukan di server web servicenya.
Komunikasi data melalui web service dilakukan melalui  http  atau  Internet protocol  terbuka lainnya. Hal ini sangat memudahkan karena  protocol tersebut adalah protocol yang umum dipakai.

Dalam pengertian yang sederhana , XML Web Services dapat di definisikan sebagai aplikasi yang diakses oleh aplikasi yang lain. Mungkin orang berpendapat itu semacam web site, tetapi itu bukan demikian. Ada perbedaan – perbedaan yang membedakan dengan web site.
Web service sendiri dibentuk dari :
Service provider, merupakan pemilik Web Service yang berfungsi menyediakan kumpulan operasi dari Web Service.
Service requestor, merupakan aplikasi yang bertindak sebagai klien dari Web Service yang mencari dan memulai interaksi terhadap layanan yang disediakan.
Service registry, merupakan tempat dimana Service provider mempublikasikan layanannya. Pada arsitektur Web Service, Service registry bersifat optional. Teknologi web service memungkinkan kita dapat menghubungkan berbagai jenis software yang memiliki platform dan sistem operasi yang berbeda.
Beberapa karakteristik dari web service adalah:
Message-based
Standards-based
Programming language independent
Platform-neutral
KEY STANDARD DALAM WEB SERVICE
Beberapa key standard didalam web service adalah: XML, SOAP, WSDL and UDDI.
SOAP (Simple Object Access Protocol) adalah sebuah XML-based mark-up language untuk pergantian pesan diantara aplikasi-aplikasi. SOAP berguna seperti sebuah amplop yang digunakan untuk pertukaran data object didalam network. SOAP mendefinisikan empat aspek didalam komunikasi: Message envelope, Encoding, RPC call convention, dan bagaimana menyatukan sebuah message didalam protokol transport.
Sebuah SOAP message terdiri dari SOAP Envelop dan bisa terdiri dari attachments atau tidak memiliki attachment. SOAP envelop tersusun dari SOAP header dan SOAP body, sedangkan SOAP attachment membolehkan non-XML data untuk dimasukkan kedalam SOAP message, di-encoded, dan diletakkan kedalam SOAP message dengan menggunakan MIME-multipart.
WSDL (Web Services Description Language) adalah sebuah XML-based language untuk mendeskripsikan XML. WSDL menyediakan service atau layanan yang mendeskripsikan service request dengan menggunakan protokol-protokol yang berbeda dan juga encoding. WSDL memfasilitasi komunikasi antar aplikasi. WSDL akan mendeskripsikan apa yang akan dilakukan oleh web service, bagaimana menemukannya dan bagaimana untuk mengoperasikannya.
Spesifikasi WSDL mendefinisikan tujuh tipe element:
Types – element untuk mendefinisikan tipe data. Mereka akan mendefinisikan tipe data (seperti string atau integer) dari element didalam sebuah message.
Message – abstract, pendefinisian tipe data yang akan dikomunikasikan.
Operation – sebuah deskripsi abstract dari sebuah action yang didukung oleh service.
Port Type – sebuah koleksi abstract dari operations yang didukung oleh lebih dari satu endpoints.
Binding – mendefinisikan penyatuan dari tipe port (koleksi dari operasioperasi) menjadi sebuah protokol transport dan data format (ex. SOAP 1.1 pada HTTP). Ini adalah sebuah protokol konkret dan sebuah spesifikasi data format didalam tipe port tertentu.
Port – mendefinisikan sebuah komunikasi endpoint sebagai kombinasi dari binding dan alamat network. Bagi protokol HTTP,  sebuah bentuk dari URL sedangkan bagi protokol SMTP, ini adalah sebuah form dari email address.
Service – satu set port yang terkorelasi atau suatu endpoints.
WSDL mendefinisikan service sebagai sebuah koleksi dari endpoints network. Sebuah definisi abstrak dari endpoints dan messages adalah ia bersifat terpisah dari pembangunan network atau penyatuan data format. Pembagian ini menyebabkan penggunaan kembali abstract description dari data yang akan dipertukarkan (message exchange) dan abstract collection dari operasi (ports) Protokol konkret dan spesfikasi data format bagi tipe port tertentu menentukan binding yang dapat digunakan kembali(reusable). Sebuah port adalah sebuah network address yang dikombinasikan reusable binding; sebuah service adalah koleksi dari port-port.
Sedangkan UDDI (Universal Description, Discovery and Integration) adalah sebuah service registry bagi pengalokasian web service. UDDI mengkombinasikan SOAP dan WSDL untuk pembentukan sebuah registry API bagi pendaftaran dan pengenalan service. Ia menyediakan sebuah area umum dimana sebuah organisasi dapat mengiklankan keberadaan mereka dan service yang diberikan (web service).
Semantik pada Web service adalah harapan bersama tentang perilaku layanan, khususnya dalam menanggapi pesan yang dikirim ke tujuan. Akibatnya, ini adalah “kontrak” antara entitas pemohon dan badan penyedia tentang tujuan dan konsekuensi dari interaksi. Meskipun kontrak ini merupakan keseluruhan perjanjian antara entitas penanya dan entitas penyedia tentang bagaimana dan mengapa masing-masing agen akan berinteraksi, itu belum tentu tertulis atau eksplisit dinegosiasikan. Ini mungkin eksplisit atau implisit, lisan atau tertulis, mesin processable atau manusia berorientasi, dan mungkin suatu perjanjian hukum atau kesepakatan informal (non-hukum).
Ada banyak cara bahwa entitas penanya mungkin terlibat dan menggunakan Web service. Secara umum, langkah-langkah yang luas berikut yang diperlukan, seperti yang diilustrasikan pada Gambar 1.  (1) pemohon dan penyedia entitas menjadi dikenal satu sama lain (atau setidaknya satu menjadi tahu untuk yang lain); (2) peminta dan penyedia entitas entah bagaimana setuju pada deskripsi layanan dan semantik yang akan mengatur interaksi antara pemohon dan agen penyedia; (3) deskripsi layanan dan semantik direalisasikan oleh pemohon dan agen penyedia, dan (4) pemohon dan agen penyedia bertukar pesan, sehingga melakukan beberapa tugas atas nama pemohon dan badan penyedia. (Ie, pertukaran pesan dengan agen penyedia merupakan wujud nyata dari berinteraksi dengan layanan Web penyedia entitas.)
\section{Keamanan}
Keamanan pada Web service menjadi sebuah keunikan karena interaksi yang terjadi pada Web service bukan interaksi antara manusia dan program, melainkan merupakan interaksi antara program dan program. Maka keamanan di sini adalah keamanan seperti pengontrolan akses, autentikasi, keamanan data, dan privasi. Skema keamanan yang paling umum akhir-akhir ini adalah SSL (Secure Sockets Layer), tetapi ketika diterapkan pada teknologi Web service maka akan muncul banyak keterbatasan dari SSL. Maka dari itu teknologi Web service sudah mulai bergerak menuju skema keamanan yang XML-based. Seperti XML Encryption, XKMS (XML Key Management Specification), SAML (Secure Assertion Markup Language), WS-Security (Web Services Security), atau ebXML Message Service.
Mengingat web service dibuat dengan memanfaatkan protocol http, maka web service memiliki kerentanan yang sama seperti halnya website biasa. Hal ini dapat diatasi dengan memperhatikan aspek-aspek keamanan pada saat kita membuat web service. Aspek-aspek tersebut antara lain :
Authentication : menggunakan Public Key Infrastructure, atau active directory
Authorization : membatasi hak control akses terhadap data
Confidentiality : melakukan enkripsi pada isi message
Data Integrity  :  menerapkan  Secure  Security Layer/SSL  pada  saat  proses komunikasi data dalam jaringan
Non-Repudiation : menggunakan teknologi digital signature dan timestamping dan menerapkan audit log dalam setiap transaksi.
Web service adalah aplikasi sekumpulan data (database), perangkat lunak (software) atau bagian dari perangkat lunak yang dapat diakses secara remote oleh berbagai piranti dengan sebuah perantara tertentu.  Secara umum,web service  dapat diidentifikasikan dengan menggunakan URL seperti hanya web pada umumnya. Namun yang membedakan web service dengan web pada umumnya adalah interaksi yang diberikan oleh web service. Berbeda dengan URL web pada umumnya, URL web service hanya menggandung kumpulan informasi, perintah, konfigurasi atau sintaks yang berguna membangun sebuah fungsi-fungsi tertentu dari aplikasi.
Web service dapat diartikan juga sebuah metode pertukaran data, tanpa memperhatikan dimana sebuahdatabase ditanamkan, dibuat dalam bahasa apa sebuah aplikasi yang mengkonsumsi data, dan di platform apa sebuah data itu dikonsumsi. Web service mampu menunjang interoperabilitas. Sehingga web service mampu menjadi sebuah jembatan penghubung antara berbagai sistem yang ada.
Menurut W3C Web services Architecture Working Group pengertian Web service adalah sebuah sistem softwareyang di desain untuk mendukung interoperabilitas interaksi mesin ke mesin melalui sebuah jaringan. Interfaceweb service dideskripsikan dengan menggunakan format yang mampu diproses oleh mesin (khususnya WSDL). Sistem lain yang akan berinteraksi dengan web service hanya memerlukan SOAP, yang biasanya disampaikan dengan HTTP dan XML sehingga mempunyai korelasi dengan standar Web (Web Services Architecture Working Group, 2004).
Web pada umumnya digunakan untuk melakukan respon dan request yang dilakukan antara client dan server. Sebagai contoh, seorang pengguna layanan web tertentu mengetikan alamat url web untuk membentuk sebuahrequest. Request akan sampai pada server, diolah dan kemudian disajikan dalam bentuk sebuah respon. Dengan singkat kata terjadilah hubungan client-server secara sederhana.

\end{document}
