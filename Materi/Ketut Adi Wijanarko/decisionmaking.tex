Python Decision Making
\begin{flushleft}
•
\end{flushleft}Pengambilan keputusan adalah antisipasi kondisi yang terjadi saat pelaksanaan program dan menentukan tindakan yang dilakukan sesuai kondisi.
\begin{flushleft}
•
\end{flushleft}Struktur keputusan mengevaluasi banyak ekspresi yang menghasilkan TRUE atau FALSE sebagai hasil. Anda perlu menentukan tindakan mana yang harus diambil dan pernyataan mana yang akan dijalankan jika hasilnya BENAR atau SALAH sebaliknya.
Berikut adalah bentuk umum dari struktur pengambilan keputusan yang khas yang ditemukan di sebagian besar bahasa pemrograman 
\begin{flushleft}
•
\end{flushleft}Bahasa pemrograman Python mengasumsikan nilai non-nol dan non-nullsebagai TRUE, dan jika itu adalah nol atau nol , maka diasumsikan sebagai nilai FALSE.
\begin{flushleft}
•
\end{flushleft}Bahasa pemrograman Python menyediakan jenis pernyataan pengambilan keputusan berikut. Klik link berikut untuk memeriksa detailnya.

\begin{flushleft}
•
\end{flushleft}jika pernyataan
Sebuah pernyataan jika terdiri dari ekspresi boolean diikuti oleh satu atau lebih pernyataan.
\begin{flushleft}
•
\end{flushleft}Jika ... pernyataan lain
\begin{flushleft}
•
\end{flushleft}Sebuah pernyataan jika dapat diikuti oleh opsional lain pernyataan , yang mengeksekusi ketika ekspresi boolean adalah palsu.
\begin{flushleft}
•
\end{flushleft}Bersarang jika pernyataan
\begin{flushleft}
•
\end{flushleft}Anda bisa menggunakan satu jika atau jika adapernyataan di dalam pernyataan lain jika atau jika ada pernyataan.

\begin{flushleft}
•
\end{flushleft}Suite pernyataan tunggal
Jika rangkaian klausa jika hanya terdiri dari satu baris, itu mungkin sama pada baris perintah sebagai pernyataan header.
